\documentclass[../notes.tex]{subfiles}

\pagestyle{main}
\renewcommand{\chaptermark}[1]{\markboth{\chaptername\ \thechapter\ (#1)}{}}
\setcounter{chapter}{1}

\begin{document}




\chapter{???}
\section{Lecture 4: Substitution Reactions}
% Go through and delete stuff/cite other notes (reread swaths of Labalme (2021)). Get rid of questions and reformulate as statements.
\begin{itemize}
    \item \marginnote{4/5:}Association/dissociation reactions.
    \item Fairly related to organic S\textsubscript{N}2 and S\textsubscript{N}1 reactions, respectively.
    \item General form:
    \begin{equation*}
        \ce{ML6 + L$'$ <=> ML5L$'$ + L}
    \end{equation*}
    \item We investigate the position of the equilibrium with the three main characteristics that determine reactivity.
    \begin{enumerate}
        \item Sterics.
        \begin{itemize}
            \item Related to the metal coordination number.
            \begin{itemize}
                \item $\text{C.N.}>6$ is typically disfavored.
                \item $\text{C.N.}<6$ is possible.
            \end{itemize}
            \item The size of \ce{L$'$} is also important: If $\ce{L$'$}=\ce{PPh3}$ for example, this is hard to get to $\text{C.N.}>4$.
        \end{itemize}
        \item Ligand character.
        \begin{itemize}
            \item In nonpolar media, dissociation of charged groups (e.g., \ce{Cl-}) will be disfavored. However, the opposite is true in polar media.
            \begin{itemize}
                \item This is because of the issue of making charge/ionizing.
            \end{itemize}
            \item The match between \ce{M} and \ce{L} (e.g., hard/soft, electron rich/poor) is also important.
            \begin{itemize}
                \item For example, \ce{Fe^0} will bind \ce{CO} strongly since \ce{Fe^0} is electron rich and \ce{CO} is a $\pi$ acceptor.
                \item However, \ce{Fe^{IV}} will not (as a hard, electron-poor metal center).
            \end{itemize}
        \end{itemize}
        \item Electronic structure of the metal center (whether or not the metal is electronically saturated [has 18 electrons]).
        \begin{itemize}
            \item $18\,\e[-]$: it will not want to coordinate an additional \ce{L$'$}.
            \item $20\,\e[-]$: it will want to dissociate.
            \item $16\,\e[-]$: it \emph{can} associate.
            \begin{itemize}
                \item However, it may not want to given that $16\,\e[-]$ square-planar complexes are fairly stable.
                \item The associated state may be a transition state in a square-planar ligand substitution or otherwise not a ground state.
            \end{itemize}
        \end{itemize}
    \end{enumerate}
    \item Ligand substitution reactions terms: \textbf{Kinetic} and \textbf{thermodynamic}.
    \item \textbf{Kinetic} (considerations): Elements are inert (slow) or labile (fast).
    \item \textbf{Thermodynamic} (considerations): Which side of an equilibrium will be favored. Elements are stable or reactive.
    \item In ligand substitution reactions, there are two limiting regimes:
    \begin{enumerate}
        \item Associative substitution.
        \begin{itemize}
            \item See the related discussion in \textcite{bib:CHEM20100Notes}.
            \item This is the most general reaction type, even for coordinatively saturated complexes.
            \item Rate law:
            \begin{equation*}
                \dv{\ce{[ML5L$'$]}}{t} = k_\text{obs}\ce{[ML6][L$'$]}
            \end{equation*}
        \end{itemize}
        \item Dissociative mechanism.
        \begin{itemize}
            \item See the related discussion in \textcite{bib:CHEM20100Notes}.
            \item There are many things that look dissociative that are associative (e.g., instead of forming a 5-coordinate species, you could just have a molecule of the solvent displace a ligand).
            \item This mechanism is rare and hard to prove.
            \item Rate law:
            \begin{equation*}
                \dv{\ce{[ML5L$'$]}}{t} = \frac{k_2k_1\ce{[ML6][L]}}{k_{-1}\ce{[L]}+k_2\ce{[L$'$]}}
            \end{equation*}
            \item Experimentally, we swamp the reaction with \ce{L$'$} so that $\ce{[L$'$]}>>>$ than all other reagents. This makes it so that the rate is just $k_\text{obs}\ce{[ML6]}$, i.e., pseudo-first order conditions.
        \end{itemize}
    \end{enumerate}
    \item Unfortunately, much like in orgo, very few cases are at these extremes and we can have hybrids called\dots
    \begin{enumerate}[resume]
        \item Interchange mechanisms.
        \begin{itemize}
            \item See the related discussion in \textcite{bib:CHEM20100Notes}.
            \item Within this category, we can have $I_a$ (associative interchange) and $I_d$ (dissociative interchange).
            \item In the transition state, we have \ce{L$'$} coming in and \ce{L} leaving at the same time.
        \end{itemize}
    \end{enumerate}
    \item Kinetics and rates of these mechanisms.
    \item Several categories (measure with water exchange rates; see \textcite{bib:CHEM20100Notes}):
    \begin{enumerate}[label={\Roman*)}]
        \item Very fast.
        \begin{itemize}
            \item Alkali metals (species that primarily engage in ionic bonding; little covalent character).
            \item $\SI{e8}{\per\second}$; close to the diffusion limit.
        \end{itemize}
        \item Fast.
        \begin{itemize}
            \item Higher valent ions; often \ce{M^3+} such as \ce{Al^3+}.
            \begin{itemize}
                \item Higher charge $\Rightarrow$ higher ligand affinity $\Rightarrow$ slightly slower but still pretty fast.
            \end{itemize}
            \item $\num{e3}$-$\SI{e8}{\per\second}$.
        \end{itemize}
        \item Slower.
        \begin{itemize}
            \item Getting into the transition metals: \ce{Fe^3+}, \ce{V^3+}, \ce{Ti^3+}.
            \begin{itemize}
                \item $d$-orbital splitting $+$ covalency $\Rightarrow$ stronger bonding $\Rightarrow$ slower exchange rate.
            \end{itemize}
            \item $\num{e1}$-$\SI{e4}{\per\second}$.
        \end{itemize}
        \item Inert.
        \begin{itemize}
            \item \ce{Co^3+}, \ce{Cr^3+}, \ce{Pt^2+}, and \ce{Fe^2+}(L.S.).
            \item $\num{e-8}$-$\SI{e-4}{\per\second}$.
        \end{itemize}
    \end{enumerate}
    \item The overlap between the rates reflects the fact that there is no hard and fast cut off between categories.
    \item The identity of \ce{L$'$} also influences rates.
    \begin{itemize}
        \item Reaction rates increase with the ligand field strength of \ce{L$'$}\footnote{Goes over Table IX.1 from \textcite{bib:CHEM20100Notes}.}.
    \end{itemize}
    \item Characteristics of the metal that control the observed rates.
    \begin{itemize}
        \item Ranking L.S. metal centers (slowest to fastest): $\ce{Co^{III}}<\ce{Cr^{III}}<\ce{Mn^{III}}<\ce{Fe^{III}}<\ce{Ti^{III}}<\ce{V^{III}}$.
        \item Considering the $d$ counts, we have $d^6<d^3<d^4<d^5<d^1<d^2$.
        \item Now think of this in terms of the $d$-orbitals splitting diagram (Figure \ref{fig:6-octab}).
        \begin{itemize}
            \item As the antibonding orbitals get filled, $\sigma$ bonds will weaken, promoting a faster exchange.
            \item Full and half-full $t_{2g}$ also provides stability.
        \end{itemize}
    \end{itemize}
    \item Thus, we list the following configurations as inert and labile (see the related discussion in \textcite{bib:CHEM20100Notes}):
    \begin{itemize}
        \item Inert: $d^3$, L.S. $d^{4,5,6}$, and square planar $d^8$.
        \item Labile: $d^0$, $d^1$, $d^2$, H.S. $d^{4,5,6}$, $d^7$, $d^9$, $d^{10}$.
    \end{itemize}
    \item Other important kinetic factors:
    \begin{enumerate}
        \item Oxidation state.
        \begin{itemize}
            \item As oxidation state increases, exchange rate decreases (becomes more inert).
        \end{itemize}
        \item Size.
        \begin{itemize}
            \item Smaller ions are more inert.
            \item However, first row ions are almost always labile (because they more readily populate higher spin states).
        \end{itemize}
        \item Chelate effect.
        \begin{itemize}
            \item Reviews some info from \textcite{bib:CHEM20100Notes}.
            \item Chelating ligands form a ring or a \textbf{metallacycle} (this is why 4,5-membered ligands are stable; because 5,6-membered rings are favorable).
            \item Binding of a chelating ligand is typically favored, primarily due to entropic reasons (effective concentration is secondary).
            \item Example: Gives actual $\Delta G=\Delta H-T\Delta S$ thermodynamic data for the formation reaction of \ce{Cu(MeNH2)4^2+} vs. \ce{Cu(en)2^2+} to emphasize the importance of entropy (see the related discussion in the notes on Chapter 10 in \textcite{bib:CHEM20100Notes}).
            \item EDTA is a hexadentate ligand that is commonly used in biology to pull all metal centers out of solution.
            \begin{itemize}
                \item For \ce{Fe^3+} for example, $K_f=\SI{e25}{\per\mole}$. What is $\si{\per\mole}$ and why is it here?
                \item Sidenophones and euterobactin are biology's own chelaters ($K_f=\SI{e52}{\per\mole}$).
                \item These chelaters involved because if bacteria are going to invade a host, they need to scavenge iron, but iron is pretty tightly regulated. Thus, there has been an arms race of molecules that can scavenge iron or prevent iron from being scavenged.
            \end{itemize}
            \item Chelation therapy: If exposed to a heavy metal, you will be given chelating agents that will bind to metal ions and cause them to be excreted from the body.
        \end{itemize}
        \item Trans effect.
        \begin{itemize}
            \item Reviews some info from \textcite{bib:CHEM20100Notes}.
            \item Helps predict the \textbf{regiochemistry} of where a given ligand will substitute.
            \item Cis-platin reaction mechanism: \ce{{\emph{cis}-}Pt(NH3)2(Cl)2 -> {\emph{cis}-}Pt(NH3)2(H2O)2} in the body, which binds to DNA on the \emph{cis}-water side, causing a kink, stopping transcription, and initiating apoptosis.
            \begin{itemize}
                \item Cis-platin is quite toxic (people are trying to develop formulations that are less so), but highly effective at stopping cancer.
                \item Can't have \emph{trans} because it doesn't have the \emph{cis}-water side. Thus, this synthesis mechanism doesn't work: \ce{[PtCl4]^2- ->[2NH3] {\emph{trans}-}Pt(NH3)2Cl2}.
                \item Therefore, we synthesize it as follows.
                \begin{align*}
                    \ce{K2PtCl4} &\ce{->[4KI]} \ce{PtI4^2-}\\
                    &\ce{->[2NH3]} \ce{{\emph{cis}-}Pt(NH3)2(I)2}\\
                    &\ce{->[1) AgNO3][2) XS KCl]} \ce{{\emph{cis}-}Pt(NH3)2(Cl)2}
                \end{align*}
                \item Note that we start from tetrachloroplatinate because it is the most common form of platinum.
                \item Also note that XS stands for "excess."
            \end{itemize}
            \item Trans-effect order listed.
            \item The trans-effect is kinetic; concerned with rates of exchange.
            \begin{itemize}
                \item Stronger \emph{trans}-directors \textbf{labelize} the ligands opposite them.
            \end{itemize}
            \item The trans influence is thermodynamic.
            \begin{itemize}
                \item It influences the ground state structure, causing lengthening of bonds \emph{trans} to a strong-field ligand (think of this in terms of competition for electrons on the central atom; a strong-field ligand will attract more of these, making the other bond weaker).
            \end{itemize}
        \end{itemize}
    \end{enumerate}
    \item Note that intramolecular reactions (such as a second binding of a bidentate chelating ligand) are highly favored.
\end{itemize}




\end{document}