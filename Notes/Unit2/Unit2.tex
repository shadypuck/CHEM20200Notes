\documentclass[../notes.tex]{subfiles}

\pagestyle{main}
\renewcommand{\chaptermark}[1]{\markboth{\chaptername\ \thechapter\ (#1)}{}}
\setcounter{chapter}{1}

\begin{document}




\chapter{???}
\section{Lecture 4: Substitution Reactions}
% Go through and delete stuff/cite other notes (reread swaths of Labalme (2021)). Get rid of questions and reformulate as statements.
\begin{itemize}
    \item \marginnote{4/5:}Association/dissociation reactions.
    \item Fairly related to organic S\textsubscript{N}2 and S\textsubscript{N}1 reactions, respectively.
    \item General form:
    \begin{equation*}
        \ce{ML6 + L$'$ <=> ML5L$'$ + L}
    \end{equation*}
    \item We investigate the position of the equilibrium with the three main characteristics that determine reactivity.
    \begin{enumerate}
        \item Sterics.
        \begin{itemize}
            \item Related to the metal coordination number.
            \begin{itemize}
                \item $\text{C.N.}>6$ is typically disfavored.
                \item $\text{C.N.}<6$ is possible.
            \end{itemize}
            \item The size of \ce{L$'$} is also important: If $\ce{L$'$}=\ce{PPh3}$ for example, this is hard to get to $\text{C.N.}>4$.
        \end{itemize}
        \item Ligand character.
        \begin{itemize}
            \item In nonpolar media, dissociation of charged groups (e.g., \ce{Cl-}) will be disfavored. However, the opposite is true in polar media.
            \begin{itemize}
                \item This is because of the issue of making charge/ionizing.
            \end{itemize}
            \item The match between \ce{M} and \ce{L} (e.g., hard/soft, electron rich/poor) is also important.
            \begin{itemize}
                \item For example, \ce{Fe^0} will bind \ce{CO} strongly since \ce{Fe^0} is electron rich and \ce{CO} is a $\pi$ acceptor.
                \item However, \ce{Fe^{IV}} will not (as a hard, electron-poor metal center).
            \end{itemize}
        \end{itemize}
        \item Electronic structure of the metal center (whether or not the metal is electronically saturated [has 18 electrons]).
        \begin{itemize}
            \item $18\,\e[-]$: it will not want to coordinate an additional \ce{L$'$}.
            \item $20\,\e[-]$: it will want to dissociate.
            \item $16\,\e[-]$: it \emph{can} associate.
            \begin{itemize}
                \item However, it may not want to given that $16\,\e[-]$ square-planar complexes are fairly stable.
                \item The associated state may be a transition state in a square-planar ligand substitution or otherwise not a ground state.
            \end{itemize}
        \end{itemize}
    \end{enumerate}
    \item Ligand substitution reactions terms: \textbf{Kinetic} and \textbf{thermodynamic}.
    \item \textbf{Kinetic} (considerations): Elements are inert (slow) or labile (fast).
    \item \textbf{Thermodynamic} (considerations): Which side of an equilibrium will be favored. Elements are stable or reactive.
    \item In ligand substitution reactions, there are two limiting regimes:
    \begin{enumerate}
        \item Associative substitution.
        \begin{itemize}
            \item See the related discussion in \textcite{bib:CHEM20100Notes}.
            \item This is the most general reaction type, even for coordinatively saturated complexes.
            \item Rate law:
            \begin{equation*}
                \dv{\ce{[ML5L$'$]}}{t} = k_\text{obs}\ce{[ML6][L$'$]}
            \end{equation*}
        \end{itemize}
        \item Dissociative mechanism.
        \begin{itemize}
            \item See the related discussion in \textcite{bib:CHEM20100Notes}.
            \item There are many things that look dissociative that are associative (e.g., instead of forming a 5-coordinate species, you could just have a molecule of the solvent displace a ligand).
            \item This mechanism is rare and hard to prove.
            \item Rate law:
            \begin{equation*}
                \dv{\ce{[ML5L$'$]}}{t} = \frac{k_2k_1\ce{[ML6][L]}}{k_{-1}\ce{[L]}+k_2\ce{[L$'$]}}
            \end{equation*}
            \item Experimentally, we swamp the reaction with \ce{L$'$} so that $\ce{[L$'$]}>>>$ than all other reagents. This makes it so that the rate is just $k_\text{obs}\ce{[ML6]}$, i.e., pseudo-first order conditions.
        \end{itemize}
    \end{enumerate}
    \item Unfortunately, much like in orgo, very few cases are at these extremes and we can have hybrids called\dots
    \begin{enumerate}[resume]
        \item Interchange mechanisms.
        \begin{itemize}
            \item See the related discussion in \textcite{bib:CHEM20100Notes}.
            \item Within this category, we can have $I_a$ (associative interchange) and $I_d$ (dissociative interchange).
            \item In the transition state, we have \ce{L$'$} coming in and \ce{L} leaving at the same time.
        \end{itemize}
    \end{enumerate}
    \item Kinetics and rates of these mechanisms.
    \item Several categories (measure with water exchange rates; see \textcite{bib:CHEM20100Notes}):
    \begin{enumerate}[label={\Roman*)}]
        \item Very fast.
        \begin{itemize}
            \item Alkali metals (species that primarily engage in ionic bonding; little covalent character).
            \item $\SI{e8}{\per\second}$; close to the diffusion limit.
        \end{itemize}
        \item Fast.
        \begin{itemize}
            \item Higher valent ions; often \ce{M^3+} such as \ce{Al^3+}.
            \begin{itemize}
                \item Higher charge $\Rightarrow$ higher ligand affinity $\Rightarrow$ slightly slower but still pretty fast.
            \end{itemize}
            \item $\num{e3}$-$\SI{e8}{\per\second}$.
        \end{itemize}
        \item Slower.
        \begin{itemize}
            \item Getting into the transition metals: \ce{Fe^3+}, \ce{V^3+}, \ce{Ti^3+}.
            \begin{itemize}
                \item $d$-orbital splitting $+$ covalency $\Rightarrow$ stronger bonding $\Rightarrow$ slower exchange rate.
            \end{itemize}
            \item $\num{e1}$-$\SI{e4}{\per\second}$.
        \end{itemize}
        \item Inert.
        \begin{itemize}
            \item \ce{Co^3+}, \ce{Cr^3+}, \ce{Pt^2+}, and \ce{Fe^2+}(L.S.).
            \item $\num{e-8}$-$\SI{e-4}{\per\second}$.
        \end{itemize}
    \end{enumerate}
    \item The overlap between the rates reflects the fact that there is no hard and fast cut off between categories.
    \item The identity of \ce{L$'$} also influences rates.
    \begin{itemize}
        \item Reaction rates increase with the ligand field strength of \ce{L$'$}\footnote{Goes over Table IX.1 from \textcite{bib:CHEM20100Notes}.}.
    \end{itemize}
    \item Characteristics of the metal that control the observed rates.
    \begin{itemize}
        \item Ranking L.S. metal centers (slowest to fastest): $\ce{Co^{III}}<\ce{Cr^{III}}<\ce{Mn^{III}}<\ce{Fe^{III}}<\ce{Ti^{III}}<\ce{V^{III}}$.
        \item Considering the $d$ counts, we have $d^6<d^3<d^4<d^5<d^1<d^2$.
        \item Now think of this in terms of the $d$-orbitals splitting diagram (Figure \ref{fig:6-octab}).
        \begin{itemize}
            \item As the antibonding orbitals get filled, $\sigma$ bonds will weaken, promoting a faster exchange.
            \item Full and half-full $t_{2g}$ also provides stability.
        \end{itemize}
    \end{itemize}
    \item Thus, we list the following configurations as inert and labile (see the related discussion in \textcite{bib:CHEM20100Notes}):
    \begin{itemize}
        \item Inert: $d^3$, L.S. $d^{4,5,6}$, and square planar $d^8$.
        \item Labile: $d^0$, $d^1$, $d^2$, H.S. $d^{4,5,6}$, $d^7$, $d^9$, $d^{10}$.
    \end{itemize}
    \item Other important kinetic factors:
    \begin{enumerate}
        \item Oxidation state.
        \begin{itemize}
            \item As oxidation state increases, exchange rate decreases (becomes more inert).
        \end{itemize}
        \item Size.
        \begin{itemize}
            \item Smaller ions are more inert.
            \item However, first row ions are almost always labile (because they more readily populate higher spin states).
        \end{itemize}
        \item Chelate effect.
        \begin{itemize}
            \item Reviews some info from \textcite{bib:CHEM20100Notes}.
            \item Chelating ligands form a ring or a \textbf{metallacycle} (this is why 4,5-membered ligands are stable; because 5,6-membered rings are favorable).
            \item Binding of a chelating ligand is typically favored, primarily due to entropic reasons (effective concentration is secondary).
            \item Example: Gives actual $\Delta G=\Delta H-T\Delta S$ thermodynamic data for the formation reaction of \ce{Cu(MeNH2)4^2+} vs. \ce{Cu(en)2^2+} to emphasize the importance of entropy (see the related discussion in the notes on Chapter 10 in \textcite{bib:CHEM20100Notes}).
            \item EDTA is a hexadentate ligand that is commonly used in biology to pull all metal centers out of solution.
            \begin{itemize}
                \item For \ce{Fe^3+} for example, $K_f=\SI{e25}{\per\mole}$. What is $\si{\per\mole}$ and why is it here?
                \item Sidenophones and euterobactin are biology's own chelaters ($K_f=\SI{e52}{\per\mole}$).
                \item These chelaters involved because if bacteria are going to invade a host, they need to scavenge iron, but iron is pretty tightly regulated. Thus, there has been an arms race of molecules that can scavenge iron or prevent iron from being scavenged.
            \end{itemize}
            \item Chelation therapy: If exposed to a heavy metal, you will be given chelating agents that will bind to metal ions and cause them to be excreted from the body.
        \end{itemize}
        \item Trans effect.
        \begin{itemize}
            \item Reviews some info from \textcite{bib:CHEM20100Notes}.
            \item Helps predict the \textbf{regiochemistry} of where a given ligand will substitute.
            \item Cis-platin reaction mechanism: \ce{{\emph{cis}-}Pt(NH3)2(Cl)2 -> {\emph{cis}-}Pt(NH3)2(H2O)2} in the body, which binds to DNA on the \emph{cis}-water side, causing a kink, stopping transcription, and initiating apoptosis.
            \begin{itemize}
                \item Cis-platin is quite toxic (people are trying to develop formulations that are less so), but highly effective at stopping cancer.
                \item Can't have \emph{trans} because it doesn't have the \emph{cis}-water side. Thus, this synthesis mechanism doesn't work: \ce{[PtCl4]^2- ->[2NH3] {\emph{trans}-}Pt(NH3)2Cl2}.
                \item Therefore, we synthesize it as follows.
                \begin{align*}
                    \ce{K2PtCl4} &\ce{->[4KI]} \ce{PtI4^2-}\\
                    &\ce{->[2NH3]} \ce{{\emph{cis}-}Pt(NH3)2(I)2}\\
                    &\ce{->[1) AgNO3][2) XS KCl]} \ce{{\emph{cis}-}Pt(NH3)2(Cl)2}
                \end{align*}
                \item Note that we start from tetrachloroplatinate because it is the most common form of platinum.
                \item Also note that XS stands for "excess."
            \end{itemize}
            \item Trans-effect order listed.
            \item The trans-effect is kinetic; concerned with rates of exchange.
            \begin{itemize}
                \item Stronger \emph{trans}-directors \textbf{labelize} the ligands opposite them.
            \end{itemize}
            \item The trans influence is thermodynamic.
            \begin{itemize}
                \item It influences the ground state structure, causing lengthening of bonds \emph{trans} to a strong-field ligand (think of this in terms of competition for electrons on the central atom; a strong-field ligand will attract more of these, making the other bond weaker).
            \end{itemize}
        \end{itemize}
    \end{enumerate}
    \item Note that intramolecular reactions (such as a second binding of a bidentate chelating ligand) are highly favored.
\end{itemize}



\section{Lecture 5: Electron Transfer Reactions}
\begin{itemize}
    \item \marginnote{4/7:}More unique to inorganic chemistry since metal atoms have access to many more electrons than common organic atoms.
    \item General form:
    \begin{equation*}
        \ce{M^n+ <=>[$-\e[-]$][$+\e[-]$] M^{(n+1)+}}
    \end{equation*}
    \begin{itemize}
        \item The forward reaction is known as \textbf{oxidation} (metal oxidation state increases), while the reverse is known as \textbf{reduction} (metal oxidation state decreases).
        \item This is different than the oxidation/reduction reactions of organic chemistry, which involve removing or adding, respectively, a hydrogen.
    \end{itemize}
    \item This redox chemistry is important because many transition metals have access to multiple oxidation states.
    \item Two Nobel prizes in this area:
    \begin{itemize}
        \item Henry Taube (1983): Electron transfer in metals.
        \item Rudy Marcus (1992): Marcus theory of electron transfer.
    \end{itemize}
    \item 2 general flavors of electron transfer reactions: \textbf{inner sphere} and \textbf{outer sphere}.
    \item \textbf{Inner sphere}: Bonds are formed.
    \item \textbf{Outer sphere}: No bonds are formed.
    \item Example:
    \begin{itemize}
        \item Consider the reaction \ce{Fe(CN)6^4- + Mo(CN)8^3- -> Fe(CN)6^3- + Mo(CN)8^4-} (electron transfer from iron to molybdenum).
        \item The energies at play: \ce{A^{(n+1)} + B^n -> [A^{(n+1)} + B^n] -> [A^n + B^{(n+1)}]^* -> A^n + B^{(n+1)}}.
        \item Reactants $\to$ encounter complex $\to$ electron transfer state (an excited state) $\to$ products.
    \end{itemize}
    \item Energies:
    \begin{figure}[H]
        \centering
        \begin{subfigure}[b]{0.4\linewidth}
            \centering
            \begin{tikzpicture}
                \small
                \draw (5,0) -- node[below=5mm]{Potential ($V$)} (0,0) -- node[rotate=90,above]{Current ($I$)} (0,4);
                \footnotesize
                \draw [dashed,semithick] (2.5,3.5) -- (2.5,0);
                \draw [semithick] (2.5,0.1) -- ++(0,-0.2) node[below]{$E_{1/2}$};
    
                \draw [thick,blx,decoration={markings,mark=between positions 0.1 and 0.95 step 8mm with {\arrow{stealth}}},postaction={decorate}] (1,1.5) node[circle,fill,inner sep=1.5pt]{}
                    -- (1.8,1.5)
                    to[out=0,in=180] (3.2,3)
                    to[out=0,in=180] (4,2.5)
                    -- (3.5,2.5)
                    to[out=180,in=0] (1.8,1)
                    to[out=180,in=0] cycle
                ;
            \end{tikzpicture}
            \caption{Thermodynamics.}
            \label{fig:e-transferEnergya}
        \end{subfigure}
        \begin{subfigure}[b]{0.4\linewidth}
            \centering
            \begin{tikzpicture}[
                every node/.style={black}
            ]
                \small
                \draw (5,0) -- node[below=5mm]{Reaction Coordinate} (0,0) -- node[rotate=90,above]{Energy ($E$)} (0,4);
                \footnotesize
    
                \draw [thick,blx] (0.5,2.5)
                    -- node[below]{\ce{A^{n+1} + B^n}} (1.5,2.5)
                    to[out=0,in=180] (2.4,3.3)
                    to[out=0,in=180] (3.5,1.5)
                    -- node[below]{\ce{A^n + B^{n+1}}} (4.5,1.5)
                ;
    
                \begin{scope}[on background layer]
                    \draw [very thin] (1.5,2.5) -- (4.5,2.5);
                \end{scope}
                \draw [semithick,<->] (2.4,2.5) -- node[right,fill=white,inner sep=2pt]{$\Delta G^\ddagger$} (2.4,3.3);
                \draw [semithick,<->] (4,1.5) -- node[right]{$\Delta G$} (4,2.5);
            \end{tikzpicture}
            \caption{Kinetics.}
            \label{fig:e-transferEnergyb}
        \end{subfigure}
        \caption{Electron transfer reaction energies.}
        \label{fig:e-transferEnergy}
    \end{figure}
    \begin{enumerate}
        \item Thermodynamic:
        \begin{itemize}
            \item The difference in the potentials of \ce{A^{n+1}} and \ce{A^n}, and \ce{B^{n+1}} and \ce{B^n}.
            \begin{itemize}
                \item These can be measured electrochemically.
                \item We can measure the electrochemical driving force for these processes (i.e., the change in free energy during the reaction) with cyclic voltammetry.
            \end{itemize}
            \item In a cyclic voltammetry experiment\dots
            \begin{itemize}
                \item As we increase the potential to the point where the redox reaction will occur, we will see an increase as oxidation occurs.
                \item Then as we decrease the potential again to where the redox reaction will occur in the reverse direction, we will see a decrease as reduction occurs.
            \end{itemize}
            \item The midpoint $E_{1/2}$ is the thermodynamic potential (where redox is at equilibrium and you have equal amounts of both species). Is this $\Delta G$? What is going on here? Why are the equilibria misaligned?
        \end{itemize}
        \item Kinetics:
        \begin{itemize}
            \item $\Delta G=E_{1/2_\text{A}}-E_{1/2_\text{B}}$ where $E_{1/2_\text{X}}$ is the thermodynamic potential of substance X.
            \begin{itemize}
                \item $\Delta G$ is the thermodynamic contribution.
            \end{itemize}
            \item $\Delta G^\ddagger$ is the kinetic barrier, or activation energy.
        \end{itemize}
    \end{enumerate}
    \item The role of $\Delta G^\ddagger$ in an electron transfer.
    \begin{itemize}
        \item Electrons move very quickly and are highly delocalized with respect to the nuclei, so what dictates kinetics in these processes is nuclear motion (recall reorganization energy).
        \item In a simplistic sense, the key is the \ce{[A^n + B^{(n+1)}]^*} encounter complex.
        \item Electron transfer changes bond length.
        \begin{itemize}
            \item There is a kinetic barrier to the electron transfer because the thermodynamic energy is based on minimizing the energy in the reduced and oxidized forms.
        \end{itemize}
        \item Bond lengths change upon redox, so the solvent and countercations have to reorganize.
        \item This \textbf{reorganization energy} leads to a kinetic barrier (i.e., $\Delta G^\ddagger$).
        \item You can see evidence of the reorganization energy in Figure \ref{fig:e-transferEnergya}.
        \begin{itemize}
            \item You must go past the thermodynamic potential to observe the maximum/minimum current and attain complete oxidation/reduction.
        \end{itemize}
    \end{itemize}
    \item Measuring the reorganization energy.
    \begin{itemize}
        \item We do a self-exchange reaction with radiolabeled metal centers (see \textcite{bib:CHEM20100Notes}).
        \item Think of the energy scale on Figure IX.1 in \textcite{bib:CHEM20100Notes}) as discrete. To get over $\Delta G^\ddagger$, we must change vibrational states.
        \begin{itemize}
            \item Indeed, the short- and long-bond iron complexes have two vibrational states, but their combined transition state with medium bonds has a new vibrational state.
        \end{itemize}
        \item With electronic coupling, the two parabolas split into an upper loop and a lower loop with a bump.
        \item To treat this, we use the equation $\Delta G^\ddagger=\Delta G^\ddagger_t+\Delta G^\ddagger_v+\Delta G^\ddagger_0$.
        \begin{itemize}
            \item $\Delta G^\ddagger_t$ is the translational energy, which is moving the two species together.
            \item $\Delta G^\ddagger_v$ is vibrational, which is concerned with the bond lengths of the irons' matching structures.
            \item $\Delta G^\ddagger_0$ is the solvent, dipole, counterion, etc. This can be large (so one of the greatest contributors is the environment in which the system lies).
        \end{itemize}
    \end{itemize}
    \item Example:
    \begin{itemize}
        \item \ce{Co(NH3)^2+} / \ce{Co(NH3)6^3+} is H.S. $d^7$ / L.S. $d^6$.
        \item Self-exchange is slow because nuclear reorganization is large ($\SI{0.2}{\angstrom}$ difference in bond length, which is significant).
        \begin{itemize}
            \item Note that this arises from the different electronic configurations.
        \end{itemize}
        \item The ion is getting smaller \emph{and} going low-spin during reorganization.
    \end{itemize}
    \item Another example:
    \begin{itemize}
        \item \ce{Ru(NH3)6^2+} / \ce{Ru(NH3)6^3+}.
        \item $k_\text{exch}$ is eight orders of magnitude faster than the previous example.
        \item This is because ruthenium is low-spin throughout ($\Delta(\ce{Ru-N})\approx\SI{0.04}{\angstrom}$ which is much smaller, so there is a smaller reorganization energy).
    \end{itemize}
    \item Key take aways:
    \begin{itemize}
        \item Electrons move fast, so what actually induces a kinetic barrier is the movement of the nuclei which have to reorganize in order to accommodate the electron popping between the two atoms.
        \item Can be accelerated by electron coupling, as in inner sphere mechanisms.
    \end{itemize}
    \item Inner sphere electron transfer: Some bonds are involved in the electron transfer.
    \begin{itemize}
        \item Accelerated by electron coupling, but hindered by greater nuclear reorganization energy (a bridging bond must be formed).
    \end{itemize}
    \item Example:
    \begin{itemize}
        \item Consider the reaction
        \begin{align*}
            \ce{Co(NH3)5Cl^2+ + Cr(H2O)5^2+} &\ce{->} \ce{Co(NH3)5^2+ + Cr(H2O)5Cl^2+}\\
            &\ce{->[H2O]} \ce{Co(H2O)6^2+ + Cr(H2O)5Cl^2+}
        \end{align*}
        \item The intermediates are \ce{[(H3N)5Co^{III}-Cl-Cr^{II}(OH2)5]^4+ -> [(H3N)5Co^{II}-Cl-Cr^{III}(OH2)5]^4+}.
        \item The rate is reasonably fast ($\SI{6e5}{\per\mole\per\second}$).
        \item How does this vary as a function of \ce{X-}?
        \begin{itemize}
            \item As ligand size (more diffuse; better at bridging) and charge (more electrostatic influences) increase, so does rate ($\ce{Br-}>\ce{Cl-}>\ce{F-}>\ce{H2O}>\ce{NH3}$).
        \end{itemize}
    \end{itemize}
    \item Inner-sphere electron transfer: Mixed valency.
    \item Consider the Creutz-Taube ion.
    \begin{figure}[H]
        \centering
        \chemleft{[}
            \chemfig{{(NH_3)_5}Ru-N**6(---N(-Ru{(NH_3)_5})---)}
        \chemright{]^{5+}}
        \caption{The Creutz-Taube ion.}
        \label{fig:CreutzTaube}
    \end{figure}
    \begin{itemize}
        \item The bridging ligand is a pyrazole.
        \item The electron transfer is very fast; thus, the oxidation state is approximately \ce{Ru2^{2.5}}.
    \end{itemize}
    \item Such electron transfers are measured with Near-IR spectroscopy, which can see inter-valence charge transfer bands (IVCT), which include bonds.
    \begin{itemize}
        \item This very low energy form of spectroscopy observes the energy that it takes to excite an electron between the two ruthenium centers.
    \end{itemize}
    \item Robin-Day classification:
    \begin{enumerate}[label={\Roman*)}]
        \item Completely localized.
        \begin{itemize}
            \item Regardless of the spectroscopic technique used, a difference between \ce{Ru^{II}} and \ce{Ru^{III}} can be observed.
        \end{itemize}
        \item Evidence of some delocalization.
        \begin{itemize}
            \item Most common.
        \end{itemize}
        \item Completely delocalized.
    \end{enumerate}
    \item If you go fast enough (ultrafast spectroscopy; femtosecond lasers), almost any system looks localized.
    \item Marcus theory:
    \begin{itemize}
        \item Built off of the Bell-Evans-Polanyi Relationship.
        \item See Figure IX.2 and the related discussion in \textcite{bib:CHEM20100Notes}.
        \item Two big insights:
        \begin{itemize}
            \item When $\Delta G=-\lambda$, $\Delta G^\ddagger=0$.
            \item When $\Delta G<-\lambda$, $\Delta G^\ddagger>0$.
        \end{itemize}
        \item The case where $\Delta G^\circ<-\lambda$ is called the \textbf{Marcus inverted region}.
        \item \textbf{Marcus equation}:
        \begin{equation*}
            k_\text{ET} = \nu_Nk_e\e[-\Delta G^\ddagger/RT]
        \end{equation*}
        where $\nu_N$ is the nuclear frequency (how accessible vibrational excited states are; related to the width of the parabolas in Figure IX.2 of \textcite{bib:CHEM20100Notes}) and $k_e$ is the electronic factor (related to overlap, probability of transfer, etc.; usually set to 1).
        \item More importantly, Marcus discovered that
        \begin{equation*}
            \Delta G^\ddagger = \frac{(\lambda+\Delta G)^2}{4\lambda}
        \end{equation*}
        which implies that as $\lambda\to -\Delta G$, $\Delta G^\ddagger\to 0$. Furthermore, as $\lambda$ passes $-\Delta G$, $\Delta G^\ddagger$ increases.
        \item This has important implications in biology, catalysis, etc. For example, if you want to slow down an undesirable side reaction and speed up your main reaction, provide more driving force. This accelerates your main reaction and moves your side reaction into the inverted region.
    \end{itemize}
\end{itemize}




\end{document}