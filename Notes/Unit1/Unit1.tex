\documentclass[../notes.tex]{subfiles}

\pagestyle{main}
\renewcommand{\chaptermark}[1]{\markboth{\chaptername\ \thechapter\ (#1)}{}}

\begin{document}




\chapter{???}
\section{Lecture 1: Introduction/Key Trends}
\begin{itemize}
    \item \marginnote{3/29:}Largely asynchronous, but synchronous discussions, office hours, and tests.
    \begin{itemize}
        \item Refer to the Canvas site for all information; it's the class hub.
    \end{itemize}
    \item To understand transition metal trends and properties, start with \emph{atomic} properties and trends.
    \item \textbf{Electronegativity}: The energy that an atom will gain when it gains an electron.
    \begin{itemize}
        \item Fluorine really wants to gain an electron; thus, it has high electronegativity.
        \begin{itemize}
            \item Do you mean electron affinity?
        \end{itemize}
        \item Increases across a row; decreases down a column.
        \begin{itemize}
            \item Transition metals, in general, are an exception to this rule.
            \item This is because of the \textbf{lanthanide contraction}.
        \end{itemize}
        \item Discontinuities in the transition metals (\ce{Mn} and \ce{Ni}) correspond to half-filled and filled $d$ shells, respectively.
        \begin{itemize}
            \item Extra stability means less of a desire to gain an electron.
        \end{itemize}
    \end{itemize}
    \item \textbf{Ionization potential}: The energy required to remove an electron.
    \begin{itemize}
        \item Varies with the identity of an element \emph{and} its oxidation state.
        \item Increases across a row; decreases down a column.
    \end{itemize}
    \item Size:
    \begin{itemize}
        \item Many different definitions (depending on the specific thing you're interested in, one may be more useful than another). For example,
        \begin{itemize}
            \item Atomic radius: Specific to an element.
            \item Ionic radius: Specific to an oxidation state; as in salts and coordination complexes.
            \item Covalent radius: Distance that one would expect for a bond; varies with bond order.
        \end{itemize}
        \item Decreases across a row; increases down a column (a notable exception to the latter trend follows from the lanthanide contraction).
        \item Things that affect size:
        \begin{itemize}
            \item Oxidation state.
            \item Spin state (high spin [larger; this is because the $e_g$ orbitals are antibonding, and antibonding electrons both push the bounds of the atom and weaken bonds, increasing the covalent radius] vs. low spin [smaller]).
        \end{itemize}
    \end{itemize}
    \item \textbf{Lanthanide contration}: In the transition metals, there is a small/no increase in size between the second and third rows.
    \begin{itemize}
        \item This is because the shell added in between contains the $f$ orbitals, which are small, do not extend past the noble gas core, and do not provide good shielding.
        \item $Z$ goes up a lot with little shielding, so the $5d$ orbitals are contracted; thus, $4d$/$5d$ orbitals are similar in size.
    \end{itemize}
    \item \textbf{Oxidation state}: The number of electrons a metal center is below its valence.
    \begin{itemize}
        \item Typically, the maximum oxidation state is defined by the $d$-count for the 0-valent metal.
    \end{itemize}
    \item Having discussed four trends, how are they related?
    \begin{itemize}
        \item As oxidation state increases, "electronegativity" and ionization potential will increase, and the radius will decrease.
        \begin{itemize}
            \item This is because removing an electron $\Rightarrow$ reduces shielding $\Rightarrow$ higher positive charge $\Rightarrow$ all orbitals decrease in energy $\Rightarrow$ all orbitals decrease in size (hence radius decreases, too).
            \item Watch out for discontinuities such as \ce{Mn^2+}.
        \end{itemize}
    \end{itemize}
    \item Magnetic properties: Unique to the transition metals and the $f$ block.
    \begin{itemize}
        \item Consider \ce{Fe^{II}L6^2+} ($d^6$).
        \item Possible states: Low spin ($S=0$), intermediate spin ($S=1$; rare), and high spin ($S=2$).
        \item We predict which state dominates by the:
        \begin{itemize}
            \item \textbf{Pairing energy}.
            \item \textbf{Ligand field stabilization energy}.
        \end{itemize}
    \end{itemize}
    \item \textbf{Pairing energy}: The energy cost of putting two electrons in the same orbital. \emph{Also known as} \textbf{PE}.
    \begin{itemize}
        \item Trends with orbital size/radius.
        \item Decreases down a column.
    \end{itemize}
    \item \textbf{Ligand field stabilization energy}: \emph{Also known as} \textbf{LFSE}.
    \begin{itemize}
        \item Can be thought of in terms of crystal field theory.
        \begin{itemize}
            \item Extra thoughts on Figure VI.10 of \textcite{bib:CHEM20100Notes}: Donating negative charge to a free metal ion in a spherically symmetric fashion uniformly raises the energy of the $d$ orbitals by increasing repulsions and size.
            \item Low-spin LFSE: $6\cdot\SI{-4}{Dq}+\SI{3}{PE}=\SI{-24}{Dq}+\SI{3}{PE}$.
            \item Intermediate-spin LFSE: $5\cdot\SI{-4}{Dq}+1\cdot\SI{6}{Dq}+\SI{2}{PE}=\SI{-14}{Dq}+\SI{2}{PE}$.
            \item High-spin LFSE: $4\cdot\SI{-4}{Dq}+2\cdot\SI{6}{Dq}+\SI{1}{PE}=\SI{-4}{Dq}+\SI{1}{PE}$.
            \item Thus, the energy difference between the low-spin and high-spin configurations is $\SI{20}{Dq}+\SI{2}{PE}$. It follows that if $\SI{10}{Dq}>\SI{1}{PE}$, then the complex will be low spin; and if $\SI{10}{Dq}<\SI{1}{PE}$, then the complex will be high spin.
            \item This also explains why the intermediate spin state is rare: if $\Delta_o$ is large enough to make $\SI{10}{Dq}>\SI{1}{PE}$, then it will likely take the complex all the way to a low-spin configuration (and vice versa for high spin).
            \item Lastly, this means that \ce{Fe^{II}} is a good \textbf{spin-crossover} ion.
        \end{itemize}
    \end{itemize}
    \item \textbf{Spin-crossover} (ion): An ion that can have both high- and low-spin states.
    \begin{figure}[h!]
        \centering
        \begin{tikzpicture}[
            every node/.style={black}
        ]
            \small
            \draw (4,0) -- node[below]{$T$} (0,0) -- node[left]{$\chi T$} (0,3);
    
            \footnotesize
            \draw [blx,thick] (0,0) -- (0.3,0) to[out=0,in=180] node[near start,right]{$S=0$ L.S.} (2.8,2.5) -- (3.5,2.5) node[below]{$S=2$ H.S.};
    
        \end{tikzpicture}
        \caption{Magnetic moment vs. temperature for the \ce{Fe^{II}} ion.}
        \label{fig:FeIIspin-crossover}
    \end{figure}
    \begin{itemize}
        \item The graph of the magnetic moment $\chi T$ of \ce{Fe^{II}} vs. temperature $T$ (see Figure \ref{fig:FeIIspin-crossover}) moves from $S=0$ at the bottom left to $S=2$ at the top right.
        \item $d^2$ ions are never spin-crossover ions: $\SI{-8}{Dq}+\SI{0}{PE}$ for high spin vs. $\SI{-8}{Dq}+\SI{1}{PE}$ for low spin.
    \end{itemize}
    \item Dq and PE values depend on:
    \begin{itemize}
        \item Ligand field strength.
        \begin{itemize}
            \item $\Delta O_h$ increases as $\sigma$ donation increases.
            \item $\Delta O_h$ increases as $\pi$ acceptance increases.
            \item $\Delta O_h$ decreases as $\pi$ donation increases.
            \item To what extent do we need to have the spectrochemical series memorized?
        \end{itemize}
        \item Metal center.
        \begin{itemize}
            \item Larger, more diffuse metals (i.e., second- and third-row transition metals) have better overlap with the ligands, giving rise to larger $\Delta O_h$.
            \item Note that pairing energy decreases in second- and third-row transition metals (due to the larger orbitals).
            \item These two factors imply that second- and third-row transition metals are almost always low spin.
        \end{itemize}
        \item Oxidation state.
        \begin{itemize}
            \item As oxidation state increases, $\Delta O_h$ increases (due to better energy matching, higher "electronegativity," the role of electrostatics, and the electron configuration [$d^5$ is almost always high spin, and $d^6$ is often low spin]).
            \item Why do $d^5$ and $d^6$ exhibit the above behavior? Shouldn't the stabilized orbitals be split more?
            \item See the below spectrochemical series for metals.
        \end{itemize}
        \item Geometry.
        \begin{itemize}
            \item In a $\sigma$-only sense, lower coordination numbers tend to have smaller LFSEs.
            \item $T_d<C_{4v}\approx D_{3h}<O_h<D_{4h}$.
            \item $\Delta_\text{sq. pl.}\approx 1.74\, \Delta O_h$.
        \end{itemize}
    \end{itemize}
    \item Spectrochemical series for metals (not as precise as the one for ligands, but a decent approximation):
    \begin{equation*}\hspace{-3em}
        \ce{Mn^2+} < \ce{Ni^2+}
        < \ce{Co^2+}
        < \ce{Fe^2+}
        < \ce{V^2+}
        < \ce{Fe^3+}
        < \ce{Co^3+}
        < \ce{Mn^4+}
        < \ce{Mo^3+}
        < \ce{Rh^3+}
        < \ce{Ru^3+}
        < \ce{Pd^4+}
        < \ce{Ir^3+}
        < \ce{Pd^4+}
    \end{equation*}
    \item Hard/soft acid-base theory:
    \begin{itemize}
        \item Common Lewis acids:
        \begin{itemize}
            \item Proton: \ce{H+}.
            \item Molecules with no octet: \ce{AlCl3}, \ce{BR3} (boranes), \ce{BeH2}.
            \item Metal cations: \ce{Na+}, \ce{Ti^4+}.
            \item $\pi$ acids: \ce{CO2}, \ce{CO}, \ce{PR3}.
        \end{itemize}
        \item Common Lewis bases:
        \begin{itemize}
            \item Carbanions: \ce{CR3}.
            \item Hydrides: \ce{KH}, \ce{NaH}, \ce{LiAlH4}.
            \item Amines, amides, and phosphines: \ce{NH3}, \ce{PR4}, \ce{NH2-}.
            \item \ce{OH2}, \ce{SR2}, \ce{OH-}.
            \item Halides: \ce{F-}, \ce{Cl-}, \ce{Br-}.
            \item Carbonyl: \ce{CO} (means \ce{CO} is amphoteric).
            \item Olefins: \ce{C2H4}.
        \end{itemize}
        \item Distinguishes hard vs. soft\footnote{Hard vs. soft is the basis for the solubility rules!}.
    \end{itemize}
\end{itemize}



\section{Office Hours (Whitmeyer)}
\begin{itemize}
    \item \marginnote{3/30:}Electron affinity, not electronegativity, for the periodic trend?
    \begin{itemize}
        \item At higher levels, people don't really distinguish between the two.
    \end{itemize}
    \item To what extent do we need to have the spectrochemical series memorized?
    \begin{itemize}
        \item You don't need to memorize them, but it's good to know some of them off the top (exams are open note, but there are time constraints).
    \end{itemize}
    \item In connection with oxidation state, Prof. Anderson mentioned that $d^5$ is almost always high spin, and $d^6$ is often low spin. Why? Shouldn't the stabilized orbitals be split more?
    \begin{itemize}
        \item 5 $d$ electrons each in their own orbital minimizes the pairing energy.
        \item 6 $d$ electrons all occupy the lower orbitals to minimize antibonding contributions.
    \end{itemize}
    \item Using the textbook:
    \begin{itemize}
        \item The lectures are essential in this course, and if you don't understand something in the lecture, ask Sophie or John or read the textbook.
        \item If it's in those chapters, it could be asked about, but it probably won't be if John doesn't talk about it.
    \end{itemize}
\end{itemize}




\end{document}