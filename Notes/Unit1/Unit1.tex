\documentclass[../notes.tex]{subfiles}

\pagestyle{main}
\renewcommand{\chaptermark}[1]{\markboth{\chaptername\ \thechapter\ (#1)}{}}

\begin{document}




\chapter{???}
\section{Lecture 1: Introduction/Key Trends}
\begin{itemize}
    \item \marginnote{3/29:}Largely asynchronous, but synchronous discussions, office hours, and tests.
    \begin{itemize}
        \item Refer to the Canvas site for all information; it's the class hub.
    \end{itemize}
    \item To understand transition metal trends and properties, start with \emph{atomic} properties and trends.
    \item \textbf{Electronegativity}: The energy that an atom will gain when it gains an electron.
    \begin{itemize}
        \item Fluorine really wants to gain an electron; thus, it has high electronegativity.
        \begin{itemize}
            \item Do you mean electron affinity?
        \end{itemize}
        \item Increases across a row; decreases down a column.
        \begin{itemize}
            \item Transition metals, in general, are an exception to this rule.
            \item This is because of the \textbf{lanthanide contraction}.
        \end{itemize}
        \item Discontinuities in the transition metals (\ce{Mn} and \ce{Ni}) correspond to half-filled and filled $d$ shells, respectively.
        \begin{itemize}
            \item Extra stability means less of a desire to gain an electron.
        \end{itemize}
    \end{itemize}
    \item \textbf{Ionization potential}: The energy required to remove an electron.
    \begin{itemize}
        \item Varies with the identity of an element \emph{and} its oxidation state.
        \item Increases across a row; decreases down a column.
    \end{itemize}
    \item Size:
    \begin{itemize}
        \item Many different definitions (depending on the specific thing you're interested in, one may be more useful than another). For example,
        \begin{itemize}
            \item Atomic radius: Specific to an element.
            \item Ionic radius: Specific to an oxidation state; as in salts and coordination complexes.
            \item Covalent radius: Distance that one would expect for a bond; varies with bond order.
        \end{itemize}
        \item Decreases across a row; increases down a column (a notable exception to the latter trend follows from the lanthanide contraction).
        \item Things that affect size:
        \begin{itemize}
            \item Oxidation state.
            \item Spin state (high spin [larger; this is because the $e_g$ orbitals are antibonding, and antibonding electrons both push the bounds of the atom and weaken bonds, increasing the covalent radius] vs. low spin [smaller]).
        \end{itemize}
    \end{itemize}
    \item \textbf{Lanthanide contration}: In the transition metals, there is a small/no increase in size between the second and third rows.
    \begin{itemize}
        \item This is because the shell added in between contains the $f$ orbitals, which are small, do not extend past the noble gas core, and do not provide good shielding.
        \item $Z$ goes up a lot with little shielding, so the $5d$ orbitals are contracted; thus, $4d$/$5d$ orbitals are similar in size.
    \end{itemize}
    \item \textbf{Oxidation state}: The number of electrons a metal center is below its valence.
    \begin{itemize}
        \item Typically, the maximum oxidation state is defined by the $d$-count for the 0-valent metal.
    \end{itemize}
    \item Having discussed four trends, how are they related?
    \begin{itemize}
        \item As oxidation state increases, "electronegativity" and ionization potential will increase, and the radius will decrease.
        \begin{itemize}
            \item This is because removing an electron $\Rightarrow$ reduces shielding $\Rightarrow$ higher positive charge $\Rightarrow$ all orbitals decrease in energy $\Rightarrow$ all orbitals decrease in size (hence radius decreases, too).
            \item Watch out for discontinuities such as \ce{Mn^2+}.
        \end{itemize}
    \end{itemize}
    \item Magnetic properties: Unique to the transition metals and the $f$ block.
    \begin{itemize}
        \item Consider \ce{Fe^{II}L6^2+} ($d^6$).
        \item Possible states: Low spin ($S=0$), intermediate spin ($S=1$; rare), and high spin ($S=2$).
        \item We predict which state dominates by the:
        \begin{itemize}
            \item \textbf{Pairing energy}.
            \item \textbf{Ligand field stabilization energy}.
        \end{itemize}
    \end{itemize}
    \item \textbf{Pairing energy}: The energy cost of putting two electrons in the same orbital. \emph{Also known as} \textbf{PE}.
    \begin{itemize}
        \item Trends with orbital size/radius.
        \item Decreases down a column.
    \end{itemize}
    \item \textbf{Ligand field stabilization energy}: \emph{Also known as} \textbf{LFSE}.
    \begin{itemize}
        \item Can be thought of in terms of crystal field theory.
        \begin{itemize}
            \item Extra thoughts on Figure VI.10 of \textcite{bib:CHEM20100Notes}: Donating negative charge to a free metal ion in a spherically symmetric fashion uniformly raises the energy of the $d$ orbitals by increasing repulsions and size.
            \item Low-spin LFSE: $6\cdot\SI{-4}{Dq}+\SI{3}{PE}=\SI{-24}{Dq}+\SI{3}{PE}$.
            \item Intermediate-spin LFSE: $5\cdot\SI{-4}{Dq}+1\cdot\SI{6}{Dq}+\SI{2}{PE}=\SI{-14}{Dq}+\SI{2}{PE}$.
            \item High-spin LFSE: $4\cdot\SI{-4}{Dq}+2\cdot\SI{6}{Dq}+\SI{1}{PE}=\SI{-4}{Dq}+\SI{1}{PE}$.
            \item Thus, the energy difference between the low-spin and high-spin configurations is $\SI{20}{Dq}+\SI{2}{PE}$. It follows that if $\SI{10}{Dq}>\SI{1}{PE}$, then the complex will be low spin; and if $\SI{10}{Dq}<\SI{1}{PE}$, then the complex will be high spin.
            \item This also explains why the intermediate spin state is rare: if $\Delta_o$ is large enough to make $\SI{10}{Dq}>\SI{1}{PE}$, then it will likely take the complex all the way to a low-spin configuration (and vice versa for high spin).
            \item Lastly, this means that \ce{Fe^{II}} is a good \textbf{spin-crossover} ion.
        \end{itemize}
    \end{itemize}
    \item \textbf{Spin-crossover} (ion): An ion that can have both high- and low-spin states.
    \begin{figure}[h!]
        \centering
        \begin{tikzpicture}[
            every node/.style={black}
        ]
            \small
            \draw (4,0) -- node[below]{$T$} (0,0) -- node[left]{$\chi T$} (0,3);
    
            \footnotesize
            \draw [blx,thick] (0,0) -- (0.3,0) to[out=0,in=180] node[near start,right]{$S=0$ L.S.} (2.8,2.5) -- (3.5,2.5) node[below]{$S=2$ H.S.};
    
        \end{tikzpicture}
        \caption{Magnetic moment vs. temperature for the \ce{Fe^{II}} ion.}
        \label{fig:FeIIspin-crossover}
    \end{figure}
    \begin{itemize}
        \item The graph of the magnetic moment $\chi T$ of \ce{Fe^{II}} vs. temperature $T$ (see Figure \ref{fig:FeIIspin-crossover}) moves from $S=0$ at the bottom left to $S=2$ at the top right.
        \item $d^2$ ions are never spin-crossover ions: $\SI{-8}{Dq}+\SI{0}{PE}$ for high spin vs. $\SI{-8}{Dq}+\SI{1}{PE}$ for low spin.
    \end{itemize}
    \item Dq and PE values depend on:
    \begin{itemize}
        \item Ligand field strength.
        \begin{itemize}
            \item $\Delta O_h$ increases as $\sigma$ donation increases.
            \item $\Delta O_h$ increases as $\pi$ acceptance increases.
            \item $\Delta O_h$ decreases as $\pi$ donation increases.
            \item To what extent do we need to have the spectrochemical series memorized?
        \end{itemize}
        \item Metal center.
        \begin{itemize}
            \item Larger, more diffuse metals (i.e., second- and third-row transition metals) have better overlap with the ligands, giving rise to larger $\Delta O_h$.
            \item Note that pairing energy decreases in second- and third-row transition metals (due to the larger orbitals).
            \item These two factors imply that second- and third-row transition metals are almost always low spin.
        \end{itemize}
        \item Oxidation state.
        \begin{itemize}
            \item As oxidation state increases, $\Delta O_h$ increases (due to better energy matching, higher "electronegativity," the role of electrostatics, and the electron configuration [$d^5$ is almost always high spin, and $d^6$ is often low spin]).
            \item Why do $d^5$ and $d^6$ exhibit the above behavior? Shouldn't the stabilized orbitals be split more?
            \item See the below spectrochemical series for metals.
        \end{itemize}
        \item Geometry.
        \begin{itemize}
            \item In a $\sigma$-only sense, lower coordination numbers tend to have smaller LFSEs.
            \item $T_d<C_{4v}\approx D_{3h}<O_h<D_{4h}$.
            \item $\Delta_\text{sq. pl.}\approx 1.74\, \Delta O_h$.
        \end{itemize}
    \end{itemize}
    \item Spectrochemical series for metals (not as precise as the one for ligands, but a decent approximation):
    \begin{equation*}\hspace{-3em}
        \ce{Mn^2+} < \ce{Ni^2+}
        < \ce{Co^2+}
        < \ce{Fe^2+}
        < \ce{V^2+}
        < \ce{Fe^3+}
        < \ce{Co^3+}
        < \ce{Mn^4+}
        < \ce{Mo^3+}
        < \ce{Rh^3+}
        < \ce{Ru^3+}
        < \ce{Pd^4+}
        < \ce{Ir^3+}
        < \ce{Pd^4+}
    \end{equation*}
    \item Hard/soft acid-base theory:
    \begin{itemize}
        \item Common Lewis acids:
        \begin{itemize}
            \item Proton: \ce{H+}.
            \item Molecules with no octet: \ce{AlCl3}, \ce{BR3} (boranes), \ce{BeH2}.
            \item Metal cations: \ce{Na+}, \ce{Ti^4+}.
            \item $\pi$ acids: \ce{CO2}, \ce{CO}, \ce{PR3}.
        \end{itemize}
        \item Common Lewis bases:
        \begin{itemize}
            \item Carbanions: \ce{CR3}.
            \item Hydrides: \ce{KH}, \ce{NaH}, \ce{LiAlH4}.
            \item Amines, amides, and phosphines: \ce{NH3}, \ce{PR4}, \ce{NH2-}.
            \item \ce{OH2}, \ce{SR2}, \ce{OH-}.
            \item Halides: \ce{F-}, \ce{Cl-}, \ce{Br-}.
            \item Carbonyl: \ce{CO} (means \ce{CO} is amphoteric).
            \item Olefins: \ce{C2H4}.
        \end{itemize}
        \item Distinguishes hard vs. soft\footnote{Hard vs. soft is the basis for the solubility rules!}.
    \end{itemize}
\end{itemize}



\section{Office Hours (Whitmeyer)}
\begin{itemize}
    \item \marginnote{3/30:}Electron affinity, not electronegativity, for the periodic trend?
    \begin{itemize}
        \item At higher levels, people don't really distinguish between the two.
    \end{itemize}
    \item To what extent do we need to have the spectrochemical series memorized?
    \begin{itemize}
        \item You don't need to memorize them, but it's good to know some of them off the top (exams are open note, but there are time constraints).
    \end{itemize}
    \item In connection with oxidation state, Prof. Anderson mentioned that $d^5$ is almost always high spin, and $d^6$ is often low spin. Why? Shouldn't the stabilized orbitals be split more?
    \begin{itemize}
        \item 5 $d$ electrons each in their own orbital minimizes the pairing energy.
        \item 6 $d$ electrons all occupy the lower orbitals to minimize antibonding contributions.
    \end{itemize}
    \item Using the textbook:
    \begin{itemize}
        \item The lectures are essential in this course, and if you don't understand something in the lecture, ask Sophie or John or read the textbook.
        \item If it's in those chapters, it could be asked about, but it probably won't be if John doesn't talk about it.
    \end{itemize}
\end{itemize}



\section{Chapter 1: An Overview of Organometallic Chemistry}
\emph{From \textcite{bib:SpessardMiessler}.}
\begin{itemize}
    \item \marginnote{4/1:}\textbf{Cluster compound}: A compound containing two or more metal-metal bonds.
    \item \textbf{Sandwich compound}: A compound with a metal sandwiched between two ligand rings with cyclic delocalized $\pi$ systems.
    \item \ce{CO} is the most common of all ligands in organometallic chemistry.
    \item \textbf{Carbide cluster}: A metal cluster encapsulating a carbon atom.
    \item "Strictly speaking, the only compounds classified as organometallic are those that contain metal-carbon bonds, but in practice, complexes containing several other ligands similar to \ce{CO} in their bonding, such as \ce{NO} and \ce{N2}, are frequently included" \parencite[4]{bib:SpessardMiessler}.
    \begin{figure}[h!]
        \centering
        \chemfig{Cl-[,1.4]Pt(>:[1]Cl)(<[5]Cl)-[,1.4]\phantom{i}-[:-80,0.37,,,white]C(<[:-80]H)(>:[7]H)=[2]C(<[:80]H)(>:[1]H)}
        \caption{$\pi$ bonding in Zeise's salt.}
        \label{fig:ZeisesSalt}
    \end{figure}
    \item In the anionic component of Zeise's salt, the $\pi$ electrons of ethene bond to a \ce{PtCl3-} fragment (see Figure \ref{fig:ZeisesSalt}).
    \item Complexes with chiral ligands can "catalyze the selective formation of specific enantiomers of chiral molecules. In some cases, the enantioselectivity of these reactions has even equaled that of enzymatic systems" \parencite[7]{bib:SpessardMiessler}.
\end{itemize}



\section{Chapter 2: Fundamentals of Structure and Bonding}
\emph{From \textcite{bib:SpessardMiessler}.}
\begin{itemize}
    \item Review of the Schr\"{o}dinger wave equation atomic orbitals, and molecular orbitals.
    \begin{itemize}
        \item \emph{Shell} and \emph{subshell} are older terminology.
        \item "In a bonding interaction, electrons are concentrated between the nuclei and tend to hold the nuclei together; in an antibonding interaction, electrons avoid the region of space between  the nuclei and therefore expose the nuclei to each other's positive charges, tending to cause the nuclei to repel each other" \parencite[18]{bib:SpessardMiessler}.
    \end{itemize}
    \item Discusses a bit of computational chemistry in the abstract.
    \begin{itemize}
        \item Some of this stuff relates to what I talked about with Dr. V\'{a}zquez-Mayagoitia; I should reread this before I email him.
    \end{itemize}
    \item There may be some stuff here that CHEM 20100 didn't cover, but I'll only come back if necessary.
\end{itemize}



\section[Lecture 2: Electron Counting, \texorpdfstring{$18\,\text{e}^-$}{TEXT} Rule]{Lecture 2: Electron Counting, \texorpdfstring{$\bm{18\,\textbf{e}^-}$}{TEXT} Rule}
\begin{itemize}
    \item \marginnote{3/31:}Organometallic chemistry: Strictly speaking, compounds containing metal-carbon bonds. More broadly, it's homogeneous transition metal chemistry ([frequently diamagnetic] metals bonded to light atoms).
    \begin{itemize}
        \item Deeply related to catalysis (both fine and bulk chemical synthesis, and biology).
    \end{itemize}
    \item Transition metal trends:
    \begin{enumerate}
        \item Early transition metals tend to have higher oxidation states.
        \begin{itemize}
            \item It's easier to remove electrons from less electronegative elements (electronegativity increases across a period).
        \end{itemize}
        \item Size: $\text{1st row}<\text{2nd row}\approx\text{3rd row}$.
        \item \ce{M-L} bond strengths increase down a column.
        \begin{itemize}
            \item Two reasons: Size (larger, more diffuse orbitals have better overlap) and electronegativity (increases down a column; this trend is unique to the transition metals).
        \end{itemize}
        \item Higher coordination numbers are found for heavier metals.
        \item More high-spin species in the first row.
        \item First row transition metals prefer $1\,\e[-]$ coupled.
        \begin{itemize}
            \item Why?
        \end{itemize}
        \item More difficult to reduce as you go down a triad (column).
    \end{enumerate}
    \item Common structures:
    \begin{itemize}
        \item 4 coordinate:
        \begin{figure}[H]
            \centering
            \begin{subfigure}[b]{0.3\linewidth}
                \centering
                \chemfig{M(-L)(>:[1]L)(-[4]L)(<[5]L)}
                \caption{Structure.}
                \label{fig:4-sqPlna}
            \end{subfigure}
            \begin{subfigure}[b]{0.3\linewidth}
                \centering
                \begin{tikzpicture}[
                    every node/.style={black}
                ]
                    \footnotesize
                    \draw [ultra thick,blx]
                        (-0.25,2) -- node[below]{$x^2-y^2$} ++(0.5,0)
                        (-0.25,0.6) -- node[below]{$z^2$} ++(0.5,0)
                        (-0.85,0) -- node[below]{$xy$} ++(0.5,0) ++(0.1,0) -- node[below]{$xz$} ++(0.5,0) ++(0.1,0) -- node[below]{$yz$} ++(0.5,0)
                    ;
                \end{tikzpicture}
                \caption{$d$ orbitals.}
                \label{fig:4-sqPlnb}
            \end{subfigure}
            \caption{Square planar information.}
            \label{fig:4-sqPln}
        \end{figure}
        \begin{itemize}
            \item Square planar (note that the $z^2$ orbital can swap with the three degenerate orbitals beneath it fairly easily; what's important is that $x^2-y^2$ is higher).
            \begin{figure}[h!]
                \centering
                \begin{subfigure}[b]{0.3\linewidth}
                    \centering
                    \chemfig{M(-[2]L)(-[:-30]L)(<[:-110]L)(>:[:-150]L)}
                    \caption{Structure.}
                    \label{fig:4-tetraa}
                \end{subfigure}
                \begin{subfigure}[b]{0.3\linewidth}
                    \centering
                    \begin{tikzpicture}[
                        every node/.style={black}
                    ]
                        \footnotesize
                        \draw [ultra thick,blx]
                            (-0.85,1) -- node[below]{$xy$} ++(0.5,0) ++(0.1,0) -- node[below]{$xz$} ++(0.5,0) ++(0.1,0) -- node[below]{$yz$} ++(0.5,0)
                            (-0.55,0) -- node[below]{$z^2$} ++(0.5,0) ++(0.1,0) -- node[below,xshift=3mm]{$x^2-y^2$} ++(0.5,0)
                        ;
                    \end{tikzpicture}
                    \caption{$d$ orbitals.}
                    \label{fig:4-tetrab}
                \end{subfigure}
                \caption{Tetrahedral information.}
                \label{fig:4-tetra}
            \end{figure}
            \item Tetrahedral (much smaller splitting energy than some of the others).
        \end{itemize}
        \item 5 coordinate:
        \begin{figure}[h!]
            \centering
            \begin{subfigure}[b]{0.3\linewidth}
                \centering
                \chemfig{M(-L)(-[2]L)(-[6]L)(>:[:160]L)(<[:-150]L)}
                \caption{Structure.}
                \label{fig:5-trigBipyraa}
            \end{subfigure}
            \begin{subfigure}[b]{0.3\linewidth}
                \centering
                \begin{tikzpicture}[
                    every node/.style={black}
                ]
                    \footnotesize
                    \draw [ultra thick,blx]
                        (-0.25,2) -- node[below]{$z^2$} ++(0.5,0)
                        (-0.55,1) -- node[below]{$xy$} ++(0.5,0) ++(0.1,0) -- node[below,xshift=3mm]{$x^2-y^2$} ++(0.5,0)
                        (-0.55,0) -- node[below]{$xz$} ++(0.5,0) ++(0.1,0) -- node[below]{$yz$} ++(0.5,0)
                    ;
                \end{tikzpicture}
                \caption{$d$ orbitals.}
                \label{fig:5-trigBipyrab}
            \end{subfigure}
            \caption{Trigonal bipyramidal information.}
            \label{fig:5-trigBipyra}
        \end{figure}
        \begin{itemize}
            \item Trigonal bipyramidal (the axial ligands push $d_{z^2}$ high in energy, $d_{xy,x^2-y^2}$ are degenerate by the threefold $D_{3h}$ symmetry, and $d_{xz,yz}$ are nonbonding and thus lowest in energy; note also that this geometry has \textbf{fluxional ligands}).
            \begin{figure}[h!]
                \centering
                \begin{subfigure}[b]{0.3\linewidth}
                    \centering
                    \chemfig{M(-L)(>:[1]L)(-[2]L)(-[4]L)(<[5]L)}
                    \caption{Structure.}
                    \label{fig:5-sqPyraa}
                \end{subfigure}
                \begin{subfigure}[b]{0.3\linewidth}
                    \centering
                    \begin{tikzpicture}[
                        every node/.style={black}
                    ]
                        \footnotesize
                        \draw [ultra thick,blx]
                            (-0.25,2) -- node[below]{$x^2-y^2$} ++(0.5,0)
                            (-0.25,1.4) -- node[below]{$z^2$} ++(0.5,0)
                            (-0.85,0) -- node[below]{$xy$} ++(0.5,0) ++(0.1,0) -- node[below]{$xz$} ++(0.5,0) ++(0.1,0) -- node[below]{$yz$} ++(0.5,0)
                        ;
                    \end{tikzpicture}
                    \caption{$d$ orbitals.}
                    \label{fig:5-sqPyrab}
                \end{subfigure}
                \caption{Square pyramidal information.}
                \label{fig:5-sqPyra}
            \end{figure}
            \item Square pyramidal (think of it either as square planar with an axial ligand on top, or as octahedral missing one axial ligand on the bottom; thinking of it this way also rationalizes Figure \ref{fig:5-sqPyrab} as the mean of Figures \ref{fig:4-sqPlnb} and \ref{fig:6-octab}).
        \end{itemize}
        \item 6 coordinate.
        \begin{itemize}
            \begin{figure}[h!]
                \centering
                \begin{subfigure}[b]{0.3\linewidth}
                    \centering
                    \chemfig{M(-L)(>:[1]L)(-[2]L)(-[4]L)(<[5]L)(-[6]L)}
                    \caption{Structure.}
                    \label{fig:6-octaa}
                \end{subfigure}
                \begin{subfigure}[b]{0.3\linewidth}
                    \centering
                    \begin{tikzpicture}[
                        every node/.style={black}
                    ]
                        \footnotesize
                        \draw [ultra thick,blx]
                            (-0.55,1) -- node[below]{$z^2$} ++(0.5,0) ++(0.1,0) -- node[below,xshift=3mm]{$x^2-y^2$} ++(0.5,0)
                            (-0.85,0) -- node[below]{$xy$} ++(0.5,0) ++(0.1,0) -- node[below]{$xz$} ++(0.5,0) ++(0.1,0) -- node[below]{$yz$} ++(0.5,0)
                        ;
                    \end{tikzpicture}
                    \caption{$d$ orbitals.}
                    \label{fig:6-octab}
                \end{subfigure}
                \caption{Octahedral information.}
                \label{fig:6-octa}
            \end{figure}
            \item Octahedral ($d_{xy,xz,yz}$ are nonbonding in a $\sigma$-only framework, but can take on bonding character when $\pi$ interactions are considered).
            \begin{figure}[h!]
                \centering
                \chemfig{M(-[:40]L)(>:[:-20]L)(<[:-45]L)(-[:140]L)(>:[:-160]L)(<[:-135]L)}
                \caption{Trigonal biprysmatic information (structure).}
                \label{fig:6-trigBiprys}
            \end{figure}
            \item Trigonal biprysmatic (each pyramid is eclipsed, rather than staggered as in octahedral; \textcite{bib:trigonalBiprysmatic} explores this geometry in greater depth).
        \end{itemize}
    \end{itemize}
    \item \textbf{Fluctional ligands}: A set of ligands that readily exchange positions around the molecular center via a \textbf{Berry pseudorotation}.
    \item Ligand types (see \textcite[93]{bib:CHEM20100Notes}).
    \item \textbf{X-type} (ligand): Typically anionic, covalent donors.
    \item \textbf{L-type} (ligand): Typically neutral. \emph{Also known as} \textbf{dative donor}.
    \item \textbf{Z-type} (ligand): Typically neutral, but can be cationic (no electrons to donate; these are acceptors).
    \item Note that a carbonyl group can be both an L- and a Z-type ligand (L if it participates in $\sigma$ donation, and Z if it participates in $\pi$ acceptance).
    \begin{itemize}
        \item Similar to how \ce{Cl-} can be both a $\sigma$ and $\pi$ donor.
    \end{itemize}
    \item On hard/soft matching: remember that harder ligands will prefer harder metals, and vice versa.
    \item Electron counting:
    \item Organic chemistry concerns itself with an octet.
    \item But the octet rule is really a large HOMO-LUMO gap rule; filling stable orbitals and leaving the unstable orbitals empty.
    \begin{itemize}
        \item In \ce{CH4} for example, we want to fill the $\sigma$ and $\pi$ MOs with all 8 electrons that they can hold, but leave $\sigma^*$ and $\pi^*$ unfilled (see Figure III.17 in \textcite{bib:CHEM20100Notes}).
        \item However, in \ce{ML6} for example (considering only $d$ orbital/ligand $\sigma$ orbital interactions), we have nine $\sigma$/nonbonding orbitals that are ok to fill up and two $\sigma^*$ orbitals that we should try to avoid filling up (see Figure \ref{fig:6-octab} as well as Figure VI.2 from \textcite{bib:CHEM20100Notes} for a decent approximation).
    \end{itemize}
    \item The nine orbitals that are ok to fill up in an \ce{ML6} compound can hold 18 electrons; this gives rise to the \textbf{18 electron rule}.
    \item \textbf{18 electron rule}: An octahedral \ce{ML6} transition metal complex with 18 electrons is fairly energetically favorable.
    \item Low-spin square planar:
    \begin{itemize}
        \item 4 $\sigma$/NB ligand orbitals plus 4 nonbonding metal $d$ orbitals gives 8 $\sigma$/NB orbitals that can hold 16 electrons in total (see Figure \ref{fig:4-sqPlnb}).
        \item Figure VI.13 of \textcite{bib:CHEM20100Notes} says that \emph{two} metal orbitals form bonding/antibonding orbitals with the ligand orbitals, so why does Dr. Anderson assert that only \emph{one} does? Is it because of what he said about $d_{z^2}$ being practically interchangeable with the $d_{xy,xz,yz}$ ligands in square planar complexes?
    \end{itemize}
    \item \textbf{16 electron rule}: A square planar \ce{ML4} transition metal complex with 16 electrons is fairly energetically favorable.
    \item Note that the 18 and 16 electron rules respectively imply that octahedral complexes prefer $d^6$ configurations and square planar complexes prefer $d^8$ configurations.
    \item Note also that since the HOMOs in both 18-electron octahedral and 16-electron square planar complexes are nonbonding, the 18/16 electron rules are more of a suggestion.
    \begin{itemize}
        \item In general, these numbers are more of a maximum; lower counts can still be stable.
        \item However, there are cases of 19 and 20 electron systems.
    \end{itemize}
    \item 2 schools of thought on electron counting: the \textbf{ionic method} and the \textbf{covalent method}.
    \begin{itemize}
        \item Dr. Anderson prefers the covalent method; he's of the opinion that it's a bit more foolproof.
        \item Proponents of the ionic method argue that it's nice because it gives you the oxidation state of the metal center throughout the process, but it can run into snags with certain ligands (in step 1 below, it is not always clear what splitting electronegativity dictates).
    \end{itemize}
\end{itemize}
\begin{tchart}{1.4}{Ionic Method}{Covalent Method}
    1. Break all \ce{M-L} bonds according to electronegativity (or accordingly, to form the most stable fragments). Note that \ce{M-L} bonds split homolytically.
        & 1. Draw a legitimate Lewis structure (no half bonds or circles [as in benezene]). Don't forget lone pairs.\\
    2. The charge on the metal after step 1 is its oxidation state.
        & 2. Assign formal charges (in a dative bond, these belong to the ligand).\\
    3. From 2, assign a $d$-electron count.
        & 3. The number of electrons that a given ligand donates is equal to its formal charge plus twice the number of dative bonds plus the number of covalent bonds.\\
    4. The electron count equals the $d^n$ count plus the ligand donors (typically 2 electrons per ligand).
        & 4. The electron count is that $d^n$ count for \ce{M^0} plus the sum of the ligand electron donations minus the charge on the complex.
\end{tchart}
\begin{itemize}
    \item Gain familiarity with the $d$ counts of common transition metals.
    \item A metal can actually have multiple oxidation states in resonance with each other, whereas the electron count is indisputable, i.e., the only number that you can definitively assign to a complex.
    \begin{itemize}
        \item This is why it's better to use the covalent method; it goes straight to assigning the electron count, foregoing any possible issues with the oxidation state.
    \end{itemize}
    \item If you apply each method correctly, they should both give the same answer.
    \item Examples (ligands):
    \begin{itemize}
        \item A phosphine \ce{PR3}.
        \begin{itemize}
            \item The phosphine has a lone pair to donate to the metal center, forming a \textbf{dative bond}.
            \item Alternatively, the formal charge on phosphorous in a \ce{M-PR3} situation is $+1$ and there is 1 covalent bond.
            \item Either way, the phosphine is a 2-electron donor; this is further confirmed by the fact that phosphines are L-type ligands.
        \end{itemize}
        \item \ce{CO}.
        \begin{itemize}
            \item We have multiple possible resonance structures for a \ce{M-C#O} bond, but we can robustly treat this with the covalent method.
            \item \chemfig[atom sep=1.2em]{M-C~\charge{0=\:,45=$\oplus$}{O}}\quad has a $+1$ formal charge and 1 covalent bond, suggesting that \ce{CO} is a 2-electron donor.
            \item \chemfig[atom sep=1.2em]{M=C=\charge{45=\:,-45=\:}{O}}\quad has no formal charge and 2 covalent bonds, suggesting that \ce{CO} is a 2-electron donor.
            \item \chemfig[atom sep=1.2em]{M~C-\charge{90=\:,45=$\ominus$,0=\:,-90=\:}{O}}\quad has a $-1$ formal charge and 3 covalent bonds, suggesting that \ce{CO} is a 2-electron donor.
            \item \chemfig[atom sep=1.2em]{M-[,1.2,,,<-]\charge{180=\:,90=$\ominus$}{C}~\charge{90=$\oplus$,0=\:}{O}}\quad has a net 0 formal charge and 1 dative bond, suggesting that \ce{CO} is a 2-electron donor.
        \end{itemize}
        \item \ce{NO}.
        \begin{itemize}
            \item If \ce{NO} bonds linearly, it's a 3-electron donor (take \chemfig[atom sep=1.2em]{M-\charge{90=$\oplus$}{N}~\charge{90=$\oplus$,0=\:}{O}}\quad as a possible resonance structure).
            \item If \ce{NO} bonds bent, it's a 1-electron donor (take \chemfig[atom sep=1.2em]{M-[:30]\charge{90=\:}{N}=[:-30]\charge{0=\:,-90=\:}{O}}\quad for example).
        \end{itemize}
    \end{itemize}
    \item \textbf{Dative bond}: A covalent bond between two atoms where one of the atoms provides both of the electrons that form the bond.
    \item Examples (metal complexes):
    \begin{itemize}
        \item Ferrocene, a sandwich compound with an iron atom between two cyclopentadienyl (or Cp) groups (covalent method).
        \begin{itemize}
            \item Each carbon atom forms a single covalent bond to iron. This gives each iron four covalent bonds (two to its neighbors in the ring, one to its hydrogen, and one to iron), so there are no formal charges.
            \item Thus, each Cp ligand donates 5 electrons by the covalent method, and iron as a $d^8$ compound donates 8 electrons.
            \item Therefore, this is an 18 electron complex.
        \end{itemize}
        \item Ferrocene (ionic method):
        \begin{itemize}
            \item The cyclopentadienyl anion has a $1-$ charge, making it a $6\pi$-electron aromatic system.
            \item There are two of these anions, with a total charge of $2-$ between them, so iron must be in the \ce{Fe^2+} oxidation state to compensate.
            \item This makes iron $d^6$, which plays well with 18 electron systems.
        \end{itemize}
        \item Hexamethyl tungsten \ce{W(CH3)6} (covalent method):
        \begin{itemize}
            \item Each \ce{CH3} ligand forms a single covalent bond with \ce{W} without formal charge; thus, each donates 1 electron.
            \item \ce{W} is $d^6$.
            \item Thus, the $d$ count is 12, making it a pretty reactive compound.
        \end{itemize}
        \item \ce{W(CH3)6} (ionic method):
        \begin{itemize}
            \item For each ligand, we split to \ce{W+} and \ce{CH3-}.
            \item This makes the metal center oxidation state \ce{W^{VI}}, with a resultant $d^0$ configuration.
        \end{itemize}
        \item \ce{W(CO)6} (covalent method).
        \begin{itemize}
            \item From above, \ce{CO} is a 2-electron donor. Thus, the 6 \ce{CO}'s donate 12 electrons. This combined with the fact that \ce{W} is $d^6$ makes this an 18 electron system, i.e., pretty stable.
        \end{itemize}
        \item \ce{W(CO)6} (ionic method):
        \begin{itemize}
            \item We split \ce{W-CO} into \ce{W^0 + CO}.
        \end{itemize}
        \item \ce{Pt(Cl)4^2-} (covalent method):
        \begin{itemize}
            \item Each chloride forms 1 covalent bond (donates 1 electron).
            \item Platinum is $d^{10}$ (because it's chemically bound, the $6s$ electrons fall to the $d$ orbitals; what would the $d$ count of copper or zinc be? 10 as well?).
            \item The charge on the complex is $2-$, so the electron count is $4\cdot 1+10-(-2)=16\,\e[-]$'s.
        \end{itemize}
        \item \ce{Pt(Cl)4^2-} (ionic method):
        \begin{itemize}
            \item \ce{Pt-Cl -> Pt+ + Cl-}.
            \item Thus, we have \ce{Pt^4+}. But the charge is $2-$, so we actually have \ce{Pt^2+}, which is $d^8$, which plays well with the 16-electron system.
        \end{itemize}
        \item An enzymatic cofactor (covalent method):
        \begin{figure}[h!]
            \centering
            \chemfig{
                Fe?[Fel]
                (-[:152]\chemabove{P}{Ph_2}-[:-165,0.7]-[6]-[:-15,0.7]\chembelow{P}{Ph_2}?[Fel])
                (-[6]C~[6]O)
                -[:-40,1.5]C(=[6]O)-[:40,1.5]
                Fe?[Fer]
                (-[:28]\chemabove{P}{Ph_2}-[:-15,0.7]-[6]-[:-165,0.7]\chembelow{P}{Ph_2}?[Fer])
                (-[2]H)
                (-[:150,1.2]S?[S]?[Fel])
                -[:175,1.5]S?[Fel]-[:100]-[:70,0.7]N(-[1]H)(-[3]H)-[:-40,0.3]?[S]
            }
            \caption{An enzymatic cofactor.}
            \label{fig:e-count-enzymaticCofactor}
        \end{figure}
        \begin{itemize}
            \item Each phosphine and each carbonyl is a 2-electron donor.
            \item The hydride is an X-type ligand with a covalent bond, and thus a 1-electron donor.
            \item Now for the big bulky center bridging ligand: The sulfurs each carry a $+1$ formal charge and form two covalent bonds to the metal centers, so they each contribute three electrons. The nitrogen has an additional $+1$ formal charge, so the ligand overall is a $2\cdot 3+1=7$-electron donor.
            \item Summing all of this gives us 20 electrons.
            \item Now for the metal centers: Each iron is $d^8$.
            \item Thus, that's 36 electrons in total, but divided over two iron centers.
            \item Therefore, the electron count for each iron is 18.
        \end{itemize}
    \end{itemize}
    \item Metal-metal bonds:
    \begin{itemize}
        \item Assume that each metal will want to get to an electron count of 18.
        \item Thus, the number of \ce{M-M} bonds you would expect is
        \begin{equation*}
            \text{\# of \ce{M-M} bonds} = (18\,\e[-]\text{'s}\cdot\text{\# of metals}-\text{\# of }\e[-]\text{'s from L's and \ce{M^0}'s})/2
        \end{equation*}
        \item Essentially, one \ce{M-M} covalent bond contributes one electron to each \ce{M}, or two electrons to the complex as a whole.
    \end{itemize}
    \item As we can see, this number would be 0 for the enzymatic cofactor in Figure \ref{fig:e-count-enzymaticCofactor}, which is why we'd expect no metal-metal bonding between the two iron centers.
    \item Bridging hydrides and halides:
    \begin{itemize}
        \item We can treat this by putting a $+1$ formal charge on the bridging atom: \chemfig[atom sep=1.2em]{M-[:30]\charge{45=$\oplus$}{X}-[:-30]M}.
        \item Alternatively, we can recognize what the nature of the interaction is: \tikz[baseline={(0,-0.3)}]{\node{\chemfig[atom sep=1.2em]{M-[:30]\charge{-45=\:,45=\:,135=\:}{X}}};\node at (1.1,-0.2) {M};\draw [->] (0.6,-0.05) -- (0.9,-0.2);}.
        \begin{itemize}
            \item From the above picture, it is clear that there is one covalent and one dative bond at play, making the bridging X-type ligand a 3-electron donor.
        \end{itemize}
    \end{itemize}
    \item We are now prepared to treat one final example:
    \vspace{1em}
    \begin{figure}[h!]
        \centering
        \chemfig{
            {(}OC{)}_3Os?-[:35,1.4]\chemabove{Os}{(CO)_4}-[:-35,1.4]Os?[,,dotted,thick]{(}CO{)}_3
            (-[:-170,1]H?)
            (-[:-150,1.4]H?)
        }
        \caption{Electron counting for \ce{Os3(CO)10($\mu_2${-}H)2}.}
        \label{fig:e-count-osmiumCompound}
    \end{figure}
    \begin{itemize}
        \item Each carbonyl ligand is a 2-electron donor.
        \item Each bridging hydride is a 3-electron donor.
        \item Each \ce{Os-Os} bond contributes 2 electrons.
        \item Thus, the ligands donate 30 electrons in total.
        \item Each osmium is $d^8$.
        \item Thus, the metal centers donate 24 electrons in total.
        \item Therefore, the number of \ce{Os-Os} bonds is $\frac{18\cdot 3-(30+24)}{2}=0$, i.e., there are no \ce{Os-Os} bonds.
        \item Now this question could just be a relic of my previous understanding of bonding, and the answer may just be "MO theory," but I'm still gonna ask: Where do the electrons in all of the bonds come from? It seems like if the osmiums are giving electrons to \ce{Os-Os} and \ce{Os-H} bonds, and we still count osmium as $d^8$, we are counting some electrons twice.
    \end{itemize}
    \item Isolobal/isoelectronic analogy:
    \begin{itemize}
        \item We can assume based on the fact that \ce{Cr(CO)6} has 18 electrons and is stable that the isoelectronic compounds \ce{V(CO)6-} and \ce{Mn(CO)6+} have identical electron counts and similar properties.
        \begin{itemize}
            \item Note that all of these compounds are both isoelectronic and isolobal. What does isolobal mean?
        \end{itemize}
        \item We can do the same thing between \ce{Ni(CO)4}, \ce{Co(NO)(CO)3}, and \ce{Fe(NO)2(CO)2}.
        \item Also \ce{Mn(CO)5}, \ce{[CpMn(CO)2]-}, and \ce{CpFe(CO)2} (these are isoelectronic, but not isolobal).
    \end{itemize}
    \item We can also consider isolobal analogies between transition-metal-complex electron counts and organic fragments.
    \begin{itemize}
        \item For example,
        \begin{align*}
            18\,\e[-] &\ch{<o>} \ce{CH4}\\
            17\,\e[-] &\ch{<o>} \ce{\charge{180=\.}{C}H3}\\
            16\,\e[-] &\ch{<o>} \ce{\charge{180=\:}{C}H2}\\
            15\,\e[-] &\ch{<o>} \ce{\charge{[circle]135=\.,180=\.,-135=\.}{C}H}\\
            14\,\e[-] &\ch{<o>} \charge{[extra sep=2pt]0=\.,90=\.,180=\.,-90=\.}{C}
        \end{align*}
        \item We can also make analogies between other atoms/metal fragments: $\charge{[circle]90=\.,-30=\.,-150=\.}{P}\ch{<o>}\ce{\charge{[circle]135=\.,180=\.,-135=\.}{C}H}\ch{<o>}\ce{(CO)3Co}$.
        \item Multiply bonded fragments can also work: $\ce{M=O}\ch{<o>}\ce{M=N-R}\ch{<o>}\ce{R2C=O}$ for double bonds, and for triple bonds: $\ce{M#O}\ch{<o>}\ce{M#N-R}\ch{<o>}\ce{[R-C#N-H]^+}$.
    \end{itemize}
    \item \textbf{Oxidation state}: The number of electrons a metal has given up or acquired.
    \item \textbf{Chemical valence}: The number of electrons from the metal that are engaged in bonding.
    \item In many cases, the valence and oxidation state are the same, but they can differ.
    \begin{itemize}
        \item They notably differ when \ce{M-M} bonds and Z-type ligands are in play.
    \end{itemize}
    \item Consider the structure formed by two dimerized fic (is this the right spelling? Does it have a charge by itself?) fragments (a fic fragment is \ce{CpFe(CO)2}).
    \begin{figure}[h!]
        \centering
        \chemfig{Fe?(-[3]Cp)(-[5]OC)-[1]C(=[2]O)-[7]Fe?(-[1]CO)(-[7]Cp)-[5]C?(=[6]O)}
        \caption{Two dimerized fic fragments.}
        \label{fig:ficFragments}
    \end{figure}
    \begin{itemize}
        \item The oxidation state of each iron is \ce{Fe^I} (since \ce{Cp} is the only electronegative ligand).
        \item The valence of each iron is \ce{Fe^{II}} (since \ce{Cp} takes 1 electron and the \ce{Fe-Fe} bond takes another).
        \item Note that as this is an 18-electron complex, it makes sense that the bound "iron ion" should be $d^6$ (\ce{Fe^{II}}), not $d^7$ (\ce{Fe^I}).
        \item Does the iron have tetrahedral or square planar geometry and why?
    \end{itemize}
    \item Now consider the compound \ce{[CpFe(CO)2AlMe3]-}.
    \begin{itemize}
        \item The oxidation state of the iron is \ce{Fe^{II}}.
        \item The valence of the iron is \ce{Fe^{III}} (confirm this?).
        \item Here, unlike the last example, the oxidation state is a better descriptor (we can think of the iron as donating two electrons to \ce{AlMe3}).
    \end{itemize}
\end{itemize}



\section{Lecture 3: TM Magnetism}
\begin{itemize}
    \item \marginnote{4/2:}Magnetism is unique to the transition metals, and actually predominantly the lanthanides.
    \item Goes through the "Theoretical background for determining magnetic spins experimentally" derivation from Module 34.
    \begin{itemize}
        \item $\kappa$ can also be denoted by $\chi_V$.
        \item More on $\chi_M$: Copies Table VI.2 from \textcite{bib:CHEM20100Notes}. Some differences?
    \end{itemize}
    \item Diamagnetism:
    \begin{itemize}
        \item Arises from the circulation of paired electrons.
        \item These currents generate a field opposite of $H$, which implies that $\chi_\text{dia}$ is negative.
        \item Contributions from atoms, bonds, and molecules (anything with paired electrons). These can just be summed for a given molecule (see Pascal's constants).
    \end{itemize}
    \item Paramagnetism:
    \begin{itemize}
        \item Arises from the spin-orbit angular momentum of unpaired electrons.
        \item Note that $\chi_\text{para}=\chi_\text{measured}-\chi_\text{dia}$.
        \item We can consider a value called the magnetic moment $\mu$ of the electron. We define
        \begin{equation*}
            \mu = -g\beta\vec{s}
        \end{equation*}
        where $g$ is the $g$-factor of the free electron (\emph{also known as} the gyromagnetic ratio in \textcite[119]{bib:CHEM20100Notes}), $\beta$ is the Bohr magneton, and $\vec{s}$ is the spin angular momentum.
        \item The Hamiltonian $\mathcal{H}$ describing the energy of the interaction of the magnetic field with the magnetic moment of the electron is as follows:
        \begin{equation*}
            \mathcal{H} = -\vec{\mu}\cdot\vec{H}
            = g\beta\vec{s}\cdot\vec{H}
        \end{equation*}
        \item We can visualize this with a Zeeman splitting diagram.
        \begin{figure}[h!]
            \centering
            \begin{tikzpicture}[
                every node/.style={black}
            ]
                \footnotesize
                \draw (5,0) node[below]{$\infty$} -- node[below=3mm]{\small$H$} (0,0) node[below]{0} -- (0,4);
                \draw (0.1,2.5) -- ++(-0.2,0) node[left]{$m_s=\pm\frac{1}{2}$};
                \draw [blx,thick] (0,2.5) -- (3,3.5) (0,2.5) -- (3,1.5);
                \draw [blx,ultra thick]
                    (3,1.5) -- ++(0.5,0) node[right]{$-\frac{1}{2}g\beta H$}
                    (3,3.5) -- ++(0.5,0) node[right]{$+\frac{1}{2}g\beta H$}
                ;
                \draw [<->] (3.25,1.55) -- node[right]{$\Delta E=g\beta H$} ++(0,1.9);
            \end{tikzpicture}
            \caption{A Zeeman splitting diagram for a single electron.}
            \label{fig:zeemanSplitting-1e}
        \end{figure}
        \begin{itemize}
            \item Consider a single electron.
            \item Under zero magnetic field ($H=0$), there will be no preference for spin-up or spin-down ($M_s=\pm\frac{1}{2}$).
            \item However, as we apply a magnetic field of increasing strength, a preference develops (and the diagram splits). Indeed, under some nonzero magnetic field $H$, the system will be higher energy if the electron spin is $+\frac{1}{2}$ and lower energy if the electron spin is $-\frac{1}{2}$ (why these specific spins to higher and lower energy? Or is it arbitrary?). This effect is heightened by increasing $H$.
        \end{itemize}
        \item The magnitude of $\Delta E$ in Figure \ref{fig:zeemanSplitting-1e}.
        \begin{itemize}
            \item Let $H=\SI{25}{kG}=\SI{2.5}{\tesla}$. We know that $g=2.0023$. Thus, $\Delta E=\SI{2.3}{\per\centi\meter}$ (which is very small).
            \item What is a Boltzmann population?
            \item When we sum the magnetic contributions from all possible spin states, we find that the magnetization constant can be written as
            \begin{equation*}
                M = \frac{Ng^2\beta^2}{4k_BT}H
            \end{equation*}
            where $N$ is Avogadro's number, $k_B$ is the Boltzmann constant, and $T$ is temperature.
            \item But since $\chi_M=\frac{M}{H}$, this implies that
            \begin{equation*}
                \chi_M = \frac{Ng^2\beta^2}{4k_B}\cdot\frac{1}{T}
                = C\cdot\frac{1}{T}
            \end{equation*}
            where $C=\frac{Ng^2\beta^2}{4k_B}$ is the Curie constant. How does this relate to Curie's law from last quarter?
            \item The above relationship is useful because it tells us that we can get a linear relationship through the origin between magnetic susceptibility and $\frac{1}{T}$. Additionally, it tells us that as temperature increases, magnetic susceptibility decreases (inversely proportionally).
        \end{itemize}
    \end{itemize}
    \item Ferromagnetism and antiferromagnetism:
    \begin{itemize}
        \item Temperature's influences on these types of magnetism vary from their effects on paramagnetism.
        \begin{figure}[h!]
            \centering
            \begin{subfigure}[b]{0.4\linewidth}
                \centering
                \begin{tikzpicture}
                    \footnotesize
                    \draw (5,0) -- node[below=5mm]{\small$T$} (0,0) -- node[left]{$\chi_M$} (0,4);
                    \draw
                        (1,0.1) -- ++(0,-0.2) node[below]{$T_N$}
                        (1.9,0.1) -- ++(0,-0.2) node[below]{$T_C$}
                    ;
            
                    \draw [blx,thick] plot[domain=0.25:5,samples=100,smooth] (\x,{1/\x});
                    \draw [grx,thick] plot[domain=0:5,smooth] (\x,{1.5*\x*e^(-\x)});
                    \draw [rex,thick] (1.3,4) to[out=-85,in=150,in looseness=0.3] (1.9,0.8) to[out=-60,in=178,out looseness=0.3] (5,0.3);
                \end{tikzpicture}
                \caption{Direct proportionality.}
                \label{fig:tempMagneticSusceptibilitya}
            \end{subfigure}
            \begin{subfigure}[b]{0.4\linewidth}
                \centering
                \begin{tikzpicture}
                    \footnotesize
                    \draw (5,0) -- node[below=5mm]{\small$T$} (0,0) -- node[left]{$\frac{1}{\chi_M}$} (0,4);
            
                    \draw [blx,thick] (0,0) -- (4,4);
                    \draw [grx,thick] (0,1) -- (3,4);
                    \draw [rex,thick] (1.9,0) -- (5,3.1);
                \end{tikzpicture}
                \caption{Inverse proportionality.}
                \label{fig:tempMagneticSusceptibilityb}
            \end{subfigure}
            \caption{Temperature vs. magnetic susceptibility in different magnets.}
            \label{fig:tempMagneticSusceptibility}
        \end{figure}
        \begin{itemize}
            \item The blue lines correspond to paramagnetism, the red lines correspond to ferromagnetism (which, under direct proportionality, have an abrupt inflection point at the Curie temperature $T_C$), and the green lines correspond to antiferromagnetism (which, under direct proportionality, have a maximum at the Neel temperature $T_N$).
        \end{itemize}
    \end{itemize}
    \item \textbf{Curie-Weiss law}: The relationship
    \begin{equation*}
        \chi_M = \frac{C}{T-\theta}
    \end{equation*}
    where $\theta$ is the Weiss temperature. $\theta>0$ for a ferromagnet and $\theta<0$ for an antiferromagnet.
    \item Measuring and observing magnetism:
    \begin{itemize}
        \item Goes through the $\mu_S$ derivation on \textcite[119]{bib:CHEM20100Notes}.
        \item Arrives at the following important equations (defining our observables $\chi T$ and $\mu_\text{eff}$):
        \begin{align*}
            \mu_\text{eff} &= g\sqrt{S(S+1)}&
            \chi T &= \frac{g^2}{8}(S(S+1))
        \end{align*}
        \item Allows us to experimentally determine if a complex is high- or low-spin.
        \item However, experimental values actually deviate substantially from from the spin-only predicted values.
        \begin{itemize}
            \item This is because of \textbf{spin-orbit coupling}.
        \end{itemize}
    \end{itemize}
    \item \textbf{Spin-orbit coupling}: A phenomenon where a single electron hops between degenerate orbitals, generating a ring current that either reinforces or opposes the applied magnetic field. \emph{Also known as} \textbf{S.O.C.}
    \item To account for S.O.C., we need a new Hamiltonian with $L,S$.
    \begin{equation*}
        \mathcal{H} = \lambda L\cdot S
    \end{equation*}
    \begin{itemize}
        \item In the above equation, $\lambda$ is the S.O.C. constant.
        \item Because the equation contains the product of $L$ and $S$, $L$ and $S$ are no longer good quantum numbers. Thus, we need a new quantum number, namely $J$.
        \item Adding $J$ into our new Hamiltonian gives us
        \begin{equation*}
            \mathcal{H} = \lambda L\cdot S+\beta(L+g_eS)\cdot H
        \end{equation*}
        where $g_e$ is the free-electron $g$ value and $H$ is the applied magnetic field.
        \item Let's break the above Hamiltonian down.
        \begin{itemize}
            \item It tells us that the energy $E$ that we get from an applied magnetic field is
            \begin{equation*}
                E = \vec{B}\cdot\vec{m}
                = \mu\vec{m}\cdot\vec{H}
            \end{equation*}
            where $\vec{B}$ is the induced magnetic field and $\vec{m}$ is the magnetization.
            \item Additionally, we have that $\chi=\frac{m}{H}$, so
            \begin{equation*}
                \Delta m = -\dv{E_i}{H}
                = -E_i^1-2w_i^2\cdot H
            \end{equation*}
            \item Thus, the total energy $E_\text{total}$ is given by
            \begin{equation*}
                E_\text{total} = E_i^0+HE_i^1+H^2E_i^2
            \end{equation*}
            where $HE_i^1$ is the first-order term, and $H^2E_i^2$ is the second-order term.
        \end{itemize}
        \item We can break this down even further.
        \begin{itemize}
            \item We find that
            \begin{equation*}
                m_i = \dv{H}(E_i^0+E_i^0H+E_i^2H^2+\cdots)
                = -E_i^1+2E_i^2H
            \end{equation*}
            \item Additionally, we know that $\chi=\frac{m_i}{H}$, so
            \begin{equation*}
                \chi_i = \left( -\dv{E_i}{H} \right)\frac{1}{H}
            \end{equation*}
        \end{itemize}
        \item The final result above tells us that that the magnetic susceptibility depends on the magnetic field. This is rooted in the fact that there are different Boltzmann populations in different ligand field states.
        \item Now we can account for all of the Boltzmann populations and math that back to the observed $\chi$.
        \item To begin,
        \begin{equation*}
            \chi = \frac{N}{H}
            = \frac{\sum_i[-E_i^1-2E_i^2H]\exp\left( -\frac{E_i}{k_BT} \right)}{\sum_i\exp\left( \frac{E_i}{k_BT} \right)}
        \end{equation*}
        where $N$ is the number of atoms (if it's molar, $N$ is Avogadro's number).
        \item Additionally, we know that
        \begin{align*}
            \e[-E_i/k_BT] &= 1-\frac{E_i}{k_BT}\\
            &= \e[-(E_i^0+E_i^1H+E_i^2H^2+\cdots)/k_BT]\\
            &= \e[-E_i/k_BT]\left( 1-\frac{E_i^1H}{k_BT} \right)\left( 1-\frac{E_i^2H^2}{k_BT} \right)(\cdots)
        \end{align*}
        \begin{itemize}
            \item Note that if $E_i^1<<k_BT$, then $\e[-x]=1-x$ for small $x$. This is what allows us to get from the second to the third line above.
        \end{itemize}
        \item Combining the above two equations, we have
        \begin{equation*}
            \chi = \frac{N}{H}\cdot\frac{(-E_i^1-2E_i^2H)\e[-E_i/k_BT]\left( 1-\frac{E_i^1H}{k_BT} \right)\left( 1-\frac{E_i^2H^2}{k_BT} \right)}{\sum_i\e[-E_i^0k_BT]\left( 1-\frac{E_i^1H}{k_BT} \right)\left( 1-\frac{E_i^2H^2}{k_BT} \right)}
        \end{equation*}
        \item If we measure at a constant field, we can simplify the above to
        \begin{equation*}
            \chi = \frac{N\beta^2}{3k_BT}{\mu_\text{eff}}^2
        \end{equation*}
        \item This implies that
        \begin{equation*}
            \mu_\text{eff} = \sqrt{\frac{8+\left( \frac{3\lambda}{k_BT}-8 \right)\left( \exp(\frac{-3\lambda}{2k_BT}) \right)}{\frac{\lambda}{k_BT}\left[ 2+\exp(\frac{-3\lambda}{k_BT}) \right]}\cdot\beta^2}
        \end{equation*}
    \end{itemize}
    \item \textbf{Van Vleck equation}: The above equation\footnote{Having seen this equation is all that is required for this course (it's good to have seen it). The derivation will never come up again.}.
    \begin{itemize}
        \item It is only valid for $S=\frac{1}{2}$ (single electron terms).
        \begin{itemize}
            \item Multi-electron terms are too complicated to do by hand and must be handled by computer modeling.
        \end{itemize}
        \item You can explicitly account for S.O.C. with a known $\lambda$.
        \begin{itemize}
            \item For \ce{Ti^3+} for example, $\lambda=\SI{154}{\per\centi\meter}$.
            \item For \ce{Zr^3+} for example, $\lambda=\SI{500}{\per\centi\meter}$.
        \end{itemize}
    \end{itemize}
    \item S.O.C. constants increase for heavier elements. Elements with large relativistic effects have large S.O.C. constants.
    \item Intuitive insight into the conditions surrounding a large degree of spin-orbit coupling:
    \begin{itemize}
        \item An applied magnetic field causes the electron to rotate through the $d$-orbitals.
        \begin{itemize}
            \item If an electron hops between $d_{x^2-y^2}$ and $d_{xy}$ for instance, this creates rotation (and a ring current).
        \end{itemize}
        \item The ring current will create a magnetic field $B$ around the $z$ axis that opposes $H$.
        \item This opposing magnetic field gives a lower magnetic moment than spin-only calculations would predict.
        \item Note that rotation of a single electron in a $d^1$ complex lowers the magnetic moment, but rotation of a single "hole" (positive charge) in a $d^9$ complex raises the magnetic moment.
        \begin{itemize}
            \item In $d^5$ (and technically $d^{10}$), no electrons can move without violating the Pauli exclusion principle, so we would expect the observed moment to be closed to the spin-only value. (Quantum mechanical explanation: The electronic symmetry of a totally symmetric state precludes any mixing of the orbital angular momentum with the spin angular momentum.)
        \end{itemize}
    \end{itemize}
    \item A first approximation of S.O.C.:
    \begin{itemize}
        \item For a less than half-filled set of orbitals, expect $\mu_\text{eff}$ / $\chi T$ to be lower than the spin-only value, i.e., $g<2$.
        \begin{itemize}
            \item Typically $\mu_\text{eff}$ / $\chi T$ is close to the spin-only value.
        \end{itemize}
        \item For a greater than half-filled set of orbitals, expect $\mu_\text{eff}$ / $\chi T$ to be larger than the spin-only value, i.e., $g>2$.
        \begin{itemize}
            \item $\mu_\text{eff}$ / $\chi T$ can be much larger.
        \end{itemize}
        \item For a half-filled set of orbitals, we expect little to no S.O.C. Thus, $\mu_\text{eff}$ / $\chi T$ should be roughly equal to the spin-only value, i.e., $g\approx 2$.
    \end{itemize}
    \item The magnitude of S.O.C.: The key is orbital degeneracy.
    \begin{itemize}
        \item For example, if we have a $d^3$ \ce{Cr^3+} complex, we would predict more S.O.C. in the tetrahedral state (Figure \ref{fig:4-tetrab}) than in the octahedral state (Figure \ref{fig:6-octab}).
        \begin{itemize}
            \item This is because in the tetrahedral orbitals, there is only one electron in the upper orbitals; thus, we have a set of degenerate orbitals with a non-degenerate electron configuration, which promotes S.O.C. Note that $d^3$ octahedral is totally degenerate by contrast.
        \end{itemize}
        \item Note that this is totally dependent on the ligand field: Stronger ligand fields increasingly break degeneracy, thus quenching S.O.C.
    \end{itemize}
    \item S.O.C. trends:
    \begin{enumerate}
        \item Heavy elements have larger S.O.C.
        \item Lanthanides have larger S.O.C. from larger $\lambda$'s.
        \begin{itemize}
            \item Lanthanides in general also just have very large S.O.C. effects; this effect arises from the degenerate $f$ set.
            \begin{itemize}
                \item Since the $f$ orbitals do not extend beyond the noble gas core to engage in bonding, they do not split; thus, they are totally generate; thus, there is large S.O.C.
            \end{itemize}
            \item Note that neodymium magnets are strong because they combine a lanthanide that contributes massive S.O.C. effects, lots of electrons, and anisotropy (neodymium) and a transition metal that effectively couples with orbitals that aren't as buried as neodymium's $f$ orbitals (iron).
        \end{itemize}
    \end{enumerate}
\end{itemize}



\section{Office Hours (Anderson)}
\begin{itemize}
    \item \marginnote{4/5:}Electron counting for a metal center bonded to a $\pi$ bond?
    \begin{itemize}
        \item As in PSet 1 Questions 1.2 and 1.6.
        \item Example: \ce{Ni-}ethene.
        \begin{itemize}
            \item The $\pi$ electron density can donate to the nickel atom in a dative fashion.
            \item We can also think of two covalent bonds to the nickel; one from each carbon.
        \end{itemize}
        \item Example: \ce{Ni(COD)2}.
        \begin{itemize}
            \item Electron count: 18 (8 from the ligands [2 from each of 4 $\pi$ bonds] and 10 from the $d^{10}$ nickel atom).
        \end{itemize}
    \end{itemize}
    \item Electron counting for a metal center bonded to a $\sigma$ bond?
    \begin{itemize}
        \item Also donates two electrons datively, as in PSet 1 Question 1.9.
    \end{itemize}
    \item \ce{M-M} bonds and oxidation states?
    \begin{itemize}
        \item Reviews the discussion surrounding Figure \ref{fig:ficFragments}.
    \end{itemize}
    \item The binding of a Z-type ligand does not affect the electron count of a compound at all.
    \item The formal charges on the molecules in the PSet are the charges of the molecules as a whole, not any one part.
    \item \textbf{Metal-organic frameworks} or MOFs.
\end{itemize}



\section{Office Hours (Whitmeyer)}
\begin{itemize}
    \item \marginnote{4/6:}Transition metal trends: Why do first row transition metals prefer to have $1\,\e[-]$ coupled?
    \begin{itemize}
        \item Referring to reactivity? B/c higher oxidation states are more stable for early row transition metals.
    \end{itemize}
    \item Figure VI.13 of \textcite{bib:CHEM20100Notes} says that \emph{two} metal orbitals form bonding/antibonding orbitals with the ligand orbitals, so why does Dr. Anderson assert that only \emph{one} does? Is it because of what he said about $d_{z^2}$ being practically interchangeable with the $d_{xy,xz,yz}$ ligands in square planar complexes?
    \begin{itemize}
        \item Yes it is.
    \end{itemize}
    \item Now this question could just be a relic of my previous understanding of bonding, and the answer may just be "MO theory," but I'm still gonna ask: Where do the electrons in all of the bonds come from? It seems like if the osmiums are giving electrons to \ce{Os-Os} and \ce{Os-H} bonds, and we still count osmium as $d^8$, we are counting some electrons twice.
    \begin{itemize}
        \item It's fine to count them twice. The electron count is a computational tool, not a count of electrons.
        \item Does the electron count have physical meaning?
        \begin{itemize}
            \item Guides you toward what compounds are more reactive. And other chemical properties.
        \end{itemize}
    \end{itemize}
    \item What does isolobal mean?
    \item The fic fragment: Right spelling/charge?
    \item Does the iron in Figure \ref{fig:ficFragments} have tetrahedral or square planar geometry and why?
    \item Copies Table VI.2 from \textcite{bib:CHEM20100Notes}. Some differences?
    \item Why these specific spins to higher and lower energy (Figure \ref{fig:zeemanSplitting-1e})? Or is it arbitrary?
    \item What is a Boltzmann population?
    \item Differences in Curie's Law from last quarter?
    \item How do you define $d$ count? \textcite{bib:CBC} defines it as $d^n-x$ where $d^n$ is the ground state configuration and $x$ is the number of X-functions.
    \begin{itemize}
        \item The $d^n$ count should reflect the oxidation state.
    \end{itemize}
    \item Charges on compounds affect the oxidation state of the metal.
    \item Where do the extra electrons come from in the \ce{CO} resonance structures (the ones that are bonded to the metal)?
    \begin{itemize}
        \item All \ce{M-C} bonds are 2-electron? The IR stretching frequency just changes in some resonance structures so we call this a multiple \ce{M-C} bond even though it isn't.
    \end{itemize}
    \item How do we handle ligands with preexisting charges (i.e., \ce{NO+}, \ce{SO4^2-}, \ce{$\eta^3${-}C3H5+}, etc.)
\end{itemize}



\section{Discussion Section}
\begin{itemize}
    \item No discussion or office hours on the April 20; instead, there will be a special 2-hour exam review on April 15, 16, 17 (she will send a When2Meet poll). There will be 3 lectures per week, of 2-3 videos on Panopto. Office hours (one for Dr. Anderson and one for Sophie) will be one time per week. Please come to discussion: this is your question venue! Feel free to send along things you want discussed to Sophie in advance. Sophie can answer things much more thoroughly in discussion.
    \item Difference between L- and X-type ligands:
    \begin{itemize}
        \item Draw out lone pairs. Break bonds (pushing electrons back onto ligands). If the ligand is neutral, it's L-type. If it's negatively charged, it's X-type.
        \begin{itemize}
            \item So if you have an oxygen as a singly bonded ligand and break the bond, oxygen will be negative. If you do the same thing with nitrogen, it will be neutral.
            \emph{draw a picture}
        \end{itemize}
    \end{itemize}
    \item When dealing with $f$-block compounds, be careful about the periodic placement (different periodic tables place lanthanium and actinium in different places).
    \item Why $f$-block electrons for the electron count?
    \begin{itemize}
        \item The lanthanides and actinides don't follow the 18 electron rule.
        \item This is a very active area of research (at the forefront of bonding actually), but the growing consensus is that\dots
        \begin{itemize}
            \item The lanthanides do not engage in covalent bonding; they're purely ionic.
            \item The actinides can engage in limited covalent bonding with their $f$ block because their larger $f$ block extends a little bit further.
        \end{itemize}
    \end{itemize}
    \item We can do isolobal analogies with bridging ligands, too!
    \item One should never \emph{assume} that there's a \ce{M-M} bond, but one should always \emph{wonder} if there's one.
    \item Potentially you can invoke $\pi$ donation from a ligand to get a compound to 18 electrons.
    \item \textbf{Hapticity}: The number of carbons in the $\pi$ system. \emph{Also known as} $\bm{\eta}$.
\end{itemize}




\end{document}