\documentclass[../notes.tex]{subfiles}

\pagestyle{main}
\renewcommand{\chaptermark}[1]{\markboth{\chaptername\ \thechapter\ (#1)}{}}
\setcounter{chapter}{2}

\begin{document}




\chapter{???}
\section{Lecture 12: Intro to Catalysis}
\begin{itemize}
    \item \marginnote{4/26:}We're now moving from theoretical chemistry to some applications (namely catalysis) of the theory we've been learning.
    \item History of defining catalysts:
    \begin{itemize}
        \item Berzelius (in 1836) becomes interested in this behavior.
        \item Ostwald (in 1894) defines a \textbf{catalyst}.
    \end{itemize}
    \item \textbf{Catalyst}: A substance that increases the rate of a chemical reaction without being consumed.
    \item Energetically, this must happen by altering the transition state (this is a simplistic explanation).
    \begin{itemize}
        \item The thermodynamics ($\Delta G$, $\Delta S$, and $\Delta H$) are unaffected.
        \item The kinetics ($\Delta G^\ddagger$, $\Delta S^\ddagger$, and $\Delta H^\ddagger$) are reduced.
    \end{itemize}
    \item More realistically, a catalyst often substantially changes the reaction coordinate (one big hump in the energy diagram to many small humps).
    \begin{itemize}
        \item The general set of intermediates during the reaction is the starting material plus the catalyst ($S+C$), the starting material-catalyst complex ($S\cdot C$), the product-catalyst complex ($P\cdot C$), and the product plus the catalyst ($P+C$).
        \item Note that $\Delta G^\ddagger$ is the activation energy for the highest barrier step, as measured against the free energy of the reactants. This notably implies that $\Delta G^\ddagger$ is not necessarily the $E_A$ of the \emph{first} step; only the biggest. See Figure \ref{fig:catalysisEnergyDiagramsb}.
        \item Even though there are more steps, the rate increases because $\Delta G^\ddagger$ decreases.
    \end{itemize}
    \item If you get stuck at a low energy intermediate, this can reduce reaction rate, and the process is no longer being catalyzed.
    \begin{itemize}
        \item Indeed, if your catalyst is a different structure at the end of the reaction, it's not a catalyst but a reagent.
        \item Excessively stabilizing the starting materials can create a higher energy barrier to the products.
    \end{itemize}
    \item Example:
    \begin{figure}[H]
        \centering
        \schemestart
            \chemfig{R-[:30]=[:-30]}
            \arrow{->[\small\ce{HB(OR)2}]}[,1.4]
            \chemfig{R-[:30](-[2]H)-[:-30]-[:30]B{(}OR{)_2}}
        \schemestop
        \caption{A process to be catalyzed.}
        \label{fig:reactionBeforeCatalysis}
    \end{figure}
    \begin{figure}[h!]
        \centering
        \begin{subfigure}[b]{0.98\linewidth}
            \centering
            \begin{tikzpicture}[
                every node/.style={black}
            ]
                \footnotesize
                \draw [very thin] (0,2.2) -- (4.8,2.2);
                \draw [<->] (0.5,0) -- node[right]{$\Delta G^\ddagger$} (0.5,2.2);
    
                \draw [blx,thick] (0,0)
                    -- node[below]{
                        \schemestart
                            \chemfig{R-[:30]=[:-30]}
                            \arrow(.south--.north){0[*{90}+]}[-90,0.4]
                            \chemfig{HB{(}OR{)_2}}
                        \schemestop
                    } (1,0)
                    to[out=0,in=180] (2,0.6)
                    to[out=0,in=180] (3,-0.2) node[below]{
                        \chemfig{R-[:30]=[:-30]-[:150,0.4,,,white]\phantom{i}-[:60,,,,dashed]B{(}OR{)_2}-[:160,,1,,dashed]H}
                    }
                    to[out=0,in=180,out looseness=0.6,in looseness=0.8] (4.8,2.2)
                    to[out=0,in=180,out looseness=0.8] (7,-1.5)
                    -- node[below]{
                        \chemfig{R-[:30](-[2]H)-[:-30]-[:30]B{(}OR{)_2}}
                    } (8,-1.5)
                ;
            \end{tikzpicture}
            \caption{Before catalysis.}
            \label{fig:catalysisEnergyDiagramsa}
        \end{subfigure}\\
        \vspace{2em}
        \begin{subfigure}[b]{0.98\linewidth}
            \centering
            \begin{tikzpicture}[
                xscale=1.3,
                every node/.style={black}
            ]
                \footnotesize
                \draw [very thin] (0,0.8) -- (8.5,0.8);
                \draw [<->] (0.5,0) -- node[right]{$\Delta G^\ddagger$} (0.5,0.8);
    
                \draw [rex,thick] (0,0)
                    -- node[below]{
                        \schemestart
                            \chemfig{R-[:30]=[:-30]}
                            \arrow(.south--.north){0[*{90}+]}[-90,0.4]
                            \chemfig{HB{(}OR{)_2}}
                            \arrow(.south--.north){0[*{90}+]}[-90,0.4]
                            \chemfig{M}
                        \schemestop
                    } (1,0)
                    to[out=0,in=180] (2,0.6)
                    to[out=0,in=180] (3,-0.3) node[below]{
                        \schemestart
                            \chemfig{R-[:30]=[:-30]}
                            \arrow(.south--.north){0[*{90}+]}[-90,0.4]
                            \chemfig{M(-[1]B{(}OR{)_2})(-[7]H)}
                        \schemestop
                    }
                    to[out=0,in=180] (4,0.4)
                    to[out=0,in=180] (5,-0.7) node[below]{
                        \chemfig{M(-[1]B{(}OR{)_2})(-[7]H)(-[6]\phantom{i}-[,0.4,,,white]=[4]-[:-120]R)}
                    }
                    to[out=0,in=180] (6,-0.1)
                    to[out=0,in=180] (7,-1.1) node[below]{
                        \chemfig{M(-[1]B{(}OR{)_2})(-[6]-[:-150]-[6]R)}
                    }
                    to[out=0,in=180] (8.5,0.8)
                    to[out=0,in=180] (10,-1.5)
                    -- node[below]{
                        \schemestart
                            \chemfig{R-[:30](-[2]H)-[:-30]-[:30]B{(}OR{)_2}}
                            \arrow(.south--.north){0[*{90}+]}[-90,0.4]
                            \chemfig{M}
                        \schemestop
                    } (11,-1.5)
                ;
            \end{tikzpicture}
            \caption{With catalysis.}
            \label{fig:catalysisEnergyDiagramsb}
        \end{subfigure}
        \caption{Catalysis energy diagrams.}
        \label{fig:catalysisEnergyDiagrams}
    \end{figure}
    \begin{itemize}
        \item With transition metal catalysts, we have many more steps but potentially a lower $\Delta G^\ddagger$.
    \end{itemize}
    \item Anatomy/terminology of a catalytic cycle.
    \begin{figure}[h!]
        \centering
        \begin{subfigure}[b]{0.35\linewidth}
            \centering
            \begin{tikzpicture}[
                every path/.append style={blx,semithick},
                every node/.append style={black}
            ]
                \footnotesize
                \node (1) at (90:2) {\ce{L_nM}};
                \node (2) at (0:2.2) {\ce{L_nM*S}}
                    edge [stealth-,out=90,in=0] (1)
                ;
                \node (3) at (-90:2) {\ce{L_nM*I}}
                    edge [stealth-,out=0,in=-90] (2)
                ;
                \node (4) at (180:2.2) {\ce{L_nM*P}}
                    edge [stealth-,out=-90,in=180] (3)
                    edge [-stealth,out=90,in=180] (1)
                ;
    
                \draw [-stealth] (135:2.24) to[out=45,in=-90] (-1,3) node[above]{\ce{P}};
            \end{tikzpicture}
            \caption{A typical catalytic cycle.}
            \label{fig:catalyticCyclesa}
        \end{subfigure}
        \begin{subfigure}[b]{0.63\linewidth}
            \centering
            \begin{tikzpicture}[
                every path/.append style={blx,semithick},
                every node/.append style={black}
            ]
                \footnotesize
                \node (1) at (90:2) {\ce{L_nM}};
                \node (2) at (0:2.2) {\ce{L_nM*S}}
                    edge [stealth-,out=90,in=0] (1)
                ;
                \node (3) at (-90:2) {\ce{L_nM*I}}
                    edge [stealth-,out=0,in=-90] (2)
                ;
                \node (4) at (180:2.2) {\ce{L_nM*P}}
                    edge [stealth-,out=-90,in=180] (3)
                ;
    
                \node (5) at (0:4.2) {\ce{[L_nM*S]2}};
                \draw [arrows={-Stealth[harpoon]}] (2.5) -- (5.176);
                \draw [arrows={-Stealth[harpoon]}] (5.-176) -- (2.-5);
                \node (6) at (180:4.2) {\ce{L_nMX}};
                \draw [-stealth] (4) -- node[above]{\ce{X*}} (6);
    
                \draw [-stealth] (-1,3) node[above]{\ce{L_nMX}} to[out=-90,in=150] node[above right]{activation} (1.north west);
            \end{tikzpicture}
            \caption{An obstructed catalytic cycle.}
            \label{fig:catalyticCyclesb}
        \end{subfigure}
        \caption{The anatomy of catalytic cycles.}
        \label{fig:catalyticCycles}
    \end{figure}
    \begin{itemize}
        \item Simplistically, a catalytic cycle occurs as in Figure \ref{fig:catalyticCyclesa}.
        \item However, catalytic cycles can be complicated by \textbf{catalyst precursors}, \textbf{inactive off-cycles}, and \textbf{inactive poisoned states}.
    \end{itemize}
    \item \textbf{Catalyst precursor}: A complex that must go through an activation step before it can be used as a catalyst.
    \begin{itemize}
        \item See the reactant that becomes \ce{L_nM} in Figure \ref{fig:catalyticCyclesb}.
    \end{itemize}
    \item \textbf{Inactive off-cycle}: A reversible reaction that an intermediate participates in that is different from the intended reaction.
    \begin{itemize}
        \item See the alternate pathway that \ce{L_nM*S} participates in in Figure \ref{fig:catalyticCyclesb}.
    \end{itemize}
    \item \textbf{Inactive poisoned state}: When an intermediate follows an alternate nonreversible reaction pathway.
    \begin{itemize}
        \item See the alternate pathway that \ce{L_nM*P} participates in in Figure \ref{fig:catalyticCyclesb}.
    \end{itemize}
    \item Gives two examples of activating a catalyst (just a chemical reaction where one product is the catalyst).
    \begin{itemize}
        \item A \ce{Pd^2+} precursor must often be activated into a \ce{Pd^0} catalyst.
    \end{itemize}
    \item \textbf{Turnover number}: The quotient of the moles of product and the moles of catalyst. \emph{Also known as} \textbf{TON}.
    \item \textbf{Turnover frequency}: The quotient of the TON and time. \emph{Also known as} \textbf{TOF}.
    \item Kinetics:
    \begin{itemize}
        \item The rate constant for individual steps can vary, but at a steady state (where the catalyst is transforming the substrate as fast as possible), the rates must be identical.
    \end{itemize}
    \item \textbf{Catalyst resting state}: The highest concentration form of the catalyst.
    \begin{itemize}
        \item Can also be off-cycle or dormant.
        \item The rate constant after it (i.e., the rate constant of the \textbf{turnover limiting step}) is the smallest among all rate constants for steps.
    \end{itemize}
    \item \textbf{Turnover limiting step}: The step which proceeds from the catalyst resting state. \emph{Also known as} \textbf{rate-determining step}.
    \item Heterogeneous vs. homogeneous catalysis:
\end{itemize}
\begin{tchart}{1.4}{Heterogeneous}{Homogeneous}
    Solid state (2 phases). & Solution phase.\\
    Robust (high pressures and temperatures are ok). & Selective.\\
    Low-cost. & Tunable.\\
    Easy separation. & Easy to study (comparatively).
\end{tchart}
\begin{itemize}
    \item We distinguish between the two with \textbf{transmission-electron microscopy}, kinetics, a \textbf{filtration test}, a \textbf{mercury drop test}, and/or a \textbf{three-phase test}.
    \item \textbf{Transmission-electron microscopy}: Looks for nanoparticles. \emph{Also known as} \textbf{TEM}.
    \item Kinetics: We need to observe soluble intermediates and show that they are \textbf{kinetically competent}.
    \item \textbf{Kinetically competent}: The step from the soluble intermediates to the products under another reagent must be at least as fast as the overall rate of catalysis.
    \item \textbf{Filtration test}: For heterogeneous catalysts. Do the reaction over the heterogeneous catalyst, filter out the solid heterogeneous catalyst, add in more substrate and see if the mother liquor or supernatant still catalyzes the reaction. If so, then something is leaching out of the catalyst.
    \item \textbf{Mercury drop test}: Mercury can typically block the pores of high-surface area catalysts or poison the cycle by forming TM alloys. \emph{Also known as} \textbf{Hg-drop test}.
    \item \textbf{Three-phase test}: Attach a substrate to an insoluble support. If the catalyst is solid state, we see no reaction (poor phase mixing).
    \begin{itemize}
        \item The gold standard.
    \end{itemize}
    \item Asymmetric catalysis:
    \begin{itemize}
        \item Propene is pro-chiral (it has a Re and a Si face) at the central carbon.
    \end{itemize}
    \item \textbf{Dynamic kinetic resolution}: When a racemic mixture goes to an enantioenriched product.
    \begin{itemize}
        \item Occurs when an achiral intermediate is accessed and selectively filtered into an enantioenriched product.
    \end{itemize}
    \item \textbf{Kinetic resolution}. When a racemic mixture goes to an enantioenriched starting material and enantioenriched product.
    \begin{itemize}
        \item Arises from selective reaction with one enantiomer over another.
    \end{itemize}
    \item \textbf{Enantiomeric excess}: The quantity given by the following formula. \emph{Also known as} \textbf{ee}.
    \begin{equation*}
        \frac{\ce{[{major en.}]}-\ce{[{minor en.}]}}{\ce{[{major en.}]}+\ce{[{minor en.}]}}\times 100\%
    \end{equation*}
    \item Enantiomeric ratio:
    \begin{equation*}
        \frac{\ce{[{major en.}]}}{\ce{[{minor en.}]}} = \e[-\Delta\Delta G^\ddagger/RT]
    \end{equation*}
    \begin{itemize}
        \item Note that $\Delta\Delta G^\ddagger$ is the difference in the $\Delta G^\ddagger$ values for the major and minor enantiomers.
    \end{itemize}
    \item Ee is dependent on $\Delta\Delta G^\ddagger$ and $T$:
    \begin{figure}[h!]
        \centering
        \begin{subfigure}[b]{0.49\linewidth}
            \centering
            \begin{tikzpicture}[scale=1.3]
                \small
                \draw (4,0) -- node[below=5mm]{$\Delta\Delta G^\ddagger$} (0,0) -- node[left=7mm]{ee} (0,3.3);
                \footnotesize
                \node [anchor=north east] {0};
                \foreach \x in {0.5,1,...,3} {
                    \draw (\x,0.1) -- ++(0,-0.2) node[below]{$\x$};
                }
                \foreach \y/\val in {0.75/25,1.5/50,2.25/75,3/100} {
                    \draw (0.1,\y) -- ++(-0.2,0) node[left]{$\val$};
                }
    
                \draw [blx,thick] plot[domain=0:3.7,smooth] (\x,{3-3*e^(-\x)});
            \end{tikzpicture}
            \caption{Dependence on $\Delta\Delta G^\ddagger$.}
            \label{fig:ee-DeltaDelta-Ta}
        \end{subfigure}
        \begin{subfigure}[b]{0.49\linewidth}
            \centering
            \begin{tikzpicture}[
                scale=1.3,
                every node/.style={black}
            ]
                \small
                \draw (-3,0) -- node[pos=0.6,below=5mm]{$T\,\si{\celsius}$} (2,0);
                \draw (0,0) -- (0,3.3) node[above]{ee};
                \node [align=center] at (1.8,3.3) {$\Delta\Delta G^\ddagger$\\$(\si[per-mode=symbol]{kcal\per\mole})$};
                \footnotesize
                \node [below=1mm] {0};
                \foreach [evaluate=\x as \val using {int(\x*40)}] \x in {-2.5,-2,-1.5,-1,-0.5,0.5,1,1.5} {
                    \draw (\x,0.1) -- ++(0,-0.2) node[below]{$\val$};
                }
    
                \draw [blx,thick]
                    (-2.5,2.95) -- (1.5,2.75) node[right]{1.6}
                    (-2.5,2.75) -- (1.5,2.4) node[right]{1.4}
                    (-2.5,2.5) -- (1.5,2.1) node[right]{1}
                    (-2.5,0.8) -- (1.5,0.3) node[right]{0.2}
                ;
    
                \foreach \y/\val in {0.75/25,1.5/50,2.25/75,3/100} {
                    \draw (0.1,\y) -- ++(-0.2,0) node[left,fill=white,inner sep=2pt]{$\val$};
                }
            \end{tikzpicture}
            \caption{Dependence on $T$.}
            \label{fig:ee-DeltaDelta-Tb}
        \end{subfigure}
        \caption{Dependence of ee on $\Delta\Delta G^\ddagger$ and $T$.}
        \label{fig:ee-DeltaDelta-T}
    \end{figure}
    \item Some enantiomers have differing catalytic properties.
    \begin{itemize}
        \item Furthermore, chirality is not necessarily set at the RDS.
        \item If A and B are two enantiomers, then it is possible that $\Delta G^\ddagger$ will differ for the formation of \ce{M*A + B} vs. \ce{M*B + A} (remember that this difference is $\Delta\Delta G^\ddagger$).
        \item If the activation energy required to form \ce{M*A + B} (for example) is greater than that required to get from \ce{M*A + B} to the products, formation of the \ce{M*A + B} intermediate will be irreversible.
        \begin{itemize}
            \item In this case, the kinetic product will be favored.
        \end{itemize}
        \item If the activation energy required to form \ce{M*A + B} (for example) is less than that required to get from \ce{M*A + B} to the products, formation of the \ce{M*A + B} intermediate will be reversible.
        \begin{itemize}
            \item In this case, the thermodynamic product will be favored (we may see more of the intermediate that forms quickly, but in time, more of the other product will be formed because formation of the other product is irreversible\footnote{Think the Mrs. Meer pens example.}).
        \end{itemize}
    \end{itemize}
\end{itemize}



\section{Office Hours (Anderson)}
\begin{itemize}
    \item What does kinetically competent mean?
    \begin{center}
        \begin{tikzpicture}
            \node{
                \schemestart
                    \chemfig{[:45]=}
                    \arrow{0}[,0]\+ {$\tfrac{1}{2}$}
                    \chemfig{O_2}
                    \arrow{->[cat]}
                    \chemfig{-[:30](=[2]O)-[:-30]H}
                \schemestop
            };
    
            \draw [-stealth] (-1.78,-0.3) -- (-1.78,-1.5) -- (-0.6,-1.5);
            \node at (0.1,-1.5) {\chemfig{?-[,1.2]-[:120]O?}};
            \draw [-stealth] (0.8,-1.5) -- (1.8,-1.5) -- (1.8,-0.7);
    
            \node at (0,-3.5) {
                \schemestart
                    \chemfig{?-[,1.2]-[:120]O?}
                    \arrow{->[cat]}
                    \chemfig{-[:30](=[2]O)-[:-30]H}
                \schemestop
                \quad\LARGE ?
            };
        \end{tikzpicture}
    \end{center}
    \begin{itemize}
        \item Consider the reaction of ethylene and \ce{$\frac{1}{2}$O2} with a catalyst to make acetaldehyde.
        \item We propose that the intermediate is an epoxide.
        \item To test this hypothesis, we take ethylene oxide and react it with the catalyst to see if it makes acetaldehyde.
        \item If it does, that's a good first step. If it reacts with a rate at least as fast or faster than the overall catalyzed reaction, then we know that this intermediate \emph{is} an intermediate in a catalytic cycle.
        \item Essentially, this test confirms that an intermediate is one. If the rate of epoxide to product is at least as fast as the overall rate, then we're good.
    \end{itemize}
    \item What do all of those tests test for? I.e., if we find nanoparticles with TEM, what does this mean?
    \begin{itemize}
        \item We need all of these tests because none of them are definitive.
        \item You can never prove a mechanism; you can only disprove it.
        \item If you see nanoparticles with TEM, it might make you ask if a heterogeneous pathway is present.
        \item You go through these experiments to try to determine that it's \emph{not} homogeneous or \emph{not} heterogeneous.
    \end{itemize}
    \item Resources for synthesis?
    \begin{itemize}
        \item Inorganic is not like organic where we're pushing electrons and stuff.
        \item Don't worry about step efficiency.
        \item Don't worry about reactive fragments.
        \item For syntheses, completely ignore mechanism. Just balance the reaction.
        \begin{itemize}
            \item Balance in terms of electrons and atoms.
            \item Just focus on stoichiometry to start.
            \item $99\%$ of the time, this will get you to the right conclusion.
        \end{itemize}
        \item If you have methyl iodide and you want to methylate the metal center, you want to make the metal nucleophilic and then react it.
        \item If you want to be really lazy, just show the reagents that will give you the products and be done.
        \begin{itemize}
            \item You'll probs lose a few points for this, but that's ok.
            \item Try and show the intermediates.
        \end{itemize}
    \end{itemize}
    \item Time crunch?
    \begin{itemize}
        \item Skip a question.
        \item Go through the easiest questions first.
        \item You can probably get $80\%$ of the points on the first question without writing a single synthesis.
        \item Do the easy things fast and then move on.
        \item View it as a scavenger hunt for points.
    \end{itemize}
\end{itemize}



\section{Discussion Section}
\begin{itemize}
    \item \marginnote{4/27:}There may be a midterm key.
    \item HW3 will be due Monday 5/3/2021 at 5:00 PM; HW4 will be due Monday 5/10/2021 at 12:00 PM (the usual time, no changes).
    \item Notes on cross coupling:
    \begin{itemize}
        \item Mizoroki-Heck cross couplings are characterized by a migratory insertion into an olefin in place of transmetallation.
        \item In transmetallation, we typically use Grignards, boronic acids (Negishi), tin reagents (Stille), or silylethers.
        \item $\alpha$ and $\beta$ elimination (commonly hydride) can give competing reactions with the $\beta$-hydrogens especially.
    \end{itemize}
    \item Alkyls (such as ethyl groups) are more electron-rich than aryls (such as phenyl groups).
    \begin{itemize}
        \item Think Homework 2, Problem 12.
    \end{itemize}
    \item Goes over the arrow pushing from a ring-closing and ring-opening metathesis (very similar to Figure \ref{fig:mechanism-enyneMet}).
    \item Note the use of a Grignard in the basic cross coupling mechanism.
\end{itemize}



\section{Lecture 13: Cross Coupling}
\begin{itemize}
    \item \marginnote{4/28:}Probably the single most-employed catalytic reaction in synthetic chemistry.
    \begin{itemize}
        \item More important in pharmaceuticals and fine chemical synthesis, not industrial processes.
        \item Nobel prize (2010) to Heck, Negishi, and Suzuki for aryl-type couplings.
    \end{itemize}
    \item Dominated by \ce{Pd^0} / \ce{Pd^{II}} ("God's Metal" - John Bercaw).
    \begin{itemize}
        \item There are also examples with \ce{Ni}, \ce{Rh}, \ce{Ir}, and \ce{Cu}.
        \item These metals are so good because the reactivity is often based in oxidative addition and reductive elimination, but radical reactions are also useful (particularly with nickel).
    \end{itemize}
    \item General form:
    \begin{equation*}
        \ce{R-X + R$'$-M ->[{cat.}] R-R$'$ + M-X}
    \end{equation*}
    \begin{itemize}
        \item Basically a nucleophilic attack.
        \begin{itemize}
            \item But since substrates like \ce{ArI} don't have very good reactivity for nucleophilic aromatic substitutions (despite the fact that iodide is a great leaving group), we need this mechanism.
        \end{itemize}
        \item \ce{R}, \ce{R$'$} are typically carbon-based, but carbon-heteroatom (where the heteroatom is \ce{N}, \ce{O}, \ce{S}, or something else) couplings are also advancing.
    \end{itemize}
    \item History:
    \begin{itemize}
        \item Kumada and Corriu report \ce{ArBr + RMgX ->[Et2O, $\SI{25}{\celsius}$][{\emph{substance}}] Ph-C=C-R + R$'$-C=C-R}\footnote{Note that Anderson often uses $\emptyset$ instead of \ce{Ph} to denote a phenyl group.}.
        \begin{itemize}
            \item The \emph{substance} is \ce{Ni(acac)2}, dppe, \ce{NiCl2}, \ce{FeCl3}, \ce{CoCl2}, or \ce{CrCl2}.
            \item Note that the aryl group can be phenyl, but it can also be vinylic or allylic (basically any $sp^2$-hybridized carbon).
        \end{itemize}
    \end{itemize}
    \item There are many named reactions (usually defined by the nucleophile).
    \begin{itemize}
        \item This is a con of this field.
        \item Negishi: \ce{RZnX}.
        \item Stille: \ce{RSnR3$'$}.
        \item Hiyama: \ce{RSiMe2F}.
        \item Suzuki: \ce{RB(OH)2} or \ce{RBF3-}.
        \item Mizoroki/Heck: \ce{Ph-Br + C=C-R ->[cat][base] Ph-C=C-R}.
    \end{itemize}
    \item Alkynes can also be coupled.
    \item Enolates can also be used as nucleophiles.
    \item Most cross coupling reactions involve aryl nucleophiles, but we can also use aliphatic ones.
    \item Aliphatic couplings:
    \begin{itemize}
        \item Alkyl electrophiles: \ce{CRR$'$X + Ar-M ->[{Ni or Pd}] CRR$'$Ar}.
        \begin{itemize}
            \item These frequently go through a radical mechanism.
            \item $\beta$-\ce{H} can be a problem.
        \end{itemize}
    \end{itemize}
    \item Cyanide couplings:
    \begin{itemize}
        \item \ce{Ph-X + M(CN)_n} or \ce{CMe2(CN)(OH)} can react with a palladium catalyst to form \ce{Ph-CN}.
        \begin{itemize}
            \item Note that $\ce{M}=\ce{K},\ce{Zn},\ce{K4Fe(CN)6}$.
        \end{itemize}
    \end{itemize}
    \item Enantioselective cross couplings:
    \begin{itemize}
        \item The nucleophile racemizes quickly.
        \item $93\%$ ee with nickel.
        \item $99\%$ ee with palladium.
    \end{itemize}
    \item Homo-couplings:
    \begin{itemize}
        \item An alkyl/aryl halide plus that same substance but as a Grignard forming a "dimer" at the place where the Grignard is attached.
        \item Use nickel as a catalyst here.
    \end{itemize}
    \item Heteroarene coupling partners:
    \begin{figure}[h!]
        \centering
        \schemestart
            \chemfig{[:-18]*5(-X-=--)}
            \+
            \chemfig{PhOTf}
            \arrow{->[Pd][base]}
            \chemfig{[:-18]*5(-X-(-Ph)-=-)}
        \schemestop
        \caption{Heteroarene coupling partners.}
        \label{fig:heteroareneCoupling}
    \end{figure}
    \begin{itemize}
        \item Note that \ce{TfO} is triflate, an abbreviation of trifluoromethanesulfonate, which is an excellent leaving group.
        \item Also with enolates.
    \end{itemize}
    \item Mechanism:
    \begin{figure}[h!]
        \centering
        \begin{subfigure}[b]{0.49\linewidth}
            \centering
            \begin{tikzpicture}
                \footnotesize
                \node (1) at (90:2) {\ce{L_{n-1}Pd^0}};
                \node (2) at (0:2.2) {\chemfig{{\ce{L_{n-1}}Pd^{II}(-[1]Ar)(-[7]X)}}}
                    ([yshift=-4mm]2.north) edge [blx,semithick,stealth-,out=90,in=0] node[black,above right]{\ce{Ar-X}} (1)
                ;
                \node (3) at (180:2.2) {\chemfig{{\ce{L_{n-1}}Pd^{II}(-[1]Ar)(-[7]Nu)}}}
                    ([yshift=4mm]3.south) edge [blx,semithick,stealth-,out=-90,in=-90,looseness=1.4] ([yshift=5mm]2.south)
                    ([yshift=-4mm]3.north) edge [blx,semithick,-stealth,out=90,in=180] (1)
                ;
    
                \draw [blx,semithick,-stealth] (150:2.35) to[out=60,in=-90] (-1.75,2.5) node[black,above]{\ce{Ar-Nu}};
    
                \node (4) at (90:3.4) {\ce{L_nPd^0}};
                \draw [blx,semithick,arrows={-Stealth[harpoon]}] (1.98) -- node[black,left]{\ce{L}} (4.-98);
                \draw [blx,semithick,arrows={-Stealth[harpoon]}] (4.-82) -- node[black,right]{\ce{-L}} (1.82);
    
                \node (5) [align=center] at (2.1,-3.2) {\ce{Nu-H}, base\\ or\\ \ce{Nu-M}};
                \node (6) [align=center] at (-2,-3.2) {\ce{H-X}\\ or\\ \ce{M-X}}
                    (6.45) edge [blx,semithick,stealth-,bend left=33] (5.150)
                    (6.144) edge [blx,semithick,-stealth] node[black,above]{base} ++(-1,0)
                ;
            \end{tikzpicture}
            \caption{Basic.}
            \label{fig:mechanismCCa}
        \end{subfigure}
        \begin{subfigure}[b]{0.49\linewidth}
            \centering
            \begin{tikzpicture}
                \footnotesize
                \node (1) at (90:2) {\ce{L_nNi^IX}};
                \node (2) at (0:2.2) {\chemfig{{\ce{L_n}Ni^{III}(-[:30]Ar)(-[6,,2]X)(-[:-30]X)}}}
                    ([yshift=-4mm]2.north) edge [blx,semithick,stealth-,out=90,in=0] node[black,above right]{\ce{Ar-X}} (1)
                ;
                \node (3) at (180:2) {\chemfig{{\ce{L_n}Ni^{III}(-[:30]Ar)(-[6,,2]X)(-[:-30]Ar)}}}
                    (3.south) edge [blx,semithick,stealth-,out=-90,in=-90,looseness=1.4] (2.south)
                    ([yshift=-4mm]3.north) edge [blx,semithick,-stealth,out=90,in=180] (1)
                ;
    
                \draw [blx,semithick,-stealth] (150:2.2) to[out=65,in=-90] (-1.6,2.5) node[black,above]{\ce{Ar-Ar}};
    
                \node (5) at (2.2,-4.7) {\chemfig{L_\emph{n}Ni(-[1]Ar)(-[7]X)}};
                \node (6) at (-1.8,-4.7) {\chemfig{L_\emph{n}Ni(-[1]X)(-[7]X)}}
                    ([xshift=-1mm,yshift=-4mm]6.north) edge [blx,semithick,stealth-,out=90,in=90,looseness=1.4] ([xshift=-1mm,yshift=-4mm]5.north)
                ;
                \node (7) at (0.2,-6.7) {\ce{L_nNi^0}}
                    edge [blx,semithick,stealth-,out=180,in=-90] ([xshift=-1mm,yshift=4mm]6.south)
                    edge [blx,semithick,-stealth,out=0,in=-90] node[black,below right]{\ce{Ar-X}} ([xshift=-1mm,yshift=4mm]5.south)
                ;
    
                \node (8) at (-2.8,-5.7) {\ce{M^0}};
                \node (9) at (-0.8,-7.7) {\ce{MX2}}
                    edge [blx,semithick,stealth-,bend right=36] (8)
                ;
            \end{tikzpicture}
            \caption{Homocoupling.}
            \label{fig:mechanismCCb}
        \end{subfigure}\\[2em]
        \begin{subfigure}[b]{0.8\linewidth}
            \centering
            \begin{tikzpicture}
                \footnotesize
                \node (1) at (90:3) {\ce{L_{n-1}Pd^0}};
                \node (2) at (18:3) {\chemfig{{\ce{L_{n-1}}Pd(-[1]Ar)(-[7]X)}}}
                    edge [blx,semithick,stealth-,bend right=30] node[black,above right]{\ce{Ar-X}} (1)
                ;
                \node (3) at (-54:3) {\chemfig{{\ce{L_{n-2}}Pd(-[:30]Ar)(-[6,,2]\phantom{i}-[4,0.4,,,white]=-[:-60]R)(-[:-30]X)}}}
                    edge [blx,semithick,stealth-,bend right=15] node[black,left]{\ce{-L}} node[black,thin,-,right]{\chemfig{[:-30]=[:30]-R}} (2)
                ;
                \node (4) at (-126:3) {\chemfig{{\ce{L_{n-2}}Pd(-[:30](-[2]R)-[:-30]-[:30]Ar)(-[7]X)}}}
                    edge [blx,semithick,stealth-,bend right=15] (3)
                ;
                \node (5) at (162:3) {\chemfig{{\ce{L_{n-2}}Pd(-[1]X)(-[7]H)}}}
                    edge [blx,semithick,stealth-,bend right=19] (4)
                    edge [blx,semithick,-stealth,bend left=30] node[black,below right]{L, base} (1)
                ;
    
                \draw [blx,semithick,-stealth] (-162:2.924) to[out=110,in=-15] (-4,0) node[black,thin,-,left]{\chemfig{R-[:30]=[:-30]-[:30]Ar}};
    
                \draw [blx,semithick,-stealth] (140:3.08) to[out=58,in=-87] (-1.8,3.8) node[black,above]{\ce{H-X}};
    
                \node (6) at (90:4.4) {\ce{L_nPd^0}};
                \draw [blx,semithick,arrows={-Stealth[harpoon]}] (1.98) -- node[black,left]{\ce{L}} (6.-98);
                \draw [blx,semithick,arrows={-Stealth[harpoon]}] (6.-82) -- node[black,right]{\ce{-L}} (1.82);
            \end{tikzpicture}
            \caption{Mizoroki-Heck.}
            \label{fig:mechanismCCc}
        \end{subfigure}
        \caption{Cross coupling mechanisms.}
        \label{fig:mechanismCC}
    \end{figure}
    \begin{itemize}
        \item Basic (see Figure \ref{fig:mechanismCCa}):
        \begin{itemize}
            \item We must first activate our palladium catalyst by causing it to lose a ligand, thus becoming lower-coordinate.
            \item Our aryl hydride will now oxidatively add to the activated catalyst.
            \item Next, we have transmetallation (the most complicated step of the cycle). There are two possible paths by which we can introduce our nucleophile into the system and remove an X (to make space for it). First, we can use a hydride of the nucleophile which will do what we want and then form an acid with \ce{X-}; this acid can then be neutralized by the base. Alternatively, we can bond a metal to the nucleophile and simply have the metal swap \ce{Nu-} for \ce{X-} with the palladium center, forming a metal salt as a byproduct.
            \item The last step is reductive elimination, which (re)generates our product and starting material.
        \end{itemize}
        \item Homocoupling (see Figure \ref{fig:mechanismCCb}):
        \begin{itemize}
            \item Once again, we first oxidatively add our aryl hydride.
            \item Then we transmetallate. Note that to regenerate our transmetallating nucleophile, we must treat it with some reductant (typically zinc metal); it can then oxidatively add \ce{Ar-X}.
            \item Finally, we reductively eliminate agin.
        \end{itemize}
        \item Mizoroki-Heck (see Figure \ref{fig:mechanismCCc}):
        \begin{itemize}
            \item As in the basic cycle, we must activate our palladium catalyst.
            \item Next, we oxidatively add \ce{Ar-X} once again.
            \item This time, however, we use an olefin as our electrophile. Note that $\ce{R}=\ce{H},\ce{CN}$, a ketone, etc. Additionally, note that we show in Figure \ref{fig:mechanismCCc} that we dissociate a ligand \ce{L} at this point, but we could also dissociate an \ce{X-} and carry that through. Overall, this step is a ligand substitution.
            \item A 1,2-migration follows. However, different types of migratory insertion can also occur as dictated by sterics.
            \item After this, we do a $\beta$-\ce{H} elimination to kick out our product (note that the product doesn't have to be \emph{trans} but it typically is) and bring us closer to regenerating the catalyst.
            \item Lastly, we can regenerate our initial catalyst with the reductive elimination of an acid (which will then be neutralized by an added base) as triggered by the addition of a ligand. Alternatively, the base can deprotonate the \ce{L_{n-2}XH} intermediate before pulling off the \ce{X-}.
            \item Note that this reaction must be carried out in basic media.
        \end{itemize}
    \end{itemize}
    \item Oxidative addition:
    \begin{itemize}
        \item Occurs from a coordinatively and electronically unsaturated metal center
        \item Thus, we usually use either a monoligated or diligated neutral palladium center.
        \item Large, bulky monodentate ligands (e.g., phosphines) are good.
        \begin{itemize}
            \item More steric hindrance enhances oxidative addition because we need easy dissociation later.
        \end{itemize}
        \item Influence of the electrophile on the speed of oxidative addition:
        \begin{equation*}
            \ce{ArI}
            > \ce{ArBr}
            > \ce{ArCl}
            > \ce{ArOTf}
            > \ce{ArOTs}
        \end{equation*}
        \begin{itemize}
            \item This shows that electron poor substrates are faster.
            \item Note that this makes sense because in an oxidative addition, the substrate is reduced, and more electron poor substrates will want to be reduced more.
        \end{itemize}
        \item \ce{Ni} is generally faster than \ce{Pd}.
    \end{itemize}
    \item Transmetallation:
    \begin{itemize}
        \item By far the most complicated step in cross coupling mechanisms. The least is known about it and it exhibits great variability. Thus, the following is just a couple notes; there's not necessarily anything definite that you should take away.
        \item General form:
        \begin{figure}[h!]
            \centering
            \schemestart
                \chemfig{L_\emph{n}MX}
                \+
                \chemfig{RMgX{(}THF{)_\emph{n}}}
                \arrow
                \chemleft{[}
                    \chemfig{L_\emph{n}M?-[,,,,dashed]X-[6,,,,dashed]Mg(-X)(-[6]solv)-[4,0.89,,,dashed]R?[,,dashed]}
                \chemright{]^\ddagger}
                \arrow
                \chemfig{L_\emph{n}MR}
                \+
                \chemfig{MgX_2{(}sol{)_\emph{n}}}
            \schemestop
            \caption{The general form of transmetallation.}
            \label{fig:transmetallationReaction}
        \end{figure}
        \item Some examples listed.
        \item On the subject of boronic acids, Negishi proposes the following: \ce{L2PhPdBr ->[KOH] [LPhPd($\mu${-}OH)]2 ->[Ar(BOH)2] Ar-Ph}, $70\%$.
        \begin{itemize}
            \item Negishi's conclusion: You typically need a base/nucleophile (such as hydroxide, water, or fluoride) to do the transfer of aryl boronic acids.
        \end{itemize}
        \item Stille (\ce{Sn}): Requires either a closed (such as in Figure \ref{fig:transmetallationReaction}) or open (as from a standard nucleophilic attack) transition state.
    \end{itemize}
    \item Reductive elimination:
    \begin{itemize}
        \item Steric pressure favors reductive elimination, yet it typically occurs from a less stable 3-coordinate intermediate.
        \item Such an intermediate can be generated after ligand dissociation.
        \item The two groups that reductively eliminate must be \emph{cis}.
        \item Aryl and vinyl groups are typically faster than alkyls.
    \end{itemize}
    \item Ligands:
    \begin{itemize}
        \item A huge number.
        \begin{itemize}
            \item These are typically phosphines, though.
        \end{itemize}
        \item Tri(\emph{tert}-butyl)phosphine is also known as tri(chicken foot)phosphine!
        \item Each ligand has very specific properties (better for one cross coupling than another).
        \item Very bulky phosphines tend to be good for these reactions, but chelating phosphines can be good, too.
        \item The reaction of a butyl Grignard in a cross coupling reaction gives us a couple of products (i.e., 1-Rbutane, 2-Rbutane).
        \begin{itemize}
            \item Yields are poor with phosphines as \ce{L} ligands.
            \item Yields are much better with dppf.
        \end{itemize}
        \item Dppf is ferrocene with a \ce{PPh2} off of each \ce{Cp} group. It's a chelating ligand (through the phosphines) with a very wide bite angle.
        \item Dppf is better because\dots
        \begin{itemize}
            \item With phosphines, during the reductive elimination step of the basic cycle (Figure \ref{fig:mechanismCCa}), we have interference from $\beta$-\ce{H} elimination (and sometimes ensuing migratory insertion of the eliminated hydride).
            \item $\beta$-\ce{H} elimination occurs with phosphine ligands because $\text{C.N.}=3$, so there are open axial coordination sites to which the hydride can migrate.
            \item However, with the bidentate dppf ligand, we essentially have a square-planar palladium species, which is very stable and will not easily form a 5-coordinate intermediate.
        \end{itemize}
        \item Note that because dppf yields a 4-coordinate species, reductive elimination will be a bit slower, but we'll happily sacrifice speed for the much greater yield.
        \begin{itemize}
            \item However, the large bite angle of dppf forces the aryl and R groups attached to the metal closer together, which promotes reductive elimination.
            \item Thus, the decrease in speed is not that significant.
        \end{itemize}
    \end{itemize}
    \item Alternative schemes for cross coupling: Heteroatoms.
    \begin{itemize}
        \item With nitrogen, we can do this with \ce{HNR2}, \ce{H2NR}, \ce{NH3}, imines, hydrosomes, etc.
        \item Heterocycles, azoles, carbonates, sulfoximines, amides, etc.
        \item Note that as \ce{N-} becomes less donating, reductive elimination slows down.
        \begin{itemize}
            \item There's not really a transmetallation step here.
            \item The issue is that these nucleophiles are not as basic as carbon-based ones.
        \end{itemize}
        \item We still need to worry about $\beta$-\ce{H} elimination.
    \end{itemize}
    \item Alternate mechanisms: Radicals.
    \begin{itemize}
        \item Happens more with first-row transition metals (they're better at 1-electron redox chemistry).
        \item Ullmann coupling (often with copper, but sometimes with nickel, too): \ce{2Ph-X ->[Cu][$>\SI{200}{\celsius}$] Ph2} Cu salts, diamines, diols.
        \begin{itemize}
            \item The temperature can be lowered with an appropriate ligand.
        \end{itemize}
        \item More flexibility in terms of the coupling partner.
        \begin{itemize}
            \item We can do \ce{C-O}, \ce{C-S}, and \ce{C-N} coupling in addition to \ce{C-C} coupling.
        \end{itemize}
        \item Possible mechanisms (note that $\ce{Y}=\ce{Ar},\ce{R}$):
        \begin{itemize}
            \item \ce{Ar-X + Cu^IY -> YCu^{III}XAr ->[{red. elim.}] Ar-Y + CuX}.
            \item \ce{Ar-X + Cu^IY -> Cu^{II}Y(XAr*) -> Cu^{II}XY + Ar* -> Ar-Y + CuX} (most likely).
            \item \ce{Ar-X + Cu^IY -> Cu^{II}Y(XAr*) -> Cu^{II}ArY- + X* -> Ar-Y + CuX}.
        \end{itemize}
        \item Ullmann couplings can be photoactivated (suggests a radical mechanism).
        \item \ce{Cu^{III}} is uncommon (suggests a radical mechanism).
        \item Radical mechanisms can enable $sp^3$-$sp^3$ couplings, even bulky ones (outer sphere radical attacks aren't as affected by steric bulk).
    \end{itemize}
\end{itemize}



\section{Lecture 14: Olefin Metathesis}
\begin{itemize}
    \item \marginnote{4/30:}Applies to both alkenes and alkynes.
    \begin{itemize}
        \item Metal carbene, alkylidene, and alkylidyne complexes are also important for their catalytic applications to olefin metathesis.
    \end{itemize}
    \item Nobel prize (2005) to Richard Schrock, Bob Grubbs, and Yves Chauvin.
    \item General form:
    \begin{figure}[h!]
        \centering
        \schemestart
            \chemfig{A-[1](-[3]A)=(-[1]A)(-[7]A)}
            \+{,,1.8em}
            \chemfig{B-[1](-[3]B)=(-[1]B)(-[7]B)}
            \arrow{<=>}
            \chemfig{2} \arrow{0}[,0.2] \chemfig{A-[1](-[3]A)=(-[1]B)(-[7]B)}
        \schemestop
        \caption{The general form of olefin metathesis.}
        \label{fig:olefinMetathesis}
    \end{figure}
    \begin{itemize}
        \item The olefins are not always tetrasubstituted, but we draw it as in Figure \ref{fig:olefinMetathesis} to illustrate the point.
        \item Alkyne metathesis is symmetric; just with disubstituted alkynes.
        \item Equilibrium reaction (we need a thermodynamic driving force to favor specific products).
        \begin{itemize}
            \item Can be limiting, but there are simple workarounds in the reaction types we are going to discuss.
        \end{itemize}
    \end{itemize}
    \item Not as widely used as cross coupling, but still important.
    \item Six basic classes of reactions:
    \begin{enumerate}
        \item Distribution.
        \begin{figure}[h!]
            \centering
            \schemestart
                \subscheme{
                    \chemfig{R_1-[:30]=[:-30]-[:30]R_2}
                    \arrow(.south--.north){0[*{180}+]}[-90,0.4]
                    \chemfig{R_3-[:30]=[:-30]-[:30]R_4}
                }
                \arrow{<=>}\subscheme{
                    \chemfig{R_1-[:30]=[:-30]-[:30]R_3}
                    \arrow(.south--.north){0[*{180}+]}[-90,0.4]
                    \chemfig{R_2-[:30]=[:-30]-[:30]R_4}
                    \arrow(.south--.north){0[*{180}+]}[-90,0.4]
                    \chemfig{R_1-[:30]=[:-30]-[:30]R_4}
                    \arrow(.south--.north){0[*{180}+]}[-90,0.4]
                    \chemfig{R_2-[:30]=[:-30]-[:30]R_3}
                }
            \schemestop
            \caption{Olefin metathesis classes: Distribution.}
            \label{fig:olefinMetathesis-distribution}
        \end{figure}
        \begin{itemize}
            \item Two reactants form all possible permutations of the carbene fragments.
            \item Not often useful, but\dots
            \item Useful applications:
            \begin{itemize}
                \item 2 propylenes make 2-butene and ethene, the latter of which is volatile and can be distilled off. Thus, distribution allows us to selectively make 2-butene from propylene, a widely available, mass-produced compound.
                \item Shell higher olefin process (SHOP): Used to selectively make higher-mass olefins and distill off lighter ones.
            \end{itemize}
        \end{itemize}
        \item Ring Opening Metathesis Polymerization (ROMP).
        \begin{figure}[H]
            \centering
            \schemestart
                \chemfig{[:30]*6(=-----)}
                \arrow{->}
                \chemleft{(}
                    \chemfig{=[,0.5]-[:120]-[:60]--[:-60]-[:-120]=[,0.5]}
                \chemright{)_n}
            \schemestop
            \caption{Olefin metathesis classes: ROMP.}
            \label{fig:olefinMetathesis-ROMP}
        \end{figure}
        \begin{itemize}
            \item A ring with one double bond opens and polymerizes at the double bond.
            \item Note that this reaction can proceed with rings containing any number of carbons.
            \item Driving force: Release of ring strain.
        \end{itemize}
        \item Addition Metathesis Polymerization (ADMET).
        \begin{figure}[h!]
            \centering
            \schemestart
                \chemfig{=[:30]-[:-30]-[:30]-[:-30]=[:30]}
                \arrow
                \chemleft{(}
                    \chemfig{=[:30,0.5]-[:-30]-[:30]-[:-30]=[:30,0.5]}
                \chemright{)_n}
                \arrow{0}[,0]\+{,,0.5em}
                \chemfig{[:45]=}
            \schemestop
            \caption{Olefin metathesis classes: ADMET.}
            \label{fig:olefinMetathesis-ADMET}
        \end{figure}
        \begin{itemize}
            \item A diolefin (under metathesis conditions) breaks off ethylene at both ends and polymerizes.
            \item Not as broadly useful as ROMP due to competitive ring-closing metathesis.
        \end{itemize}
        \item Ring-closing metathesis.
        \begin{figure}[h!]
            \centering
            \schemestart
                \chemfig{=[:-30]-[:30]-[:-30]X-[:30]-[:-30]=[:30]}
                \arrow{->}
                \chemfig{[:18]*5(=--X--)}
                \arrow{0}[,0]\+{,,0.6em}
                \chemfig{[:45]=}
            \schemestop
            \caption{Olefin metathesis classes: Ring-closing metathesis.}
            \label{fig:olefinMetathesis-ringClosing}
        \end{figure}
        \begin{itemize}
            \item A diolefin (under metathesis conditions) makes a cyclic heterocycle and ethylene.
            \item Driving force: Release of a gas (ethene).
            \item Has utility in natural product synthesis (as a last step to close macrocyclic rings).
        \end{itemize}
        \item Cross metathesis.
        \begin{figure}[h!]
            \centering
            \schemestart
                \chemfig{R-[:30]=[:-30]}
                \+{,,0.6em}
                \chemfig{R'-[:30]=[:-30]}
                \arrow{->}
                \chemfig{R-[:30]=[:-30]-[:30]R'}
                \+{,,0.6em}
                \chemfig{[:45]=}
            \schemestop
            \caption{Olefin metathesis classes: Cross metathesis.}
            \label{fig:olefinMetathesis-cross}
        \end{figure}
        \begin{itemize}
            \item Two terminal olefins break of their end carbons (which combine to form ethene) and then combine, themselves, as carbene fragments to form an internal olefin.
            \item An equilibrium process; must compete with homometathesis (forming \ce{R-R} and \ce{R$'$-R$'$}).
            \item Driving force: Release of a gas (ethene).
        \end{itemize}
        \item Enyne metathesis.
        \begin{figure}[H]
            \centering
            \schemestart
                \chemfig{R-~-R}
                \arrow{0}[,0]\+{,,0.4em}
                \chemfig{[:45]=}
                \arrow
                \chemfig{R-[2](=[:150])-[:30](=[:-30])-[2]R}
            \schemestop
            \caption{Olefin metathesis classes: Enyne metathesis.}
            \label{fig:olefinMetathesis-enyneMet}
        \end{figure}
        \begin{itemize}
            \item An alkyne and ethylene makes a diene.
        \end{itemize}
    \end{enumerate}
    \pagebreak
    \item Alkyne metathesis polymerization.
    \begin{figure}[H]
        \centering
        \schemestart
            \chemfig{~-**6(---(-~)---)}
            \arrow
            \chemleft{(}
                \chemfig{~[,0.5]-**6(---(-~[,0.5])---)}
            \chemright{)_n}
            \+
            \chemfig{H-~-H}
        \schemestop
        \caption{Alkyne metathesis polymerization.}
        \label{fig:alkyneMetathesisPolymerization}
    \end{figure}
    \begin{itemize}
        \item A dialkyne makes a polymer and acetylene.
        \item Very similar to ADMET.
    \end{itemize}
    \item Ring-closing alkyne metathesis exists as well.
    \item Catalysts:
    \begin{figure}[h!]
        \centering
        \begin{subfigure}[b]{0.25\linewidth}
            \centering
            \chemfig{Mo(~[2]N-[2]Ar)(=[:-30]-[:30]Bu^\emph{t})(<[:-110]RO)(>:[:-150]RO)}
            \caption{A Schrock catalyst.}
            \label{fig:olefinMetathesisCatalystsa}
        \end{subfigure}
        \begin{subfigure}[b]{0.2\linewidth}
            \centering
            \chemfig{Ru(=-[:-60]Ph)(>:[1]Cl)(-[2]PR_3)(<[5]Cl)(-[6]PR_3)}
            \caption{Grubbs I.}
            \label{fig:olefinMetathesisCatalystsb}
        \end{subfigure}
        \begin{subfigure}[b]{0.26\linewidth}
            \centering
            \chemfig{Ru(=-[:-60])(>:[1]Cl)(-[2]*5(-N(-Ar)---N(-Ar)-))(<[5]Cl)(-[6]PR_3)}
            \begin{tikzpicture}[remember picture,overlay]
                \draw [dashed] (-2.21,1.27) arc (-130:-50:0.5);
            \end{tikzpicture}
            \caption{Grubbs II.}
            \label{fig:olefinMetathesisCatalystsc}
        \end{subfigure}
        \begin{subfigure}[b]{0.27\linewidth}
            \centering
            \chemfig{Ru?(=-[:-60]**6(-(-[,1.13]O?-[:-120]R)--(-R)---))(>:[1]Cl)(-[2]L)(<[5]Cl)}
            \caption{Grubbs-Hoveyda.}
            \label{fig:olefinMetathesisCatalystsd}
        \end{subfigure}
        \caption{Olefin metathesis catalysts.}
        \label{fig:olefinMetathesisCatalysts}
    \end{figure}
    \begin{enumerate}
        \item Schrock catalysts (general form of Figure \ref{fig:olefinMetathesisCatalystsa}).
        \begin{itemize}
            \item Many of these exist.
            \begin{itemize}
                \item The metal center, as well as the amido and alkoxide ligands can be varied.
                \item Note that any X-type ligand can be substituted for the alkoxides.
            \end{itemize}
            \item Molybdenum or tungsten based (molybdenum is more active).
            \item Since molybdenum is an early transition metal, the catalyst is very electropositive. Thus, it's an alkylidene with strong nucleophilic character at the carbon.
            \item Extremely reactive. This is\dots
            \begin{itemize}
                \item Good for hard reactions.
                \item Bad because they're air sensitive and don't have a lot of functional group tolerance.
            \end{itemize}
        \end{itemize}
        \item Grubbs catalysts (Figures \ref{fig:olefinMetathesisCatalystsb}, \ref{fig:olefinMetathesisCatalystsc}, and \ref{fig:olefinMetathesisCatalystsd}).
        \begin{itemize}
            \item Ruthenium based.
            \item Less active, but more tolerant of air and water.
            \item Note that the top ligand on Grubs II (Figure \ref{fig:olefinMetathesisCatalystsc}) is called an NHC ligand, and that it is the same ligand that is typically present where "\ce{L}" is written on Grubbs-Hoveyda (Figure \ref{fig:olefinMetathesisCatalystsd}).
            \item Note also that the big bidentate ligand on the the Grubbs-Hoveyda catalyst (Figure \ref{fig:olefinMetathesisCatalystsd}) is a tethered alkylidene.
            \item There has probably been a lot more utility in these since they're initially less reactive.
        \end{itemize}
    \end{enumerate}
    \item General rule in inorganic chemistry: You can often make things more reactive (e.g., by heating them up), but rarely less (cooling things down can make the chemistry less reliable).
    \item Two mechanisms that were proposed early-on (related to Figure \ref{fig:mechanism-alkyneMetathesisb}):
    \begin{enumerate}
        \item Metal-diolefin adduct to $\eta^4$-cyclobutadiene adducct, then refragmentation.
        \item Metal-diolefin adduct reductively couples into a 5-membered metallacycle, then to a metallacycle butane alkylidene, then rearrangement and separation.
    \end{enumerate}
    \item Actual mechanism:
    \begin{figure}[h!]
        \centering
        \schemestart
            \chemfig{L_{\emph{n}}M=-[:60]R}
            \arrow{->[\small\chemfig{R'-[:30]=[:-30]}]}[,1.5]
            \chemleft{[}
                \chemfig{L_{\emph{n}}M?=(<[:-30,0.8]R)-[6,1.2,,,dashed](>:[:-20,0.7]R')=[4,1.23]?[,,dashed]}
            \chemright{]^\ddagger}
            \arrow
            \chemfig{[7]L_{\emph{n}}M*4(--(>:R')-(<[2]R)-)}
            \arrow{-U>[][*{0.-160}\small\subscheme{
                \chemfig{R-[:-60]=-[:60]R'}
                \arrow{0[*{0.135}or]}[-90,0.4]
                \chemfig{R-[:30]=[:-30]-[:30]R'}
            }][][][135]}[-90,2]
            \chemfig{L_{\emph{n}}M=CH_2}
            \arrow{->[\small\chemfig{R-[:30]=[:-30]}]}[180,4.5]
            \chemfig{[7]L_{\emph{n}}M*4(-(-R)---)}
            \arrow{-U>[][*{0.-20}\small\chemfig{[:45]=}][][][110]}[90,1.8]
        \schemestop
        \caption{Olefin metathesis mechanism.}
        \label{fig:mechanism-olefinMetathesis}
    \end{figure}
    \begin{itemize}
        \item Chauvin proposes this; Schrock and Grubbs prove it.
        \item Every step is an equilibrium reaction, but Figure \ref{fig:mechanism-olefinMetathesis} has monodirectional arrows for simplicity.
        \item On the transition state of the first step.
        \begin{itemize}
            \item A $2+2$ cycloaddition.
            \item Forbidden in organic chemistry, but allowed in inorganic because of the symmetry of the $d$ orbitals.
        \end{itemize}
        \item The first intermediate is a metallacyclobutane, which is a very important intermediate.
        \item The second intermediate is a terminal methylidene, which is highly reactive.
        \item Note that the mechanism drawn in Figure \ref{fig:mechanism-olefinMetathesis} is technically that of cross metathesis.
        \item There is frequently an activation step involving the loss of a ligand to free up a coordination site.
        \item There is also a degenerate pathway.
        \begin{itemize}
            \item This starts with the formation of a \emph{trans} metallacyclobutate in step 1. Thus, this pathway is actually favored sterically.
            \item This metallacyclobutane can then collapse back down to an olefin and a metal alkylidene, possibly with \ce{R$'$} on the metal alkylidene instead of \ce{R$'$}.
            \item This does indeed happen quite often, but it's clearly pretty harmless (also because this pathway is an equilibrium one, too).
            \item If we have a means of siphoning off ethylene, we can favor the primary pathway.
        \end{itemize}
    \end{itemize}
    \item Mechanistic probes (things that indicated the actual mechanism over the other proposed ones):
    \begin{itemize}
        \item Take a diphenyl compound with two styrenyl ligands (one with hydrogens; one with deuteriums).
        \item The products are a cyclized olefin and distribution products.
        \item A statistical mixture of \ce{CR2} fragments is observed; this is only possible with the mechanism in Figure \ref{fig:mechanism-olefinMetathesis}.
    \end{itemize}
    \item Decomposition:
    \begin{itemize}
        \item Where the catalysis can go wrong (and how to shut down these alternate pathways).
        \item Primarily caused by the electrophilic methylidenes.
        \item What happens is a nucleophylic attack by a phosphine on the electrophilic alkylidenes.
    \end{itemize}
    \item Notes on stereochemistry.
    \begin{itemize}
        \item Monoalkoxide pyridine (MAP) catalysts (recently reported by Schrock).
        \begin{itemize}
            \item Chiral at \ce{Mo}.
            \item Can resolve enantiomers.
            \item 100 times more active than bis-alkoxides (owing to the pyrilide ligand, which can coordinate face-on with a Cp and do some other odd things).
        \end{itemize}
        \item \emph{trans} metallacyclobutanes (which lead to \emph{trans} products) are preferred, but \emph{cis} metallacyclobutanes can be favored with sterics.
        \begin{itemize}
            \item For example, with respect to a Schrock catalyst (Figure \ref{fig:olefinMetathesisCatalystsa}), making the alkoxides large and the nitrido small can force both R groups above the plane of the metallacyclobutane.
        \end{itemize}
    \end{itemize}
    \item Alkyne metathesis.
    \begin{itemize}
        \item Again, largely developed by Schrock.
    \end{itemize}
    \item \ce{RC#Mo(OR)3} is the active catalyst.
    \begin{itemize}
        \item The method of generating this catalyst is shown.
    \end{itemize}
    \item Mechanisms:
    \begin{figure}[h!]
        \centering
        \begin{subfigure}[b]{0.9\linewidth}
            \centering
            \schemestart
                \chemfig{W?(>:[:30]Cl)(~[2]-[2]Bu^\emph{t})(>:[:150]O(-[2])-[:-160,0.9]-[6,0.8]-[:-20,0.9]O?[,5](-[6]))(-[6]Cl)(<[:-30]Cl)}
                \arrow{->[\small\chemfig{R-~-R}]}[,2.5]
                \chemfig{W(-[2]Cl)(-[4]Cl)(-[6]Cl)**4(-(-R)-(-R)-(-Bu^\emph{t})-)}
            \schemestop
            \caption{The likely mechanism.}
            \label{fig:mechanism-alkyneMetathesisa}
        \end{subfigure}\\[1em]
        \begin{subfigure}[b]{0.9\linewidth}
            \centering
            \schemestart
                \chemfig{\ce{L_n}M}
                \arrow{->[\small\chemfig{R-~-R}]}[,2.5]
                \chemfig{[:-36]\ce{L_n}M*5(-(-B)=(-B)-(-A)=(-A)-)}
                \arrow
                \chemfig{**4((-[:-120]B)-(-[:-60]B)-(-[:60]A)-(-[:120]A)-)-[1,0.7,,,opacity=0]-[4,1.5]\ce{L_n}M}
                \arrow
                \chemfig{[:-36]\ce{L_n}M*5(-(-A)=(-B)-(-B)=(-A)-)}
            \schemestop
            \caption{A possible alternate mechanism.}
            \label{fig:mechanism-alkyneMetathesisb}
        \end{subfigure}
        \caption{Alkyne metathesis mechanisms.}
        \label{fig:mechanism-alkyneMetathesis}
    \end{figure}
    \begin{itemize}
        \item By Schrock.
        \item In Figure \ref{fig:mechanism-alkyneMetathesisa}, a tungsten alkylidyne reacts with an alkyne to generate an isolable metallacyclobutadiene in a pathway quite similar to Figure \ref{fig:mechanism-olefinMetathesis}.
        \item However, the pathway in Figure \ref{fig:mechanism-alkyneMetathesisb} has also been observed, although not yet in a mechanistically, kinetically competent manner.
    \end{itemize}
    \item Note that alkyne metathesis polymerization is a really interesting area being worked on by Adam Veige.
    \item Enyne metathesis.
    \item General form (related to Figure \ref{fig:olefinMetathesis-enyneMet}):
    \begin{equation*}
        \ce{R-#-H + =-R ->[Ru] R-C(=Me)-C=-R}
    \end{equation*}
    \begin{itemize}
        \item This chemistry is once again dominated by ruthenium.
    \end{itemize}
    \item Mechanism.
    \begin{figure}[h!]
        \centering
        \schemestart
            \chemfig{\ce{L_n}Ru=-[:60]R}
            \arrow{->[\small\chemfig{~-R'}]}[,1.5]
            \chemfig{\ce{L_n}Ru(=-[:60]R)-[6]\phantom{i}-[,0.4,,,white]~[4]-[4]R'}
            \arrow
            \chemfig{*4((-R')=-(-R)-\ce{L_n}Ru-[,,2])}
            \arrow[-90,1.5]
            \chemfig{\ce{L_n}Ru=[6,,2](-[:-150]R')-[:-30]=[:30]-[:-30]R}
            \arrow{->[\small\chemfig{R-[:30]=[:-30]}]}[180,4]
            \chemfig{*4((-[:-150]R')(-[:-80,0.9]=[0]-[:-60]R)--(-R)-\ce{L_n}Ru-[,,2])}
            \arrow{-U>[][*{0.40}\small\chemfig{R'-[:30](=[2])-[:-30]=[:30]-[:-30]R}][][][135]}[90,1.5]
        \schemestop
        \caption{Enyne metathesis mechanism.}
        \label{fig:mechanism-enyneMet}
    \end{figure}
    \begin{itemize}
        \item The second step is a $2+2$ cycloaddition, and yields a metallacyclobutene.
        \begin{itemize}
            \item For the reaction to proceed, we must have a 1,3-regioisomer, as shown.
        \end{itemize}
        \item The third step sees the double bond electrons push up and to the right, and the electrons from the top bond in the ring push down and to the left.
        \item The final step occurs partially because of the instability of the tertiary carbon.
        \item The product is often \emph{trans} (as shown), but can be \emph{cis}.
    \end{itemize}
\end{itemize}




\end{document}