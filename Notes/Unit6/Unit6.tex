\documentclass[../notes.tex]{subfiles}

\pagestyle{main}
\renewcommand{\chaptermark}[1]{\markboth{\chaptername\ \thechapter\ (#1)}{}}
\setcounter{chapter}{5}

\begin{document}




\chapter{???}
\section{Lecture 20: Fischer-Tropsch}
\begin{itemize}
    \item \marginnote{5/19:}Industrial process mainly, but also has applications to renewable energy.
    \item Mechanism of the industrial catalysis in the first video; more on the molecular chemistry in the second video.
    \item General form:
    \begin{equation*}
        \ce{\{CO + H2\} -> {fuel (gasoline)}}
    \end{equation*}
    \begin{itemize}
        \item Clearly, the reactants are Syn gas.
    \end{itemize}
    \item Thermodynamics of \ce{CO + $x$ H2 -> {stuff}}.
    \begin{itemize}
        \item As $x$ increases, $\Delta H$ and $\Delta G$ decrease.
        \item Conclusion: The more \ce{C-H} bonds you make, the more thermodynamically favorable the process is.
        \begin{itemize}
            \item A lot of this comes from the thermodynamic reducing power of hydrogen.
        \end{itemize}
    \end{itemize}
    \item History:
    \begin{itemize}
        \item 1914: \ce{CO + 2H2 ->[Fe][silica supports] \frac{1}{n}C_nH_{2n} + H2O}.
        \begin{itemize}
            \item The ratio of the reactants can be tuned.
            \item Iron is a solid-state catalyst here. Note that other metals can be used, but iron was the most common.
            \item The products are a mixture of medium to long chain alkanes, alkenes, and methane.
            \item The challenge and limitation of this process is that it yields a mixture (a Schultz-Flory distribution) of products.
            \item This means that there is no real selectivity, although \ce{C7} is most common.
            \item This is also why chemists became interested in homogeneous catalysis for this area, because in principle it could give you better selectivity here.
        \end{itemize}
    \end{itemize}
    \item Industrially:
    \begin{equation*}
        \ce{{coal} ->[steam][reformed] CO + H2}
    \end{equation*}
    \begin{itemize}
        \item This is commonly done in South Africa because they have a lot of coal.
        \begin{itemize}
            \item \ce{CH4} can also be used as a reactant.
        \end{itemize}
        \item The ratio of the product gasses can be tuned by the water-gas shift reaction (see Lecture 17), which is \ce{CO + H2O <=> H2 + CO2} (in the forward direction).
        \begin{itemize}
            \item \ce{CO2} can then be removed.
            \item One potentially interesting exciting direction that's not been industrially accomplished due to cost is the reverse water-gas shift reaction, which is just the reverse direction.
            \item The reverse water-gas shift reaction is interesting because if you could convert \ce{CO2} into \ce{CO + H2O} and remove the water, you could feed the \ce{CO} back into the original process to make gasoline.
        \end{itemize}
        \item Challenges:
        \begin{itemize}
            \item Selectivity in the Fischer-Tropsch process.
            \item Source of \ce{H2} (right now we get it from fossil fuels). There is research into how to reduce \ce{CO2} into fuels, but if we can get \ce{H2} from somewhere other than fossil fuels, it's a solved problem.
        \end{itemize}
    \end{itemize}
    \item Conditions (for the following reaction):
    \begin{equation*}
        \ce{\{CO + H2\} -> C_nH_{2n} + $n$H2O}
    \end{equation*}
    \begin{itemize}
        \item High temperatures (150-$\SI{300}{\celsius}$).
        \begin{itemize}
            \item Favor faster rates and better conversion yield percentages, but more methane (the least valuable product).
        \end{itemize}
        \item High pressure (1-$\SI{30}{atm}$).
        \begin{itemize}
            \item Favors longer chain alkanes and better conversion since methane is a gas (lower likelihood of eliminating it).
        \end{itemize}
        \item Optimal \ce{CO} / \ce{H2} concentration.
        \begin{itemize}
            \item Cobalt catalysts use $1:2$.
            \item Iron catalysts should have relatively more \ce{CO}.
        \end{itemize}
    \end{itemize}
    \item Product (Schultz-Flory) distribution:
    \begin{equation*}
        \frac{W_n}{n} = (1-\alpha)^2\alpha^{n-1}
    \end{equation*}
    \begin{itemize}
        \item $\frac{W_n}{n}$ is the average molecular weight?
        \item $W_n$ is the weight fraction of hydrocarbons containing $n$ carbon atoms and $\alpha$ is the chain growth probability (controlled by the catalyst and the conditions).
        \item If $\alpha<0.5$, then methane dominates.
        \item However, as $\alpha\to 1$, the methane fraction decreases relative to the sum of heavy products.
        \item Sasol (a South African company) mediates this process.
        \item We can use the products for a variety of different things (heptane for fuel, heavier ones for wax [you can buy Sasol wax], etc.).
        \item If we were going to do this globally for fuel, we'd have a lot of wax as a byproduct, which we neither have a place to store nor can afford to do since it's so wasteful. Thus, there is great interest in selectivity for \ce{C6}, \ce{C7}, \ce{C8}, or even butene, from which we could do olefin metathesis to make hexanes.
        \item Two necessary things to use this process for renewable energy:
        \begin{enumerate}
            \item Find a good source of hydrogen so that we could run the reverse water-gas shift reaction.
            \begin{itemize}
                \item The reaction might need to be optimized a bit as well, but in principle that can be done; hydrogen is the real problem.
            \end{itemize}
            \item Improve the selectivity of the Fischer-Tropsch chemistry.
        \end{enumerate}
    \end{itemize}
    \item As with the Haber-Bosch process, this is solid support chemistry.
    \item Mechanism (initial proposal):
    \begin{itemize}
        \item Solid-state iron binds hydrides and \ce{CO} on its surface, leading the formation of a formyl species before adding another hydrogen to go to a bound alcohol species, losing \ce{H2O} to go to a bound carbene (the carbene is important because it facilitates chain growth), and then proceeding to a bound methyl species. With enough methyl species, you could do other things.
        \item Gerhardt Ertl proves this wrong.
    \end{itemize}
    \item Mechanism (actual):
    \begin{itemize}
        \item Solid-state iron binds hydrides and carbonyls on its surface. Next, we split \ce{CO} to make bound carbides and bound oxides. With more \ce{H2}, we can mix and scramble the carbides, oxides, and hydrides, kicking out \ce{H2O} and making bound carbene species, which can then go on and form the product that we want as described in the initial proposal.
        \item Proven with in situ studies performed on the surface of these materials.
    \end{itemize}
    \item Another key contribution to the proof came from molecular chemistry, which showed that formyl species are generally unstable:
    \begin{figure}[h!]
        \centering
        \begin{subfigure}[b]{\linewidth}
            \centering
            \schemestart
                \chemfig{\ce{L_n}M(-[1]CO)(-[7]H)}
                \arrow{<<->}
                \chemfig{\ce{L_n}M-[:30](=[2]O)-[:-30]H}
                \+{,,1.8em}
                \chemfig{\ce{L_n}@{M}M-[:30](=[2]@{O}\charge{180=\:}{O})-[:-30]H}
            \schemestop
            \chemmove{\draw [blx,semithick,shorten >=1pt,shorten <=4pt] (O) to[out=180,in=90] (M);}
            \caption{Formyl species.}
            \label{fig:formylInstabilitya}
        \end{subfigure}\\[1em]
        \begin{subfigure}[b]{\linewidth}
            \centering
            \schemestart
                \chemfig{\ce{L_n}M(-[1]CO)(-[7]CH_3)}
                \arrow{<->>}
                \chemfig{\ce{L_n}M-[:30](=[2]O)-[:-30]}
            \schemestop
            \caption{Acyl species.}
            \label{fig:formylInstabilityb}
        \end{subfigure}
        \caption{Stability of metal-carbonyl derivatives.}
        \label{fig:formylInstability}
    \end{figure}
    \begin{itemize}
        \item In Figure \ref{fig:formylInstabilitya}, the reactant is generally favored because it has two bonds instead of one, and metal-hydride bonds are highly thermodynamically favored over metal alkyl bonds.
        \item In Figure \ref{fig:formylInstabilityb}, the product is generally favored.
        \item Think about the equilibrium in terms of the nucleophilicity of the X group. Hydrides are not nucleophilic, whereas alkyl species are.
    \end{itemize}
    \item The synthesis of metal formyl species was pioneered by Jim Collman in 1973.
    \begin{equation*}
        \ce{Fe(CO)5 ->[Na][-HOC-O-COMe] Fe(CO)4^2- ->[][-OAc^-] Fe(CO)4(COH)- ->[][-CO] HFe(CO)4-}
    \end{equation*}
    \begin{itemize}
        \item The first step is ill defined, and the products of it are a messy mixture of carbon-containing products.
        \item The second intermediate (the formyl species) is unstable and will decompose over time to form the final product.
    \end{itemize}
    \item Chuck Casey and John Gladys found stable formyl species (the product of the following reaction):
    \begin{equation*}
        \ce{CpRe(CO)(NO)(L) ->[HBR3^-] CpRe(NO)(L)(COH)}
    \end{equation*}
    \begin{itemize}
        \item The reactant is a chiral, $18\,\e[-]$, $d^6$ species.
    \end{itemize}
    \item John Bercaw did more with metal-formyl species:
    \begin{equation*}
        \ce{Cp^*2ZrH2 + O#C-ML_n ->[][fast] L_nM=CH-O-ZrCp^*2H}
    \end{equation*}
    \begin{itemize}
        \item The first species is an extremely hydridic metal hydride that has very similar properties to \ce{LiAlH}. It is a $16\,\e[-]$, $d^0$ species.
        \item This chemistry is driven by the strength of the zirconium bond ($\approx\SI{130}{kcal\per\mole}$).
        \item Note that although this is technically a formyl species, it's in practice more of a Fischer carbene structure.
        \begin{itemize}
            \item The bonding is very different and the Lewis acid activation of that strong \ce{Zr^{IV}} Lewis acid makes this more of a carbene structure.
            \item This is useful in \ce{C-C} coupling reactions.
        \end{itemize}
        \item One example of \ce{ML_n} is \ce{Cp2W}.
        \item Another one is \ce{Cp2Nb(H)(CO) + Cp^*2ZrH2 -> Cp2Nb(H)(CH-O-ZrCp^*2H) <=>[fast]}\\ \ce{Cp2Nb(H)(CH2-O-ZrCp^*2) ->[H2] Cp2NbH3 + Cp^*2Zr(H)(OCH3) ->[][no rxn]}.
        \begin{itemize}
            \item \ce{Nb} is niobium.
            \item A major problem is once we form that extremely strong metal-oxygen bond, it will not break, so it's very hard to get \ce{H2O} off of these species.
        \end{itemize}
    \end{itemize}
    \item How we think about the molecular systems that form \ce{C-C} bonds.
    \begin{itemize}
        \item This is ultimately what happens on solid support surfaces in industrial catalysis.
        \item The challenge is that those industrial systems don't have selectivity; molecular systems could in principle provide this (this is an area of active research).
    \end{itemize}
    \item To recap, we just talked about the potential of early metal hydrides to activate \ce{CO} (Bercaw's niobium system took it all the way to a methanol equivalent).
    \item Bercaw (a titan of organometallic chemistry):
    \begin{figure}[h!]
        \centering
        \schemestart
            \chemfig{\ce{Cp^*2}Zr(-[1]Me)(-[7]Me)}
            \arrow{->[\small\ce{CO}]}
            \chemleft{[}
                \subscheme{
                    \chemfig{\ce{Cp^*2}@{Zr1}Zr(-[1]Me)(-[7]C(-[6])=[1]@{O1}\charge{[extra sep=2.5pt]180=\:}{O})}
                    \arrow{<->}
                    \chemfig{\ce{Cp^*2}@{Zr2}Zr(-[1]Me)(-[7]O-[5]@{C2}\charge{[extra sep=2.5pt]90=\:}{C}-[6])}
                }
            \chemright{]}
            \arrow{->[\small\ce{CO}][
                \footnotesize
                \chemleft{[}
                    \chemfig{\ce{Cp^*2}@{Zr3}Zr*5([:-36]-O-@{C3b}\charge{150=\:}{C}(-)-[,,,,white]@{C3a}\charge{-150=\:}{C}(-)-O-)}
                \chemright{]^\ddagger}
            ]}[,2.3]
            \chemfig{\ce{Cp^*2}Zr*5([:-36]-O-(-)=(-)-O-)}
        \schemestop
        \chemmove{
            \draw [blx,semithick,shorten <=5pt,shorten >=2pt] (O1)  to[bend right=20]  (Zr1);
            \draw [blx,semithick,shorten <=5pt,shorten >=2pt] (C2)  to[bend left=20]   (Zr2);
            \draw [blx,semithick,shorten <=5pt,shorten >=2pt] (C3a) to[out=-150,in=15] (Zr3);
            \draw [blx,semithick,shorten <=5pt,shorten >=2pt] (C3b) to[out=150,in=-15] (Zr3);
            \draw [blx,shorten <=2pt,shorten >=2pt,stealth-stealth] (C3a) to[bend left=20] (C3b);
        }
        \caption{Bercaw's first \ce{C-C} coupling.}
        \label{fig:CCcoupling-Bercaw}
    \end{figure}
    \begin{itemize}
        \item The first ever \ce{C-C} coupling that was observed in a well-defined and pretty clean way from \ce{CO}.
        \begin{itemize}
            \item Using methyl groups makes it imperfect.
            \item The mechanism isn't entirely clear.
        \end{itemize}
        \item The second step proceeds from the second resonance structure.
        \item Additionally, Bercaw studied reactions with \ce{Cp^*2ZrH2CO}, a compound with virtually no $\pi$ backbonding:
        \begin{itemize}
            \item \ce{Cp^*2ZrH2CO ->[H2] Cp^*2Zr(H)(OCH3)}, where \ce{-OCH3} is a methanol equivalent.
            \item \ce{Cp^*2ZrH2CO ->[CO] Cp^*2Zr(H)-O-=-O-Zr(H)(Cp^*2)}. Here, we actually see \ce{C-C} coupling.
            \item \ce{Cp^*2ZrH2CO ->[CO, H2] Cp^*2Zr(H)-O---O-Zr(H)(Cp^*2)}. This is even closer to ethylene glycol, but the \ce{Zr-O} bonds are still an issue.
            \item \ce{Cp^*2ZrH2 + Cp^*2Zr(CO)2 -> Cp^*2Zr(H)-O-=-O-Zr(H)(Cp^*2)}
        \end{itemize}
    \end{itemize}
    \item Similarly, Tobin Marks looked at \ce{Cp^*2Th(OR)(H)} (thorium hydride alkoxides), where $\ce{R}=\ce{CH($t${-}Bu)2}$ is a pretty bulky super-isopropal ligand abbreviated \ce{$i${-}Pr^*}:
    \begin{figure}[h!]
        \centering
        \schemestart
            \chemfig{\ce{Cp^*2}Th(-[1]H)(-[7]OR)}
            \arrow{->[\small\ce{CO}]}
            \chemfig{\ce{Cp^*2}Th(-[2,,2]CO)(-H)(-[6,,2]OR)}
            \arrow
            \chemfig{\ce{Cp^*2}@{Th}Th(-[7]OR)(-[:30](=[2]@{O}\charge{180=\:}{O})-[:-30]H)}
            \arrow{->[][\SI{25}{\celsius}]}
            \chemfig{Th(-[7]OR)-[1]O-[1]=_-[7]O-[7]Th(-[5]RO)}
        \schemestop
        \chemmove{\draw [blx,semithick,shorten <=4pt,shorten >=2pt] (O) to[out=180,in=90] (Th);}
        \caption{Marks' \ce{C-C} coupling.}
        \label{fig:CCcoupling-Marks}
    \end{figure}
    \begin{itemize}
        \item The theme with early metals: If you can get to the Fischer carbene type structure, you can get coupling.
    \end{itemize}
    \item Pete Wolczanski worked with tantalum silox complexes.
    \begin{itemize}
        \item \ce{Ta(OSi($t${-}Bu)3)3 + 2CO -> 2({silox})3TaO + ({silox})3Ta=C=C=Ta({silox})3}.
        \item This is \ce{C-C} coupling \emph{and} complete deoxygenation.
    \end{itemize}
    \item Challenge with all of this chemistry: early-metal oxygen bonds are an extreme thermodynamic sink.
    \begin{itemize}
        \item Unfortunately, the transition from metal-carbon bonds to metal-oxygen bonds is also the driving force of this reactivity.
        \item We can get around this issue by trapping the oxygen with other electrophiles.
    \end{itemize}
    \item Kit Cummins: \ce{MoL3 ->[CO] L3MoCO ->[1Na] Na[L3MoCO] ->[Bu^t-COCl][-NaCl] L3Mo#C-O-COBu^t ->[\Delta][-CO2,H2C=CMe2] L3Mo#C-H ->[KBn] [L3Mo#C]-}.
    \begin{itemize}
        \item $\ce{L}=\ce{N($t${-}Bu)({xylyl})}$.
        \item The third step is more explained in organic chemistry, where an electrophile like the one added (pivaloyl chloride) should create an ester speies and lose \ce{NaCl}.
        \item Note that \ce{KBn} is benzyl potassium.
        \item There are very few examples of terminal carbides, but note that this is similar to what happens on the surface of the solid state iron catalyst.
        \item This isn't perfect because we're releasing \ce{CO2}, but it is a step where the oxygen isn't trapped on the metal center.
    \end{itemize}
    \item Theodor Agapie (with his graduate student Joshua Buss):
    \begin{figure}[h!]
        \centering
        \schemestart
            \chemfig{Mo?[Mo](-[,1.3]PR_2-[:-60,,1]**6(-(-**6(-?[Mo,,dashed]-?[Mo,,dashed]-?[L]---))-----))(-[4,1.3]R_2P-[:-120,,2]?[close]-[:-60]?[L]-[::-60]-[::-60]-[::-60]-[::-60]?[close]**6([::-120]))(>:[:65]CO)(<[:115]OC)}
            \arrow{0}[,0.1]\+
            $4\,\e[-]$
            \+
            $4\,\text{E}^+$
            \arrow
            \chemfig{LMo{(}N_2{)}}
            \+
            \chemfig{E_2O}
            \+\arrow{0}[,0.1]
            \chemfig{E-[:60]O-C~C-E}
        \schemestop
        \caption{Agapie's \ce{C-C} coupling.}
        \label{fig:CCcoupling-Agapie}
    \end{figure}
    \begin{itemize}
        \item \ce{KC8} yields the $4\,\e[-]$.
        \item The electrophile $\text{E}^+$ is something like TMSCl, a silane, or another silyl electrophile.
        \item This also proceeds through carbides.
    \end{itemize}
    \item Jonas Peters (on iron): \ce{Fe#C-O + TMSCl -> Fe#C-OTMS}.
    \begin{itemize}
        \item We are trapping the oxygen with an oxygen-silion bond, which thermodynamially are quite strong.
        \item Problem: While a silyl electrophile is substantially better than trapping the metal center (which entirely precludes catalysis), the silyl electrophiles generate stoichiometric waste. We need to trap the oxygen with water, not a silyl ether.
        \item Thus, what you ultimately want to do is use later metals (like iron), but use hydrogen instead.
    \end{itemize}
    \item Dan Suess (graduate student of Jonas Peters):
    \begin{figure}[h!]
        \centering
        \chemfig{M?[M](-[2,1.7]P(-Pr^i)(-[2]Pr^i)-[4,1.7]**6(-----(-[6]B?[M,,dashed]?[B]>:[1]?[M,,dashed]**6([:45,0.5]------))-))(-[6,1.7]P(-Pr^i)(-[6]Pr^i)-[4,1.7]**6(-?[B]-----))}
        \caption{The DPB ligand.}
        \label{fig:DPB}
    \end{figure}
    \begin{itemize}
        \item Investigates various reactions of DPB-bound irons.
        \item \ce{DPBFe-N#N-FeDPBB ->[4CO] DPBFe(CO)2 ->[XS CO] FeP2B(CO)3}.
        \begin{itemize}
            \item The product here is formally \ce{Fe^0}, but actually slightly higher than that.
        \end{itemize}
        \item \ce{DPBFe-N#N-FeDPBB ->[4CO] DPBFe(CO)2 ->[H2] FeP2($\kappa^2${-}BH)H(CO)2}.
        \item \ce{DPBFe-N#N-FeDPBB ->[4CO] DPBFe(CO)2 ->[1K^0] [FeP2($\kappa^2${-}PhB)(CO)2]- ->[XS K^0] [FeP2B(CO)2]^2- ->[2TMSOTf]} a very weird product.
        \begin{itemize}
            \item The third intermediate is highly activated and somewhat analogous to \ce{Fe(CO)5^2-}.
            \item The product has a \ce{-C-OTMS} substituent, as well as a tridentate derivative of the DPB ligand that still bonds through the two phosphines, but instead of partially bonding through the boron and the ipso-carbon of the phenyl ring, the boron binds to an additional \ce{C-OTMS} ligand through the carbon, which in turn binds back to the iron center.
            \item This product has many resonance structures; one with particular merit is \ce{Fe(#C-OTMS)2} with one of the triple bonds bonding datively to the boron, although it implies an \ce{Fe^{VI}} oxidation state that is likely an overestimation, even though the iron is certainly high-valent here.
            \item Note that if the product is treated with hydrogen, we selectively create (Z)-\ce{TMSO-=-OTMS}.
            \item This is not catalytic, but it is one of the few examples where we don't generate an intractable metal-oxide, and we do generate a \ce{C-C} coupled product.
        \end{itemize}
    \end{itemize}
    \item Conclusion: There's a lot of interesting organometallic chemistry surrounding \ce{C-O} activation and getting that prepped for functionalization, specifically via \ce{C-C} bond formation.
    \item This is not a solved problems.
    \begin{itemize}
        \item While molecular chemistry offers the promise of selectivity, there are real challenges with activating it and getting \ce{H2O} out instead of trapping the oxygen with silyl electrophiles.
        \item Part of the issue is the stability of the molecular complexes (they're not as stable as a solid iron surface) and you need a lot of energy to split \ce{C#O} bonds. On the other hand, if you apply that energy thermally, it will hurt your selectivity.
    \end{itemize}
    \item There's some extra molecular chemistry in the notes.
\end{itemize}



\section{Lecture 21: \ce{C-H} Activation}
\begin{itemize}
    \item \marginnote{5/21:}A really active area within the last ten years, and a really powerful synthetic technique for functionalizing molecules.
    \begin{itemize}
        \item The major problem with selectivity is not yet solved though.
    \end{itemize}
    \item General form:
    \begin{equation*}
        \ce{R-H + A ->[{cat}] R-FG + B}
    \end{equation*}
    \begin{itemize}
        \item \ce{A} is a functionalizing reagent, \ce{FG} is a functional group, and \ce{B} is a byproduct that usually accepts the proton.
    \end{itemize}
    \item Examples from organic chemistry:
    \begin{itemize}
        \item \ce{C4H10 ->[Br2] C4H9Br}.
        \item \ce{C4H10 ->[O2][\Delta] CO2 + H2O}.
    \end{itemize}
    \item The real challenge from an organometallic perspective is selectivity:
    \begin{enumerate}
        \item Internal \ce{C-H} bonds are typically more reactive.
        \item The product is typically more reactive than the starting material.
        \begin{itemize}
            \item We need to prevent over-oxidation.
        \end{itemize}
    \end{enumerate}
    \item Another challenge is the thermodynamics of this process.
    \item For example:
    \begin{itemize}
        \item \ce{C6H6 + CO -> C6H5(COH)} has $\Delta H=\SI{1.7}{kcal\per\mole}$.
        \item \ce{R-H + H-X -> R-X + H2} has $\Delta H= \SI{22}{kcal\per\mole}$ when $\ce{X}=\ce{OH}$.
        \item \ce{R-- -> R-= + H2} has $\Delta H=\SI{30}{kcal\per\mole}$.
        \item You will get some entropic favorability, but you need to put in a lot of driving force --- and whenever you do this, selectivity and over-oxidation become problems.
    \end{itemize}
    \item Classic studies (Shilov's catalyst):
    \begin{itemize}
        \item \ce{K2PtCl6 + CH3CO2D + D2O} can be mixed with alkanes to incorporate deuterium into said alkanes.
        \begin{itemize}
            \item Note that \ce{K2PtCl6} is an oxidant.
        \end{itemize}
        \item Data listed.
        \item Conclusion: Higher degrees of substitution at less substituted carbons.
        \item Basically, you can do this: \ce{R-H ->[K2PtCl4 (12\%)][DClO4, CH3CO2D, D2O] R-D}.
        \item Also: \ce{C5H12 + H2PtCl6 ->[NaPtCl4][H2O] C5H11Cl} on both the 1- and 2-carbons.
    \end{itemize}
    \item Mechanism (monooxidation to form alcohols [very useful]):
    \begin{equation*}
        \ce{[PtCl4]^2- <=>[][{aqueous}] ({sol})2PtCl2 ->[MeH][-HCl] ({sol})2PtCl(CH3) ->[Pt^{IV}Cl2] ({sol})2PtCl3(CH3) ->[H2O][-CH3OH, HCl] ({sol})2PtCl2}
    \end{equation*}
    \begin{itemize}
        \item The fact that this process activates methane (\ce{MeH}) is very good because we have a lot of it and the only thing it's good for is burning.
        \begin{itemize}
            \item It's not economically viable to capture all the extra methane on oil fields and transport it (instead, they flair it off because that's better for the environment than just releasing it).
            \item However, if they could functionalize it immediately and turn it into methanol liquid, which can be transported, that would be a huge boon to industry and the environment.
        \end{itemize}
        \item In the last step, water attacks the methyl group on the platinum catalyst, releasing methanol and then reductively eliminating \ce{HCl}.
        \item However, the challenge with this is the second step (with \ce{Pt^{IV}Cl})
        \begin{itemize}
            \item For whatever reason, the only oxidant that kinetically works is \ce{Pt^{VI}}. But we need a lower oxidation state oxidant to solve this problem. Yet we haven't despite decades of research.
        \end{itemize}
    \end{itemize}
    \item The closest we've come to a solution is work by Periana: \ce{CH4 + H2SO4 ->[{cat}] CH3SO4H + 2H2O + SO2}.
    \begin{itemize}
        \item \ce{CH3SO4H} is methyl sulfuric acid.
        \item The catalyst is a dichloride platinum species with a bidentate chelating dipyrmidine species.
        \item $\text{TOF}=\SI{e-2}{\per\second}$.
        \item $\text{TON}\approx 7500$.
        \item The TOF and TON are not viable for a large scale industrial process with platinum, but are pretty good.
    \end{itemize}
    \item Thus, in principle, we could do this industrially with the following.
    \begin{figure}[h!]
        \centering
        \schemestart
            \chemfig{CH_4}
            \+\arrow(--SA1){0}[,0.1]
            \chemfig{H_2SO_4}
            \arrow{0}[,0.1]\+\arrow(--SO1){0}[,0.1]
            \chemfig{SO_3}
            \arrow(--MSA)
            \chemfig{CH_3SO_4H}
            \arrow{0}[,0.1]\+
            \chemfig{H_2O}
            \+\arrow(--SO){0}[,0.1]
            \chemfig{SO_2}
            \arrow(@MSA--){->[*{0}\small\ce{H2O}]}[-90]
            \subscheme{
                \arrow(--SA2){0}[,0.1]
                \chemfig{H_2SO_4}
                \arrow{0}[,0.1]\+
                \chemfig{CH_3OH}
            }
            \arrow(@SA2--@SA1)
            \arrow(@SO--SO2){->[*{0}\small\ce{\frac{1}{2}O2}]}[90]
            \chemfig{SO_3}
            \arrow(@SO2--@SO1)
        \schemestop
        \caption{An industrial version of Shilov's monooxidation to form alcohols.}
        \label{fig:ShilovAlcoholFormation}
    \end{figure}
    \begin{itemize}
        \item This is pretty powerful, but there are problems:
        \item \ce{H2SO4} is corrosive, which means that this process is hard to scale up.
        \item Separation of \ce{CH3OH} from \ce{H2SO4} is also hard (adds cost).
        \item The platinum catalyst is also expensive, so you would need a much higher turnover number to make it viable.
        \begin{itemize}
            \item This is more of a solvable issue; the others, not so much.
        \end{itemize}
    \end{itemize}
    \item Mechanism:
    \begin{itemize}
        \item \ce{N2Pt(OS)2 <=> [N2PtOS]+ ->[CH4, OS][-HOS] N2Pt(OS)(CH3) ->[SO3, 2HOS][SO2, H2O] [N2Pt(OS)2(CH3)]OS ->[][-CH3OS] [N2PtOS]+}.
        \begin{itemize}
            \item Sulfate and related species are abbreviated to \ce{OS}.
        \end{itemize}
        \item This is the closest we've gotten to a catalytic process for converting methane into methanol so far.
        \item Mechanistic notes:
        \begin{enumerate}
            \item $\sigma$ adducts.
            \begin{itemize}
                \item A key step is binding methane, but methane is a terrible ligand.
                \item \ce{M(H|CH3) -> M(H)(CH3)}.
                \item There exist other platinum systems where we can study this process in some detail, such as: \ce{($\kappa^2${-}Ph2B(CH2PR2)2)Pt(THF)(CH3) ->[C6H6, \SI{50}{\celsius}][-CH4] Pt(THF)(C6H5)}.
                \begin{itemize}
                    \item This presumably proceeds through benzene as a $\sigma$ adduct.
                \end{itemize}
            \end{itemize}
            \item Oxidation.
            \begin{itemize}
                \item In the Shilov system, if you use labels, quenching \ce{{}^195PtCl6^2-}, you get no \ce{{}^195PtMe} complexes.
                \item Indeed, it appears that \ce{PtCl6^2-} is acting purely as a \ce{Cl2} source.
            \end{itemize}
            \item Reductive elimination.
            \begin{itemize}
                \item An external nucleophilic attack, as generally supported by mechanistic studies.
                \item \ce{($\kappa^2${-}PR2CH2CH2PR2)Pt(CH3)3OR <=> [($\kappa^2${-}PR2CH2CH2PR2)Pt(CH3)3]OR ->[OR$'$]}\\ \ce{($\kappa^2${-}PR2CH2CH2PR2)Pt(CH3)2 + CH3OR + CH3OR$'$}.
                \item \ce{[PtCl5(CH3)]^2- ->[][-Cl^-] [PtCl4Me]^-Cl^- ->[H2O] PtCl4^2- + MeOH + H+}.
                \item \ce{[PtCl5(CH3)]^2- ->[][-Cl^-] [PtCl4Me]^-Cl^- ->[Cl] PtCl4^2- + CH3Cl}.
                \item Labeling studies also support the inversion of stereochemistry.
            \end{itemize}
        \end{enumerate}
    \end{itemize}
    \item While methane activation may not yet be feasible industrially, it is in the pharmaceutical industry.
    \item Directed \ce{C-H} activation.
    \item Examples:
    \begin{itemize}
        \item Early: \ce{($i${-}Pr)(H2N)(H)C-COOH ->[{cat} / {ox}] H2 +} the cyclized product.
        \begin{itemize}
            \item Catalyst is \ce{K2PtCl4}.
            \item Using \ce{PtCl6^2-} ($10\%$) as an oxidant gives a $21\%$ yield.
            \item Using \ce{CoCl2} ($10\%$) as an oxidant and a lower catalyst loading gives a $67\%$ yield.
            \item Using \ce{CoCl2} ($5\%$) as an oxidant and a dramatically lower catalyst loading gives a $20\%$ yield.
            \item Because this is giving a higher yield than catalytically, something else is probably going on.
        \end{itemize}
        \item Palladium has started to dominate over platinum.
        \item Sanford: \ce{(Me)(CH3N)C-CMe3 ->[Pd(OAc)2] [(Me)(CH3N)C-CMe2CH2Pd($\mu${-}OAc)]2 ->[PhI(OAc)2]}\\ \ce{(Me)(CH3N)C-C(CH2X)(CH2Y)(MeOAc)} where \ce{X} and \ce{Y} are functional groups.
        \begin{itemize}
            \item This gives a picture of where palladium can directly activate a \ce{C-H} bond.
        \end{itemize}
        \item Further examples from Jin-Quan Yu and Guangbin Dong (UChicago).
        \item The key in all of these examples is a directing nitrogen, to which the palladium catalyst bonds, bringing it close to the hydrogen that we want to activate.
        \begin{figure}[h!]
            \centering
            \schemestart
                \chemfig{(-[:30](-[2]H)(-[:-30]R))(-[6]R)(-[:150](-[2]NH_2)(>:[:-150])(<[:-110]))}
                \arrow{->[\small\ce{PdX2, ArX}][\footnotesize\chemfig[atom sep=1.3em]{**6(--**6(----N-)--(-=_[:30]O)--)}]}[,1.5]
                \chemfig{(-[:30](-[2]H)(-[:-30]R))(-[6]R)(-[:150](-[2]N*6(-Pd-N**6(----**6(----)-)---=))(>:[:-150])(<[:-110]))}
                \arrow
                \chemfig{(-[:30](-[2]Ar)(-[:-30]R))(-[6]R)(-[:150](-[2]NH_2)(>:[:-150])(<[:-110]))}
            \schemestop
            \caption{Dong's directed \ce{C-H} activation.}
            \label{fig:CHactivation-Dong}
        \end{figure}
        \begin{itemize}
            \item This is nicely demonstrated by Guangbin Dong's example, as seen in Figure \ref{fig:CHactivation-Dong}.
            \item By placing the palladium and hydrogen close together, we can facilitate the oxidative addition even if it is unfavorable or has a high barrier. The chelate effect really helps here.
            \item One of the key features of \ce{C-H} activation is \ce{C-H}'s are terrible ligands, so tricks like this are essential to get a high effective concentration to drive the forward reaction.
        \end{itemize}
    \end{itemize}
    \item You can also intercept \ce{C-H} activations to do carbonylations.
    \item Examples:
    \begin{itemize}
        \item \ce{C6H6 + CO ->[Pd(OAc)2 (10\%)][K2S2O8 / TFA] C6H5CO2H (100\%)}\footnote{Note that TFA is \underline{t}ri\underline{f}luoro \underline{a}cetic acid.}.
        \begin{itemize}
            \item Persulfate is our oxidant here.
            \item Run at room temperature for 20 hours.
        \end{itemize}
        \item \ce{C6H12 + CO ->[RT (10\%)] C6H11CO2H (4\%)}.
        \item \ce{CH4 + CO ->[VO(acac)2, \SI{80}{\celsius}][K2S2O8 / TFA] HC2H3O2 (93\%)}\footnote{\ce{VO(acac)2} is vanadyl acetylacetanate.}.
        \begin{itemize}
            \item Run at $\SI{80}{\celsius}$ for 20 hours.
            \item $\text{TON}=18$.
            \item We can also do this without persulfate:
        \end{itemize}
        \item \ce{CH4 + CO + \frac{1}{2}O2 ->[RhCl3][H2O] HC2H3O2}.
        \begin{itemize}
            \item Run at $\SI{100}{\celsius}$.
        \end{itemize}
        \item \ce{2CH4 + H2SO4 ->[Pd^{II}] CH3CO2H + 4SO2 + 6H2O}.
        \begin{itemize}
            \item Work by Periana.
            \item \ce{C-C} coupling reactivity (related to Fischer-Tropsch).
            \item Can pretty easily be pushed to make methane: Reverse water-gas shift to syn gas, Fischer-Tropsch to methane, couple methane to an acetic acid product, to make a chain, to make higher order \ce{C-C}-coupled products, to feed into other industrial processes.
            \item Mechanism shown (not a ton is known, but there are a lot of proposed possible interconversions).
        \end{itemize}
    \end{itemize}
    \item The key equilibrium that one needs to control for \ce{C-H} activation: \ce{M + H-CR3 <<=> M(H|CR3) <=> M(H)(CR3)}.
    \begin{itemize}
        \item Typically, the first step is the hard step because \ce{C-H} ligands are very poor ligands.
        \begin{itemize}
            \item Even if the \ce{C-H} activation is thermodynamically stable, you're stuck without the adduct.
        \end{itemize}
        \item How to control these equilibria:
        \begin{enumerate}
            \item Ligands that are good $\sigma$-donors (i.e., electron-rich metal centers) favor the products.
            \item \ce{M^{n+2}} $d^6$ is common, e.g., \ce{Ir^{I/III}}, \ce{Pt^{II/IV}}, \ce{Pd^{II/IV}}.
            \item Most common for $4d$ and $5d$ metals because higher oxidation states are more stable as you go down a column and there are stronger bonds that will push the equilibrium toward the right.
            \item Small(ish) \ce{R}'s help.
            \item Less sterically hindered metals.
            \item Electronically unsaturated metals ($\leq 16\,\e[-]$) help.
        \end{enumerate}
        \item Thermodynamics of activating \ce{C-H} bonds.
        \begin{itemize}
            \item Data listed.
            \item Conclusion 1: Thermodynamically, it's more favorable to activate secondary and tertiary \ce{C-H}'s, but sterically, it's more favorable to activate primary ones.
            \item Conclusion 2: Thermodynamically, it's more favorable to activate $\ce{H3C-H}>\ce{Ph-H}>\ce{RC#C-H}$. It's easier to activate \ce{Ph-H} bonds over \ce{H3C-H} bonds despite the inverse difference in bond strengths because metals can form $\pi$ adducts with arenes. This bypasses the hard step, allowing the oxidative addition to \ce{HMPh} to proceed relatively easily.
        \end{itemize}
    \end{itemize}
    \item Thinking more about $\sigma$ adducts and oxidative addition:
    \item Bob Bergman: \ce{Cp^*IrH2(PMe3) ->[h\nu][-H2] [Cp^*Ir(PMe3)] ->[C6H12] Cp^*Ir(PMe3)($\kappa^2${-}C6H12) ->[h\nu][chromatography] Cp^*Ir(PMe3)(Cy)(H) ->[C5H12][-C6H12] Cp^*Ir(PMe3)(H)(C5H11)}\footnote{Note that Cy is an abbreviation for cyclohexyl.}.
    \begin{itemize}
        \item There could also be $\sigma$-bond metathesis pathways, but it's likely a reductive elimination type process.
    \end{itemize}
    \item Isotope effects:
    \begin{itemize}
        \item Kinetic isotope effects (KIEs) are common mechanistic probes.
        \item Data listed.
        \item In a primary KIE, you would expect protons to react faster than deuterium because they're lighter, they tunnel faster, etc.
        \begin{itemize}
            \item Symbolically, you would expect $\frac{k_{\ce{H}}}{k_{\ce{D}}}>1$.
        \end{itemize}
        \item However, tungsten and iridium have an inverse KIE (deuterium is faster than hydrogen).
        \item Explaining KIEs (in terms of vibrational energy).
        \item Normal KIEs:
        \begin{figure}[h!]
            \centering
            \begin{subfigure}[b]{0.45\linewidth}
                \centering
                \begin{tikzpicture}[
                    xscale=1.5,
                    every node/.style=black
                ]
                    \draw [blx,thick] (0,0) node[below]{\chemfig{M(-[1]Z)(-[7]R)}}
                        to[out=0,in=180] (1,0.7) node[above]{\chemfig{M*3([:-60,1.4,,,dashed]-R-Z-)}}
                        to[out=0,in=180] (2,-0.4) node[below]{\chemfig{M-\phantom{i}-[2,0.3,,,white]Z-[6]R}}
                        to[out=0,in=180] (3,-0.2)
                        to[out=0,in=180] (4,-1.1) node[below]{\ce{M + Z-R}}
                    ;
                \end{tikzpicture}
                \caption{Reaction coordinate diagram.}
                \label{fig:KIE-normala}
            \end{subfigure}
            \begin{subfigure}[b]{0.45\linewidth}
                \centering
                \begin{tikzpicture}[
                    yscale=0.75,
                    every node/.style={black,font=\footnotesize}
                ]
                    \draw [blx,thick] (0,4) parabola bend (1,0) (2,4);
                    \draw [blx,thick] (1,4.5) parabola bend (2,3.5) (3,4.5);
        
                    \draw [grx,thick]
                        (0.5,1) -- (1.5,1)
                        (0.293,2) -- (1.707,2)
                        (0.134,3) -- (1.866,3)
        
                        (1.5,3.75) -- (2.5,3.75)
                        (1.293,4) -- (2.707,4)
                        (1.134,4.25) -- (2.866,4.25)
                    ;
                    \draw [rex,thick]
                        (0.646,0.5) -- (1.354,0.5)
                        (0.388,1.5) -- (1.612,1.5)
                        (0.209,2.5) -- (1.791,2.5)
        
                        (1.646,3.625) -- (2.354,3.625)
                        (1.388,3.875) -- (2.612,3.875)
                        (1.209,4.125) -- (2.791,4.125)
                    ;
        
                    \draw [rex,<->] (3.5,0.5) -- node[left]{$\Delta G^\ddagger_{\ce{D}}$}  (3.5,3.625);
                    \draw [grx,<->] (4,1)    -- node[right]{$\Delta G^\ddagger_{\ce{H}}$} (4,3.75);
                \end{tikzpicture}
                \caption{Energy wells.}
                \label{fig:KIE-normalb}
            \end{subfigure}
            \caption{Normal KIE.}
            \label{fig:KIE-normal}
        \end{figure}
        \begin{itemize}
            \item In Figure \ref{fig:KIE-normala}, \ce{Z} is either a proton or deuterium.
            \item Notice that this is a downhill reaction.
            \item The key transition state is the first one.
            \item Because we're eliminating a strong bond (a \ce{C-H} bond), we have a transition state with a weaker bond.
            \item Recall that $E=h\nu$ and $\nu\propto\sqrt{1/\nu}$, where $\mu$ is the effective mass of the oscillators and the transition state. Thus, energy decreases as effective mass increases.
            \item Now think of vibrational wells for the reactant and the transition state (see Figure \ref{fig:KIE-normalb}). A proton and deuterium will have identical vibrational spacing, but the levels for deuterium will be lower because deuterium has lower energy. Note also that the spacing is closer together in the wider well because it is wider.
            \item Thus, we have here that $\Delta G^\ddagger_{\ce{H}}<\Delta G^\ddagger_{\ce{D}}$ due to zero-point energy effects.
            \item Essentially, if the transition state has weaker bonds than the starting material, a normal KIE is expected.
        \end{itemize}
        \item Inverse KIE:
        \begin{figure}[h!]
            \centering
            \begin{subfigure}[b]{0.45\linewidth}
                \centering
                \begin{tikzpicture}[
                    xscale=1.5,
                    every node/.style=black
                ]
                    \draw [blx,thick] (0,0) node[below]{\chemfig{M(-[1]Z)(-[7]R)}}
                        to[out=0,in=180] (1,1)
                        to[out=0,in=180] (2,0.7) node[below]{\chemfig{M-\phantom{i}-[2,0.3,,,white]Z-[6]R}}
                        to[out=0,in=180] (3,1.4) node[above]{\chemfig{M-[,,,,dashed]\phantom{i}-[2,0.3,,,white]Z-[6]R}}
                        to[out=0,in=180] (4,-0.7) node[below]{\ce{M + Z-R}}
                    ;
                \end{tikzpicture}
                \caption{Reaction coordinate diagram.}
                \label{fig:KIE-inversea}
            \end{subfigure}
            \begin{subfigure}[b]{0.45\linewidth}
                \centering
                \begin{tikzpicture}[
                    yscale=0.75,
                    every node/.style={black,font=\footnotesize}
                ]
                    \draw [blx,thick] (1,4.5) parabola bend (2,0.5) (3,4.5);
                    \draw [blx,thick] (0,1) parabola bend (1,0) (2,1);
        
                    \draw [grx,thick]
                        (1.5,1.5)   -- (2.5,1.5)
                        (1.293,2.5) -- (2.707,2.5)
                        (1.134,3.5) -- (2.866,3.5)
        
                        (0.5,0.25)   -- (1.5,0.25)
                        (0.293,0.5)  -- (1.707,0.5)
                        (0.134,0.75) -- (1.866,0.75)
                    ;
                    \draw [rex,thick]
                        (1.646,1) -- (2.354,1)
                        (1.388,2) -- (2.612,2)
                        (1.209,3) -- (2.791,3)
        
                        (0.646,0.125) -- (1.354,0.125)
                        (0.388,0.375) -- (1.612,0.375)
                        (0.209,0.625) -- (1.791,0.625)
                    ;
        
                    \draw [rex,<->] (3.5,0.125) -- node[left]{$\Delta G^\ddagger_{\ce{D}}$}  (3.5,1);
                    \draw [grx,<->] (4,0.25)    -- node[right]{$\Delta G^\ddagger_{\ce{H}}$} (4,1.5);
                \end{tikzpicture}
                \caption{Energy wells.}
                \label{fig:KIE-inverseb}
            \end{subfigure}
            \caption{Inverse KIE.}
            \label{fig:KIE-inverse}
        \end{figure}
        \begin{itemize}
            \item Notice that this is an uphill reaction.
            \item The key transition state is the second one.
            \item Here, we have a transition state with a stronger bond.
            \item Thus, we have here that $\Delta G^\ddagger_{\ce{D}}<\Delta G^\ddagger_{\ce{h}}$.
            \item Essentially, an inverse KIE originates from a transition state having stronger bonds than the starting material, i.e., "later" transition states.
        \end{itemize}
        \item This can also arise from equilibrium isotope effects.
    \end{itemize}
    \item \textcite{bib:BernskoetterBrookhart}: Observed $\sigma$ adducts with \ce{CH4}.
    \begin{itemize}
        \item Took rhodium bound to a pincer ligand and a methyl group.
        \item Treated it with an acid (\ce{[HBAr^F]*2Et2O} [an etherate]) and a funky solvent (\ce{CD2Cl2F}) at a very cold temperature ($\SI{-110}{\celsius}$) to make a rhodium $\kappa^2$ methane complex.
        \item With NMR spectroscopy, they can tell that the methane is rapidly spinning around, so all hydrides are bonding now and then (equilibrating).
        \item Comparing the \ce{C-H} coupling constants for the adduct vs. free methane reveals a slight weakening of the \ce{C-H} bond, but more importantly, the \ce{Rh-H} coupling constant is quite a bit weaker than you would expect for a full hydride.
    \end{itemize}
    \item Primary take away: Directed \ce{C-H} activation is a very powerful technique because you can put these \ce{C-H} bonds in close proximity, but one of the biggest challenges is forming $\sigma$ adducts; in doing this, we can observe the weird inverse KIEs of Figures \ref{fig:KIE-normal} and \ref{fig:KIE-inverse}.
\end{itemize}




\end{document}