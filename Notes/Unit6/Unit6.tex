\documentclass[../notes.tex]{subfiles}

\pagestyle{main}
\renewcommand{\chaptermark}[1]{\markboth{\chaptername\ \thechapter\ (#1)}{}}
\setcounter{chapter}{5}

\begin{document}




\chapter{???}
\section{Lecture 20: Fischer-Tropsch}
\begin{itemize}
    \item \marginnote{5/19:}Industrial process mainly, but also has applications to renewable energy.
    \item Mechanism of the industrial catalysis in the first video; more on the molecular chemistry in the second video.
    \item General form:
    \begin{equation*}
        \ce{\{CO + H2\} -> {fuel (gasoline)}}
    \end{equation*}
    \begin{itemize}
        \item Clearly, the reactants are Syn gas.
    \end{itemize}
    \item Thermodynamics of \ce{CO + $x$ H2 -> {stuff}}.
    \begin{itemize}
        \item As $x$ increases, $\Delta H$ and $\Delta G$ decrease.
        \item Conclusion: The more \ce{C-H} bonds you make, the more thermodynamically favorable the process is.
        \begin{itemize}
            \item A lot of this comes from the thermodynamic reducing power of hydrogen.
        \end{itemize}
    \end{itemize}
    \item History:
    \begin{itemize}
        \item 1914: \ce{CO + 2H2 ->[Fe][silica supports] \frac{1}{n}C_nH_{2n} + H2O}.
        \begin{itemize}
            \item The ratio of the reactants can be tuned.
            \item Iron is a solid-state catalyst here. Note that other metals can be used, but iron was the most common.
            \item The products are a mixture of medium to long chain alkanes, alkenes, and methane.
            \item The challenge and limitation of this process is that it yields a mixture (a Schultz-Flory distribution) of products.
            \item This means that there is no real selectivity, although \ce{C7} is most common.
            \item This is also why chemists became interested in homogeneous catalysis for this area, because in principle it could give you better selectivity here.
        \end{itemize}
    \end{itemize}
    \item Industrially:
    \begin{equation*}
        \ce{{coal} ->[steam][reformed] CO + H2}
    \end{equation*}
    \begin{itemize}
        \item This is commonly done in South Africa because they have a lot of coal.
        \begin{itemize}
            \item \ce{CH4} can also be used as a reactant.
        \end{itemize}
        \item The ratio of the product gasses can be tuned by the water-gas shift reaction (see Lecture 17), which is \ce{CO + H2O <=> H2 + CO2} (in the forward direction).
        \begin{itemize}
            \item \ce{CO2} can then be removed.
            \item One potentially interesting exciting direction that's not been industrially accomplished due to cost is the reverse water-gas shift reaction, which is just the reverse direction.
            \item The reverse water-gas shift reaction is interesting because if you could convert \ce{CO2} into \ce{CO + H2O} and remove the water, you could feed the \ce{CO} back into the original process to make gasoline.
        \end{itemize}
        \item Challenges:
        \begin{itemize}
            \item Selectivity in the Fischer-Tropsch process.
            \item Source of \ce{H2} (right now we get it from fossil fuels). There is research into how to reduce \ce{CO2} into fuels, but if we can get \ce{H2} from somewhere other than fossil fuels, it's a solved problem.
        \end{itemize}
    \end{itemize}
    \item Conditions (for the following reaction):
    \begin{equation*}
        \ce{\{CO + H2\} -> C_nH_{2n} + $n$H2O}
    \end{equation*}
    \begin{itemize}
        \item High temperatures (150-$\SI{300}{\celsius}$).
        \begin{itemize}
            \item Favor faster rates and better conversion yield percentages, but more methane (the least valuable product).
        \end{itemize}
        \item High pressure (1-$\SI{30}{atm}$).
        \begin{itemize}
            \item Favors longer chain alkanes and better conversion since methane is a gas (lower likelihood of eliminating it).
        \end{itemize}
        \item Optimal \ce{CO} / \ce{H2} concentration.
        \begin{itemize}
            \item Cobalt catalysts use $1:2$.
            \item Iron catalysts should have relatively more \ce{CO}.
        \end{itemize}
    \end{itemize}
    \item Product (Schultz-Flory) distribution:
    \begin{equation*}
        \frac{W_n}{n} = (1-\alpha)^2\alpha^{n-1}
    \end{equation*}
    \begin{itemize}
        \item $\frac{W_n}{n}$ is the average molecular weight?
        \item $W_n$ is the weight fraction of hydrocarbons containing $n$ carbon atoms and $\alpha$ is the chain growth probability (controlled by the catalyst and the conditions).
        \item If $\alpha<0.5$, then methane dominates.
        \item However, as $\alpha\to 1$, the methane fraction decreases relative to the sum of heavy products.
        \item Sasol (a South African company) mediates this process.
        \item We can use the products for a variety of different things (heptane for fuel, heavier ones for wax [you can buy Sasol wax], etc.).
        \item If we were going to do this globally for fuel, we'd have a lot of wax as a byproduct, which we neither have a place to store nor can afford to do since it's so wasteful. Thus, there is great interest in selectivity for \ce{C6}, \ce{C7}, \ce{C8}, or even butene, from which we could do olefin metathesis to make hexanes.
        \item Two necessary things to use this process for renewable energy:
        \begin{enumerate}
            \item Find a good source of hydrogen so that we could run the reverse water-gas shift reaction.
            \begin{itemize}
                \item The reaction might need to be optimized a bit as well, but in principle that can be done; hydrogen is the real problem.
            \end{itemize}
            \item Improve the selectivity of the Fischer-Tropsch chemistry.
        \end{enumerate}
    \end{itemize}
    \item As with the Haber-Bosch process, this is solid support chemistry.
    \item Mechanism (initial proposal):
    \begin{itemize}
        \item Solid-state iron binds hydrides and \ce{CO} on its surface, leading the formation of a formyl species before adding another hydrogen to go to a bound alcohol species, losing \ce{H2O} to go to a bound carbene (the carbene is important because it facilitates chain growth), and then proceeding to a bound methyl species. With enough methyl species, you could do other things.
        \item Gerhardt Ertl proves this wrong.
    \end{itemize}
    \item Mechanism (actual):
    \begin{itemize}
        \item Solid-state iron binds hydrides and carbonyls on its surface. Next, we split \ce{CO} to make bound carbides and bound oxides. With more \ce{H2}, we can mix and scramble the carbides, oxides, and hydrides, kicking out \ce{H2O} and making bound carbene species, which can then go on and form the product that we want as described in the initial proposal.
        \item Proven with in situ studies performed on the surface of these materials.
    \end{itemize}
    \item Another key contribution to the proof came from molecular chemistry, which showed that formyl species are generally unstable:
    \begin{figure}[h!]
        \centering
        \begin{subfigure}[b]{\linewidth}
            \centering
            \schemestart
                \chemfig{\ce{L_n}M(-[1]CO)(-[7]H)}
                \arrow{<<->}
                \chemfig{\ce{L_n}M-[:30](=[2]O)-[:-30]H}
                \+{,,1.8em}
                \chemfig{\ce{L_n}@{M}M-[:30](=[2]@{O}\charge{180=\:}{O})-[:-30]H}
            \schemestop
            \chemmove{\draw [blx,semithick,shorten >=1pt,shorten <=4pt] (O) to[out=180,in=90] (M);}
            \caption{Formyl species.}
            \label{fig:formylInstabilitya}
        \end{subfigure}\\[1em]
        \begin{subfigure}[b]{\linewidth}
            \centering
            \schemestart
                \chemfig{\ce{L_n}M(-[1]CO)(-[7]CH_3)}
                \arrow{<->>}
                \chemfig{\ce{L_n}M-[:30](=[2]O)-[:-30]}
            \schemestop
            \caption{Acyl species.}
            \label{fig:formylInstabilityb}
        \end{subfigure}
        \caption{Stability of metal-carbonyl derivatives.}
        \label{fig:formylInstability}
    \end{figure}
    \begin{itemize}
        \item In Figure \ref{fig:formylInstabilitya}, the reactant is generally favored because it has two bonds instead of one, and metal-hydride bonds are highly thermodynamically favored over metal alkyl bonds.
        \item In Figure \ref{fig:formylInstabilityb}, the product is generally favored.
        \item Think about the equilibrium in terms of the nucleophilicity of the X group. Hydrides are not nucleophilic, whereas alkyl species are.
    \end{itemize}
    \item The synthesis of metal formyl species was pioneered by Jim Collman in 1973.
    \begin{equation*}
        \ce{Fe(CO)5 ->[Na][-HOC-O-COMe] Fe(CO)4^2- ->[][-OAc^-] Fe(CO)4(COH)- ->[][-CO] HFe(CO)4-}
    \end{equation*}
    \begin{itemize}
        \item The first step is ill defined, and the products of it are a messy mixture of carbon-containing products.
        \item The second intermediate (the formyl species) is unstable and will decompose over time to form the final product.
    \end{itemize}
    \item Chuck Casey and John Gladys found stable formyl species (the product of the following reaction):
    \begin{equation*}
        \ce{CpRe(CO)(NO)(L) ->[HBR3^-] CpRe(NO)(L)(COH)}
    \end{equation*}
    \begin{itemize}
        \item The reactant is a chiral, $18\,\e[-]$, $d^6$ species.
    \end{itemize}
    \item John Bercaw did more with metal-formyl species:
    \begin{equation*}
        \ce{Cp^*2ZrH2 + O#C-ML_n ->[][fast] L_nM=CH-O-ZrCp^*2H}
    \end{equation*}
    \begin{itemize}
        \item The first species is an extremely hydridic metal hydride that has very similar properties to \ce{LiAlH}. It is a $16\,\e[-]$, $d^0$ species.
        \item This chemistry is driven by the strength of the zirconium bond ($\approx\SI{130}{kcal\per\mole}$).
        \item Note that although this is technically a formyl species, it's in practice more of a Fischer carbene structure.
        \begin{itemize}
            \item The bonding is very different and the Lewis acid activation of that strong \ce{Zr^{IV}} Lewis acid makes this more of a carbene structure.
            \item This is useful in \ce{C-C} coupling reactions.
        \end{itemize}
        \item One example of \ce{ML_n} is \ce{Cp2W}.
        \item Another one is \ce{Cp2Nb(H)(CO) + Cp^*2ZrH2 -> Cp2Nb(H)(CH-O-ZrCp^*2H) <=>[fast]}\\ \ce{Cp2Nb(H)(CH2-O-ZrCp^*2) ->[H2] Cp2NbH3 + Cp^*2Zr(H)(OCH3) ->[][no rxn]}.
        \begin{itemize}
            \item \ce{Nb} is niobium.
            \item A major problem is once we form that extremely strong metal-oxygen bond, it will not break, so it's very hard to get \ce{H2O} off of these species.
        \end{itemize}
    \end{itemize}
    \item How we think about the molecular systems that form \ce{C-C} bonds.
    \begin{itemize}
        \item This is ultimately what happens on solid support surfaces in industrial catalysis.
        \item The challenge is that those industrial systems don't have selectivity; molecular systems could in principle provide this (this is an area of active research).
    \end{itemize}
    \item To recap, we just talked about the potential of early metal hydrides to activate \ce{CO} (Bercaw's niobium system took it all the way to a methanol equivalent).
    \item Bercaw (a titan of organometallic chemistry):
    \begin{figure}[h!]
        \centering
        \schemestart
            \chemfig{\ce{Cp^*2}Zr(-[1]Me)(-[7]Me)}
            \arrow{->[\small\ce{CO}]}
            \chemleft{[}
                \subscheme{
                    \chemfig{\ce{Cp^*2}@{Zr1}Zr(-[1]Me)(-[7]C(-[6])=[1]@{O1}\charge{[extra sep=2.5pt]180=\:}{O})}
                    \arrow{<->}
                    \chemfig{\ce{Cp^*2}@{Zr2}Zr(-[1]Me)(-[7]O-[5]@{C2}\charge{[extra sep=2.5pt]90=\:}{C}-[6])}
                }
            \chemright{]}
            \arrow{->[\small\ce{CO}][
                \footnotesize
                \chemleft{[}
                    \chemfig{\ce{Cp^*2}@{Zr3}Zr*5([:-36]-O-@{C3b}\charge{150=\:}{C}(-)-[,,,,white]@{C3a}\charge{-150=\:}{C}(-)-O-)}
                \chemright{]^\ddagger}
            ]}[,2.3]
            \chemfig{\ce{Cp^*2}Zr*5([:-36]-O-(-)=(-)-O-)}
        \schemestop
        \chemmove{
            \draw [blx,semithick,shorten <=5pt,shorten >=2pt] (O1)  to[bend right=20]  (Zr1);
            \draw [blx,semithick,shorten <=5pt,shorten >=2pt] (C2)  to[bend left=20]   (Zr2);
            \draw [blx,semithick,shorten <=5pt,shorten >=2pt] (C3a) to[out=-150,in=15] (Zr3);
            \draw [blx,semithick,shorten <=5pt,shorten >=2pt] (C3b) to[out=150,in=-15] (Zr3);
            \draw [blx,shorten <=2pt,shorten >=2pt,stealth-stealth] (C3a) to[bend left=20] (C3b);
        }
        \caption{Bercaw's first \ce{C-C} coupling.}
        \label{fig:CCcoupling-Bercaw}
    \end{figure}
    \begin{itemize}
        \item The first ever \ce{C-C} coupling that was observed in a well-defined and pretty clean way from \ce{CO}.
        \begin{itemize}
            \item Using methyl groups makes it imperfect.
            \item The mechanism isn't entirely clear.
        \end{itemize}
        \item The second step proceeds from the second resonance structure.
        \item Additionally, Bercaw studied reactions with \ce{Cp^*2ZrH2CO}, a compound with virtually no $\pi$ backbonding:
        \begin{itemize}
            \item \ce{Cp^*2ZrH2CO ->[H2] Cp^*2Zr(H)(OCH3)}, where \ce{-OCH3} is a methanol equivalent.
            \item \ce{Cp^*2ZrH2CO ->[CO] Cp^*2Zr(H)-O-=-O-Zr(H)(Cp^*2)}. Here, we actually see \ce{C-C} coupling.
            \item \ce{Cp^*2ZrH2CO ->[CO, H2] Cp^*2Zr(H)-O---O-Zr(H)(Cp^*2)}. This is even closer to ethylene glycol, but the \ce{Zr-O} bonds are still an issue.
            \item \ce{Cp^*2ZrH2 + Cp^*2Zr(CO)2 -> Cp^*2Zr(H)-O-=-O-Zr(H)(Cp^*2)}
        \end{itemize}
    \end{itemize}
    \item Similarly, Tobin Marks looked at \ce{Cp^*2Th(OR)(H)} (thorium hydride alkoxides), where $\ce{R}=\ce{CH($t${-}Bu)2}$ is a pretty bulky super-isopropal ligand abbreviated \ce{$i${-}Pr^*}:
    \begin{figure}[h!]
        \centering
        \schemestart
            \chemfig{\ce{Cp^*2}Th(-[1]H)(-[7]OR)}
            \arrow{->[\small\ce{CO}]}
            \chemfig{\ce{Cp^*2}Th(-[2,,2]CO)(-H)(-[6,,2]OR)}
            \arrow
            \chemfig{\ce{Cp^*2}@{Th}Th(-[7]OR)(-[:30](=[2]@{O}\charge{180=\:}{O})-[:-30]H)}
            \arrow{->[][\SI{25}{\celsius}]}
            \chemfig{Th(-[7]OR)-[1]O-[1]=_-[7]O-[7]Th(-[5]RO)}
        \schemestop
        \chemmove{\draw [blx,semithick,shorten <=4pt,shorten >=2pt] (O) to[out=180,in=90] (Th);}
        \caption{Marks' \ce{C-C} coupling.}
        \label{fig:CCcoupling-Marks}
    \end{figure}
    \begin{itemize}
        \item The theme with early metals: If you can get to the Fischer carbene type structure, you can get coupling.
    \end{itemize}
    \item Pete Wolczanski worked with tantalum silox complexes.
    \begin{itemize}
        \item \ce{Ta(OSi($t${-}Bu)3)3 + 2CO -> 2({silox})3TaO + ({silox})3Ta=C=C=Ta({silox})3}.
        \item This is \ce{C-C} coupling \emph{and} complete deoxygenation.
    \end{itemize}
    \item Challenge with all of this chemistry: early-metal oxygen bonds are an extreme thermodynamic sink.
    \begin{itemize}
        \item Unfortunately, the transition from metal-carbon bonds to metal-oxygen bonds is also the driving force of this reactivity.
        \item We can get around this issue by trapping the oxygen with other electrophiles.
    \end{itemize}
    \item Kit Cummins: \ce{MoL3 ->[CO] L3MoCO ->[1Na] Na[L3MoCO] ->[Bu^t-COCl][-NaCl] L3Mo#C-O-COBu^t ->[\Delta][-CO2,H2C=CMe2] L3Mo#C-H ->[KBn] [L3Mo#C]-}.
    \begin{itemize}
        \item $\ce{L}=\ce{N($t${-}Bu)({xylyl})}$.
        \item The third step is more explained in organic chemistry, where an electrophile like the one added (pivaloyl chloride) should create an ester speies and lose \ce{NaCl}.
        \item Note that \ce{KBn} is benzyl potassium.
        \item There are very few examples of terminal carbides, but note that this is similar to what happens on the surface of the solid state iron catalyst.
        \item This isn't perfect because we're releasing \ce{CO2}, but it is a step where the oxygen isn't trapped on the metal center.
    \end{itemize}
    \item Theodor Agapie (with his graduate student Joshua Buss):
    \begin{figure}[h!]
        \centering
        \schemestart
            \chemfig{Mo?[Mo](-[,1.3]PR_2-[:-60,,1]**6(-(-**6(-?[Mo,,dashed]-?[Mo,,dashed]-?[L]---))-----))(-[4,1.3]R_2P-[:-120,,2]?[close]-[:-60]?[L]-[::-60]-[::-60]-[::-60]-[::-60]?[close]**6([::-120]))(>:[:65]CO)(<[:115]OC)}
            \arrow{0}[,0.1]\+
            $4\,\e[-]$
            \+
            $4\,\text{E}^+$
            \arrow
            \chemfig{LMo{(}N_2{)}}
            \+
            \chemfig{E_2O}
            \+\arrow{0}[,0.1]
            \chemfig{E-[:60]O-C~C-E}
        \schemestop
        \caption{Agapie's \ce{C-C} coupling.}
        \label{fig:CCcoupling-Agapie}
    \end{figure}
    \begin{itemize}
        \item \ce{KC8} yields the $4\,\e[-]$.
        \item The electrophile $\text{E}^+$ is something like TMSCl, a silane, or another silyl electrophile.
        \item This also proceeds through carbides.
    \end{itemize}
    \item Jonas Peters (on iron): \ce{Fe#C-O + TMSCl -> Fe#C-OTMS}.
    \begin{itemize}
        \item We are trapping the oxygen with an oxygen-silion bond, which thermodynamially are quite strong.
        \item Problem: While a silyl electrophile is substantially better than trapping the metal center (which entirely precludes catalysis), the silyl electrophiles generate stoichiometric waste. We need to trap the oxygen with water, not a silyl ether.
        \item Thus, what you ultimately want to do is use later metals (like iron), but use hydrogen instead.
    \end{itemize}
    \item Dan Suess (graduate student of Jonas Peters):
    \begin{figure}[h!]
        \centering
        \chemfig{M?[M](-[2,1.7]P(-Pr^i)(-[2]Pr^i)-[4,1.7]**6(-----(-[6]B?[M,,dashed]?[B]>:[1]?[M,,dashed]**6([:45,0.5]------))-))(-[6,1.7]P(-Pr^i)(-[6]Pr^i)-[4,1.7]**6(-?[B]-----))}
        \caption{The DPB ligand.}
        \label{fig:DPB}
    \end{figure}
    \begin{itemize}
        \item Investigates various reactions of DPB-bound irons.
        \item \ce{DPBFe-N#N-FeDPBB ->[4CO] DPBFe(CO)2 ->[XS CO] FeP2B(CO)3}.
        \begin{itemize}
            \item The product here is formally \ce{Fe^0}, but actually slightly higher than that.
        \end{itemize}
        \item \ce{DPBFe-N#N-FeDPBB ->[4CO] DPBFe(CO)2 ->[H2] FeP2($\kappa^2${-}BH)H(CO)2}.
        \item \ce{DPBFe-N#N-FeDPBB ->[4CO] DPBFe(CO)2 ->[1K^0] [FeP2($\kappa^2${-}PhB)(CO)2]- ->[XS K^0] [FeP2B(CO)2]^2- ->[2TMSOTf]} a very weird product.
        \begin{itemize}
            \item The third intermediate is highly activated and somewhat analogous to \ce{Fe(CO)5^2-}.
            \item The product has a \ce{-C-OTMS} substituent, as well as a tridentate derivative of the DPB ligand that still bonds through the two phosphines, but instead of partially bonding through the boron and the ipso-carbon of the phenyl ring, the boron binds to an additional \ce{C-OTMS} ligand through the carbon, which in turn binds back to the iron center.
            \item This product has many resonance structures; one with particular merit is \ce{Fe(#C-OTMS)2} with one of the triple bonds bonding datively to the boron, although it implies an \ce{Fe^{VI}} oxidation state that is likely an overestimation, even though the iron is certainly high-valent here.
            \item Note that if the product is treated with hydrogen, we selectively create (Z)-\ce{TMSO-=-OTMS}.
            \item This is not catalytic, but it is one of the few examples where we don't generate an intractable metal-oxide, and we do generate a \ce{C-C} coupled product.
        \end{itemize}
    \end{itemize}
    \item Conclusion: There's a lot of interesting organometallic chemistry surrounding \ce{C-O} activation and getting that prepped for functionalization, specifically via \ce{C-C} bond formation.
    \item This is not a solved problems.
    \begin{itemize}
        \item While molecular chemistry offers the promise of selectivity, there are real challenges with activating it and getting \ce{H2O} out instead of trapping the oxygen with silyl electrophiles.
        \item Part of the issue is the stability of the molecular complexes (they're not as stable as a solid iron surface) and you need a lot of energy to split \ce{C#O} bonds. On the other hand, if you apply that energy thermally, it will hurt your selectivity.
    \end{itemize}
    \item There's some extra molecular chemistry in the notes.
\end{itemize}




\end{document}