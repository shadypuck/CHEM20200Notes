\documentclass[../notes.tex]{subfiles}

\pagestyle{main}
\renewcommand{\chaptermark}[1]{\markboth{\chaptername\ \thechapter\ (#1)}{}}
\setcounter{chapter}{3}

\begin{document}




\chapter{More Catalytic Processes}
\section{Lecture 15: Nitrogen Fixation}
\begin{itemize}
    \item \marginnote{5/3:}General form:
    \begin{equation*}
        \ce{N2 + 6H+ + 6e- -> 2NH3}
    \end{equation*}
    \begin{itemize}
        \item Alternatively, it could be \ce{N2 + 3H2 -> 2NH3}.
        \item A simple reaction on paper, but in practice and mechanistically, very difficult.
    \end{itemize}
    \item Nitrogen is one of the essential elements for life (CHNOPS), but we can't absorb it from the air in its elemental form as \ce{N#N} gas. Indeed, we need it to be \textbf{fixed} before we can incorporate it (i.e., through food sources).
    \item Main pathways:
    \begin{itemize}
        \item Lightning.
        \begin{align*}
            \ce{N2 + O2 ->[$h\nu$] NO, NO2^-, NO3-}
        \end{align*}
        \begin{itemize}
            \item 4-10 million tons per year.
            \item Nitrate can be used by organisms; converted into useful nitrogen.
        \end{itemize}
        \item Natural enzymatic fixation.
        \begin{itemize}
            \item 100-300 million tons per year (probably closer to the 100 side).
            \item 40/60 ocean/land ratio.
            \item Done by the enzyme nitrogenase, with \ce{FeMoCO}\footnote{Literally pronounced as its written, i.e., "fih-MOH-koh."} as a cofactor.
        \end{itemize}
        \item Mankind.
        \begin{itemize}
            \item About 190 million tons per year.
            \item Thus, mankind is currently fixing more nitrogen than all natural sources combined by a fair amount right now. This is what allows us to feed the planet at its current population.
            \item Accomplished by the \textbf{Haber-Bosch process}.
        \end{itemize}
    \end{itemize}
    \item \textbf{Haber-Bosch process}: Humanity's primary method of fixing nitrogen.
    \begin{equation*}
        \ce{N2 + 3H2 <=>[$\Delta, P$][Fe/Mg] 2NH3}
    \end{equation*}
    \begin{itemize}
        \item $-\SI{92}{kcal\per\mole}$ (thermodynamically favored but kinetically difficult).
        \item $\SI{200}{atm}$ of pressure (pushes the equilibrium to the right via Le Ch\^{a}telier's principle).
        \item $\SI{400}{\celsius}$ (required for the rate).
        \item Potassium is an activator.
    \end{itemize}
    \item 2 Nobel prizes for this: Fritz Haber (1918) and Carl Bosch (1931).
    \begin{itemize}
        \item Controversial at the time because the process facilitated the explosives industry and Haber was a Nazi.
        \item World War I and II would not have been sustainable for Germany without this process.
    \end{itemize}
    \item Mechanism:
    \begin{itemize}
        \item On the surface of the iron catalyst, the pressure binds \ce{H2} gas as hydrides and \ce{N2} gas as bridging nitrides (between various iron atoms at the surface). It follows in a statistical and thermodynamic manner that amine ligands will be formed on the surface attached to the iron. These can then break off into ammonia gas.
        \item A solid state heterogeneous process.
        \item Nobel Prize (2007) to Gerhard Ertl for this mechanism.
    \end{itemize}
    \item This process is highly efficient, but very energy intensive as well.
    \begin{itemize}
        \item As such, there is a race to find a less energy-intensive catalytic alternative.
    \end{itemize}
    \item Molecular systems: The nitrogenase enzyme.
    \begin{equation*}
        \ce{N2 + 8H+ + 8e- ->[ATP] 2NH3 + H2}
    \end{equation*}
    \begin{itemize}
        \item Other extant cofactors include \ce{FeVCO} and \ce{FeFeCO}, but they are less common.
        \item Since ATP is used, this is still a very energy-intensive process.
        \item \ce{N2} bonds to metal centers in many ways:
        \begin{itemize}
            \item Linear, bent, side-on, bridging linear, bridging side-on, multi-metal center bridging, etc.
        \end{itemize}
        \item \ce{N2} complexes are much less common than \ce{CO} complexes since \ce{N2} is a terrible ligand.
        \begin{itemize}
            \item The HOMO in \ce{N2} makes it a $\sigma$ donor, but \ce{CO}'s negative formal charge on the carbon makes it a better $\sigma$ donor.
            \item Additionally, \ce{CO} is a much better $\pi$ acid due to its polarization.
        \end{itemize}
    \end{itemize}
    \item Allen and Senoff (in 1965) reported the first dinitrogen complex.
    \begin{equation*}
        \ce{RuCl3*3H2O + N2H4*H2O -> [Ru^{II}(N2)(NH3)5]^2+[Cl2]^2-}
    \end{equation*}
    \begin{itemize}
        \item \ce{N2H4} is hydrazine (rocket fuel), and is much more stable as a hydrate.
        \item The product is $d^6$ with $18\,\e[-]$ and has $\mu_{\ce{N2}}=2170$-$\SI{2115}{\per\centi\meter}$ (the range depends on the anion).
        \item For reference, free \ce{N2} has a stretching frequency of $\SI{2331}{\per\centi\meter}$.
    \end{itemize}
    \item Yamamoto gives the first example of a dinitrogen complex formed from free \ce{N2} (\ce{HCo(PPh3)3(N2)}).
    \item Joe Chatz, George Leigh, and Dilworth in Sussex (of the British/American camp), and Hidai and Nishibayashi (of the Japanese camp) became interested in nitrogen fixation following Yamamoto's work.
    \begin{itemize}
        \item Second and third row Group 6 complexes such as molybdenum were the focus.
        \item Example: \ce{MoX4L2 ->[N2, L][Na/Hg] Mo(N2)2L4}.
        \begin{itemize}
            \item \ce{L} is typically a phosphine.
            \item With various phosphine ligands, you can get different geometries.
        \end{itemize}
        \item Another example: \ce{\emph{cis}{-}W(N2)2(PMe2Ph)4 ->[H2SO4][MeOH] 2NH3 + N2 + W^{VI}({oxo})}.
        \begin{itemize}
            \item Works with molybdenum, too, but tungsten gives a better yield.
            \item The overall reaction is \ce{W^0 + 6H+ + N2 -> 2NH3 + W^{VI}}.
            \item Some important subreactions/steps are \ce{N2 + 4e- -> N2H4} and \ce{N2 + 2e- -> N2H2}.
        \end{itemize}
    \end{itemize}
    \item Mechanism (by Chatt):
    \begin{align*}
        \ce{M-N2} &\ce{->[H+]} \ce{M-N=N-H}\\
        &\ce{->[H+]} \ce{M#N-NH2}\\
        &\ce{->[H+]} \ce{M(PR3)4(Cl)(#N-NH3)}\\
        &\ce{->[][-NH3]} \ce{M#N}\\
        &\ce{->[H+]} \ce{M=NH}\\
        &\ce{->[H+]} \ce{M-NH2}\\
        &\ce{->[H+][-NH3]} \ce{M^{6+}}
    \end{align*}
    \begin{itemize}
        \item Very few compounds in the form of the first intermediate (diazene) are known. These compounds are very susceptible to $\beta$-\ce{H} elimination from water, though.
        \item The second intermediate has a hydrazido ligand.
        \item The fourth intermediate has a nitride.
        \item The fifth intermediate has an imide.
        \item The sixth intermediate has an amide ligand.
    \end{itemize}
    \item Hidai uses silanes: \ce{M(N2)2(PR3)4 ->[TMSI] TMS-N=N-MI(PR3)4 ->[Na, THF][N2] M(N2)2(PR3)4 + NH(TMS)2 + NH3 + NaNH2 + NaNH(TMS)}, etc.
    \item \textcite{bib:SchrockYandulov} creates a well-defined catalyst for nitrogen fixation.
    \begin{itemize}
        \item The catalyst is molybdenum bound to dinitrogen and a tridentate TREN scaffold anion (with hexaisopropal \emph{tert}-phenyl (HIPT) aryl groups).
        \item They add eight equivalents of cobaltocene (\ce{CoCp2}), seven equivalents of lutidinium (a pyridinium acid), and the bulky, noncoordinating counteranion \ce{[BAr^F{}_4]-}.
        \item They fish out \ce{Mo-NH3 + NH3}. Reducing the former product gives \ce{Mo^{III}(N2)} with a weak reductant.
        \begin{itemize}
            \item If you use a slightly stronger reductant (decamethylchromocene; \ce{CrCp^*2}) and a slow addition of acid, you get approximately 8 turnovers.
        \end{itemize}
        \item This is not a great yield, but the important part is that it exists and they can observe all of the intermediates.
        \begin{itemize}
            \item Observing said intermediates verified the mechanism proposed by Chatt and Hidai.
        \end{itemize}
        \item To reiterate, this is the Chatt/Distal\footnote{Note that Chatt and Distal are interchangeable synonyms.} cycle they observed: \ce{Mo-N#N ->[H+,e-] M-N=N-H ->[H+,e-] M#N-NH2 ->[H+,e-] M#N ->[H+,e-]M#NH ->[H+,e-] M-NH2 ->[H+,e-,N2][-NH3] Mo-N#N}.
        \begin{itemize}
            \item Note that molybdenum starts in the $3+$ oxidation state at \ce{M-N#N} and goes up to $6+$ at \ce{Mo#N} before cycling back down.
        \end{itemize}
    \end{itemize}
    \item Shilov had a number of systems, but they were poorly defined.
    \item Other selected examples.
    \begin{itemize}
        \item Zirconium can do this catalysis in some cases.
        \item \textcite{bib:CumminsLaplaza} find that molybdenum trisannelides and dinitrogen can go through a kinked transition state to yield two equivalents of \ce{L3Mo#N}.
    \end{itemize}
    \item Iron:
    \begin{itemize}
        \item Present in all enzymatic cofactors that mediate nitrogen fixation.
        \begin{itemize}
            \item Until 10-15 years ago, they thought it wasn't directly involved though.
        \end{itemize}
        \item \ce{(PR3)4Fe(N2) ->[XS H+]} only trace amounts of \ce{NH3}.
        \item \textcite{bib:NishibayashiIron} finds that simple iron salts and even substituted ferrocenes react with a strong reductant, sodium, and \ce{TMSCl} to make \ce{N(TMS)3} (which is catalytic and can be transformed into ammonia). The yield is about 25 equivalents per iron.
        \begin{itemize}
            \item This shows that iron can do this chemistry.
        \end{itemize}
        \item Dr. Anderson's thesis work with Jonas Peters finds that \ce{Fe(N2)(B(PR2Ph)3) ->[XS KC8][XS HBAr^F{}_4] NH3}\footnote{Note that the reactant is the same compound discussed in problem 3 of Homework 1.} \parencite{bib:AndersonPeters}.
        \begin{itemize}
            \item Seven equivalents per iron. More recent tinkering has brought it up to nearly 100 equivalents per iron.
            \item Mechanism: \ce{L3FeN2- ->[2H+] L3Fe#N-NH2+ -> -> -> NH3}.
            \item Suggests a Chatt/Distal mechanism.
        \end{itemize}
        \item Differing mechanistic proposal: Alternating mechanism.
        \begin{align*}
            \ce{Fe-N2} &\ce{->[H+,e-]} \ce{Fe=N=N-H}\\
            &\ce{->[H+,e-]} \ce{Fe-NH=N-H}\\
            &\ce{->[H+,e-]} \ce{Fe-NH-NH2}\\
            &\ce{->[H+,e-]} \ce{Fe-NH2-NH2}\\
            &\ce{->[H+,e-][-NH3]} \ce{Fe-NH2}\\
            &\ce{->[H+,e-,N2][-NH3]} \ce{Fe-N2}
        \end{align*}
        \begin{itemize}
            \item This doesn't require as many oxidation states as the Chatt mechanism (which favors iron, which doesn't easily form oxidation states other than $2+$ and $3+$).
            \item We also don't see a nitride or imide intermediate, but we do see a diazene, hydrazido $1-$, and hydrazine adducts.
            \item Enzyme data supports this mechanism.
        \end{itemize}
    \end{itemize}
\end{itemize}



\section{Office Hours (Anderson)}
\begin{itemize}
    \item How much strain is needed for ROMP to proceed?
    \begin{itemize}
        \item Norbornene is a common one.
        \item 4 membered rings.
        \item 7- and 8-membered rings are usually not sufficiently strained.
        \item Cyclopropene does exist, and it does do ROMP, but it polymerizes so quickly that you can't do much with it.
        \begin{itemize}
            \item If it's cold and you throw in a metathesis catalyst, it will probably work.
            \item Recall that it also participates in Diels-Alder reactions.
        \end{itemize}
    \end{itemize}
    \item ROMP does \emph{not} release ethylene.
    \begin{itemize}
        \item Ring strain is the driving force.
        \item Part (d) is a typo; ethylene gas is the common byproduct of \emph{most} of these reactions. We should still show ROMP as is.
    \end{itemize}
    \item How do metal $d$ orbitals enable $2+2$ cycloaddition?
    \begin{itemize}
        \item Butadiene plus ethylene requires ethylene to have filled $\pi^*$ orbitals (these have the appropriate symmetry).
        \begin{itemize}
            \item Diels-Alder $4+2$ cycloaddition is allowed.
        \end{itemize}
        \item The symmetry of the metal orbitals (esp. $d_{z^2}$ and $d_{xy}$) enables $2+2$ cycloaddition.
    \end{itemize}
    \item $\eta$ and $\kappa$ bonding?
    \begin{itemize}
        \item IUPAC likes $\kappa$ more now, but $\eta$ is historical.
        \item $\eta$ indicates a continguous $\pi$ system while $\kappa$ simply indicates denticity.
        \item Figure \ref{fig:CO2-M-bondinga} is $\kappa^2$ as well.
        \item Bidentate ligands bond in a $\kappa^2$ fashion.
        \item Numbers on these as well as $\mu$ should exclusively be superscripts.
        \item A metal bound face-on to a carboxylate would be $\eta^3$, but this is very atypical/nontexistent bonding. Side-on bonding (i.e., to both oxygens) would be $\kappa^2$.
    \end{itemize}
\end{itemize}



\section{Discussion Section}
\begin{itemize}
    \item \marginnote{5/4:}Midterm 2 is 5/25/2021.
    \item Final paper due 5/26/2021; Sophie has asked Dr. Anderson for more information.
    \begin{itemize}
        \item Due at noon.
        \item Sophie will post an example paper.
        \item As the last assignment for this course, you will write a critical review of 1 of 6 availabble papers.
        \item Some of these papers include material outside the scope of organometallic chemistry. Try to focus on topics relevant to this course. The reviews should be no longer than 700 words not including citations or fiugre captions, althouggh additional ccitations are not requried if you only use information presented in the paper or the class. The critique should be written with Times New Roman 12 point font and 0.5 inch margins.
        \item General outline fofr the critiques:
        \begin{itemize}
            \item 1 introduction paragraph.
            \item 2 experimental summary paragraphs.
            \item 3 discussion paragraphs.
            \item 4 critique paragraphs.
            \item Figures.
        \end{itemize}
        \item Sophie is willing to look over assignments (email them to her), but we have to get them to her by the 19th for her to take a look.
        \item They should be professional but not excessively formal.
        \item A more detailed outline will be published later.
        \item A critique isn't necessarily a flaw, but more a suggestion of another experiment they could run or some other conclusion they could have drawn.
    \end{itemize}
    \item Final is 6/3/2021.
    \item Midterm 1:
    \begin{itemize}
        \item Biggest issues:
        \begin{itemize}
            \item Running out of time.
            \item Omitting parts of answers (generally not reading instructions in general). If it says draw orbitals, explain, or predict, you need to do that.
        \end{itemize}
        \item She'll talk to John about making it/the next one shorter.
        \item Syntheses are only one point?
        \item If we would like to know more feedback about our exams, we can reach out with specific questions about questions.
        \item 1a: Reduce with \ce{Na} or \ce{Hg} metal and then throw in \ce{MeI}. \ce{$\frac{1}{2}$ Mn2(CO)10 ->[Na/Hg][-2NaCl] Na[Mn(CO)5] ->[MeI][-NaI]}.
        \item 3a: Like-signed lobes donate (\emph{correct notes}?). You can also show either $d_{x^2-y^2}$ or $d_{z^2}$ for $\sigma$ donation.
        \item 3b: We need charges on the metal in \ce{M-C#O+} resonance structures (-, 0, +).
        \item 3c: Also identify the dominant resonance structure.
        \item 4c: Cp can ring-slip to stabilize the electron count.
        \item 4d: Cone angle starts at metal center and includes specifically the full van der Waals radii of the phosphine R groups at the base fo the cone. Most commonly forgotten: metal center, van der Waals radii.
        \item 4f: \ce{H2CrO4} is \ce{Cr^6+}, $d^0$, while \ce{Cr(CO)4^4-} is \ce{Cr^4-}, $d^{10}$. The difference is different than you expect because they are actually very similar.
        \item 5: First, backcalcculate out spin-state to determine $S=1$ and $S=0$. Second, luckily, they're 4-coordinate, so the simplest solutin is to loook for two geometries that ggive the two spin states. You interconvert the things in solution, giving a mixed magnetic moment. $\mu=2\sqrt{S(S+1)}$.
        \item 6: \ce{Cp-} is a $\pi$ donor. Cyclobutadiene is both (the orbital drawing shows that you have a HOMO $\pi$ donor and a LUMO $\pi$ acceptor).
        \item 7: No extra sites for a better bridge and only bad bridges present (if you draw them you can tell). Whenever you need to compare two ligands, think:
        \begin{enumerate}
            \item Chelate? Neither ligand.
            \item Hard/soft affinity? Both hard.
            \item Strong field? Isocyanide much stronger, more likely all low spin.
            \item Trans effect? All the same ligand.
        \end{enumerate}
    \end{itemize}
\end{itemize}



\section{Lecture 16: Hydrogenation}
\begin{itemize}
    \item \marginnote{5/5:}Nobel prize (2001) to Knowles and Noyori, but could have gone to Jack Halpern.
    \begin{itemize}
        \item A hugely important reaction.
    \end{itemize}
    \item General form:
    \begin{equation*}
        \ce{Y=X + H2 ->[{cat}] H-Y-X-H}
    \end{equation*}
    \item Both reagents are activated by the catalyst.
    \item The chemo, regio, and stereo selectivity has been determined for many of the mechanisms.
    \item History:
    \begin{itemize}
        \item Melvin Calvin reports the first hydrogenation reaction in 1938.
        \begin{itemize}
            \item Iguchi also studied rhodium-amine complexes around the same time.
        \end{itemize}
        \item Wilkinson discovers Wilkinson's catalyst (\ce{Rh(PPh3)3Cl}) in the 1960s and does other foundational research in this area.
        \begin{itemize}
            \item Halpern figures out the mechanism by which Wilkinson's catalyst works.
        \end{itemize}
        \item Knowles and Noyori: Developed asymmetric hydrogenations.
    \end{itemize}
    \item Wilkinson's catalyst:
    \begin{itemize}
        \item Two methods of synthesizing it are given.
    \end{itemize}
    \item Mechanism (for generic catalyzed hydrogenation):
    \begin{figure}[h!]
        \centering
        \begin{subfigure}[b]{0.45\linewidth}
            \centering
            \schemestart
                \chemfig{M(-[1]H)(-[7]H)}
                \arrow{->[\small\chemfig{[:45]=}]}
                \chemfig{M(-[7]H)(-[1]H)(-[2]\phantom{i}-[4,0.4,,,white]=)}
                \arrow[-90]
                \chemfig{M(-[7]H)(-[:30]-[:-30])}
                \arrow{-U>[][*{0.90}\small\chemfig{[:45]-}][][][90]}[180]
                \chemfig{M^{\mathit{n}-2}}
                \arrow{->[*{0}\small\ce{H2}]}[90,1.3]
            \schemestop
            \caption{Dihydride mechanism.}
            \label{fig:mechanism-hydrogenationa}
        \end{subfigure}
        \begin{subfigure}[b]{0.45\linewidth}
            \centering
            \schemestart
                \chemfig{M-H}
                \arrow{->[\small\chemfig{[:45]=}]}
                \chemfig{M(-H)(-[2]\phantom{i}-[4,0.4,,,white]=)}
                \arrow[-90]
                \chemfig{M-[:30]-[:-30]}
                \arrow{->[\small\ce{H2}]}[180]
                \chemfig{M(-[6]H)(-[7]H)(-[:30]-[:-30])}
                \arrow{-U>[][*{0}\small\chemfig{[:45]-}][][][120]}[90]
            \schemestop
            \caption{Monohydride mechanism.}
            \label{fig:mechanism-hydrogenationb}
        \end{subfigure}
        \caption{Hydrogenation mechanisms.}
        \label{fig:mechanism-hydrogenation}
    \end{figure}
    \item Reactivity trends (for Wilkinson's catalyst):
    \begin{itemize}
        \item Rates of hydrogenation for olefins: Cyclic $>$ terminal $>$ doubly substituted $>$ \emph{cis} $>$ \emph{trans} $>$ triply substituted.
        \begin{itemize}
            \item This trend can be utilized for \textbf{chemo selectivity}.
        \end{itemize}
        \item No reactivity with esters or arenes.
    \end{itemize}
    \item \textbf{Chemo selectivity}: The selective hydrogenation of olefins that more readily hydrogenate in a polyolefin compound.
    \begin{itemize}
        \item In other words, chemo selectivity relies on the principle that in a compound with multiple types of double bonds, the ones with higher rates will be selectively hydrogenated first.
    \end{itemize}
    \item Mechanism (for hydrogenation catalyzed by Wilkinson's catalyst):
    \begin{equation*}
        \ce{L3RhCl <=>[{solvent}][L] L2({sol})RhCl <=>[H2] L2({sol})RhClH2 <=>[||] L2(||)RhClH2 -> L2({sol})RhClHEt ->[L][-Et, {sol}] L3RhCl}
    \end{equation*}
    \begin{itemize}
        \item Worked out by Jack Halpern
        \item No scrambling of \ce{H2}/\ce{D2} (you only observe products with two H's added, and products with two D's added; there are no products with an H \emph{and} a D added).
        \item No scrambling with solvent.
        \item The above two observations suggest a dihydride mechanism.
        \item Note that the solvent is either an alcohol or acetonitrile (\ce{MeCN}).
        \item Note that the step with no new reactants is the turnover limiting step (it is drawn with a monodirectional arrow because it is virtually irreversible).
    \end{itemize}
    \item Catalyst optimization:
    \begin{itemize}
        \item Cationic \ce{Rh} complexes react much faster (e.g., \ce{[RhL2H2(sol)2]+}).
        \begin{itemize}
            \item In effect, removing the chloride generates a noncoordinating countercation, which is much more reactive.
        \end{itemize}
        \item If you use iridium instead, it's Crabtree's catalyst: \ce{[Ir(COD)(PCCh3)(py)]+[PF6]-}.
    \end{itemize}
    \item Compares rates of derivatives of Crabtree's and Wilkinson's catalyst at various temperatures in noncoordinating solvents.
    \begin{itemize}
        \item We use noncoordinating solvents such as arenes because solvent-binding is a step in some mechanisms, and thus affects the rate.
        \item Some variations have significantly less reactivity for cyclic olefins than Wilkinson's catalyst.
        \item Conclusion: By making slight changes to the ligands and metal center, you can finely tune the activity of these catalysts to do the type of hydrogenation you want (regioselectivity).
    \end{itemize}
    \item Directed hydrogenation.
    \begin{itemize}
        \item We can use solvent binding to dictate the selectivity of a certain hydrogenation.
        \item If we have a cyclic substrate with a coordinating ligand above one face, the catalyst can coordinate to that ligand and perform the hydrogenation from that face, heavily favoring a certain stereochemistry in the product.
        \item Note that other ligands can affect the extent of substrate coordination.
    \end{itemize}
    \item Other mechanistic details\footnote{An important process is listed, possibly to be returned to later.}.
    \begin{enumerate}
        \item The species you observe (the resting state of the catalyst) may not be on the active catalytic pathway.
        \item To deduce the mechanism, the rates of the individual steps should be determined (recall kinetic competence).
    \end{enumerate}
    \item Noyori's bifunctional catalysis (heavily related to Figure \ref{fig:noyorisCatalystMech}).
    \item $\sigma$-bond metathesis:
    \begin{figure}[H]
        \centering
        \schemestart
            \chemfig{Cp_2Lu-H}
            \arrow{->[\small\chemfig{R-[:30]=[:-30]}]}[,1.5]
            \chemleft{[}
                \chemfig{Cp_2Lu?-[,,,,dashed]H-[6,,,,dashed]C(-[:-60]R)(-[:15]H)-[4,1.12]C?[,,dashed](<[:-110]H)(>:[:-150]H)}
            \chemright{]^\ddagger}
            \arrow[-90]
            \chemfig{Cp_2Lu-[:30]-[:-30]-[:30]R}
            \arrow{->[\small\ce{H2}]}[180]
            \chemleft{[}
                \chemfig{Cp_2Lu?-[,,,,dashed](-[:45]-[::-60]R)-[6,,,,dashed]H-[4,0.92,,,dashed]H?[,,dashed]}
            \chemright{]^\ddagger}
            \arrow{-U>[][*{0.20}\small\chemfig{R-[:30]-[:-30]}][][][135]}[90,1.3]
        \schemestop
        \begin{tikzpicture}[remember picture,overlay]
            \draw [dashed] (-2.4,0.15) -- ++(0.6,0);
        \end{tikzpicture}
        \caption{Hydrogenation by $\sigma$-bond metathesis.}
        \label{fig:sigmaMetathesisHydrogenation}
    \end{figure}
    \begin{itemize}
        \item This mechanism proceeds through two consecutive $\sigma$-bond metatheses.
        \item Note the use of an early transition metal (lutetium), 4-membered transition states, and the open coordination site (refer to Lecture 7 for the more on the characteristics of $\sigma$-bond metathesis).
    \end{itemize}
    \item When catalyzed, the mechanism can be more complicated.
    \begin{itemize}
        \item There can be nearly innumerably many offshoots.
        \item There may only be one on-cycle chain among all of the intermediates. The question is just how to favor this one.
    \end{itemize}
    \item Radical hydrogenation.
    \begin{equation*}
        \ce{L_nM-H + Ph-= -> L_nM* + Ph-*= -> L_nM-CHMePh ->[H2] -> L_nM-H + Ph--}
    \end{equation*}
    \begin{itemize}
        \item This can happen with porphyrins and \ce{HMn(CO)5}.
    \end{itemize}
    \item Noyori's catalyst:
    \begin{figure}[h!]
        \centering
        \begin{subfigure}[b]{0.9\linewidth}
            \centering
            \schemestart
                \chemfig{Ru(-[:150]P)(-[:-150]P)(=[:30]\chemabove{N}{R})(-[:-30]\chembelow{N}{\,R_2}-[,0.5,,,white])(-[2,1.4]\phantom{i}-[4,0.4,,,white]H-H)}
                \arrow{->}
                \chemfig{Ru(-[:150]P)(-[:-150]P-[4,0.5,,,white])(-[:30]\chemabove{N}{\,HR})(-[:-30]\chembelow{N}{\,R_2}-[,0.5,,,white])(-[2]H)}
                \arrow{->[\small\chemfig{-[:30](=[2]O)(-[:-30])}]}[,1.5]
                \chemleft{[}
                    \chemfig{Ru?-N-[2,,,,dashed]H-[:97,,,,dashed]O-[4](-[:150])(-[:-150])(-[:-97,1.2,,,dashed]H?[,,dashed])}
                \chemright{]^\ddagger}
                \arrow{->}[-90]
                \chemfig{Ru(-[:150]P)(-[:-150]P-[4,0.5,,,white])(-[:30]\chemabove{N}{\,HR})(-[:-30]\chembelow{N}{\,R_2})(-[2]O-[:35](-[:65,0.8])(-[:10]))}
                \arrow{-U>[][*{0.north east}\small\chemfig{-[:30](-[2]OH)(-[:-30])}]}[180,4.2]
                \chemfig{Ru(-[:150]P)(-[:-150]P)(=[:30]\chemabove{N}{R})(-[:-30]\chembelow{N}{\,R_2}-[,0.5,,,white])}
                \arrow{->[*{0}\small\ce{H2}]}[90,1.3]
            \schemestop
            \begin{tikzpicture}[remember picture,overlay,xshift=-6.35cm,yshift=-6.6cm]
                \draw (-6.2,6.2) to[bend left=30] ++(0,1);
                \draw (-3.9,6.2) to[bend right=30] ++(0,1);
                \draw (-1.6,6.4) to[bend left=30] ++(0,1);
                \draw (0.7,6.4) to[bend right=30] ++(0,1);
                \draw (3.8,1.5) to[bend left=30] ++(0,1);
                \draw (6.1,1.5) to[bend right=30] ++(0,1);
                \draw (-6.45,2.2) to[bend left=30] ++(0,1);
                \draw (-4.15,2.2) to[bend right=30] ++(0,1);
    
                \draw [dashed] (4.63,7.92) -- ++(0.6,0);
            \end{tikzpicture}
            \caption{Regular.}
            \label{fig:noyorisCatalystMecha}
        \end{subfigure}\\[1em]
        \begin{subfigure}[b]{0.9\linewidth}
            \centering
            \schemestart
                \chemfig{M(-H)(-[2]H)}
                \arrow{->[\small\chemfig{-[:30](=[2]O)-[:-30]}]}[,1.5]
                \chemfig{\charge{45[xshift=1mm,yshift=1mm]=$\ominus$}{M}-[2]H}
                \arrow{0}[,0]\+{1.5em,,1.5em}
                \chemfig{-[:30]\charge{45[xshift=1.5mm,yshift=0.5mm]=$\oplus$}{}(-[2]OH)-[:-30]}
                \arrow
                \chemfig{M}
                \arrow{0}[,0]\+{0.7em,,1.5em}
                \chemfig{-[:30](-[2]OH)-[:-30]}
            \schemestop\\[1em]
            \begin{tikzpicture}
                \draw [-stealth] (4.7,0) -- (4.7,-1) -- node[above]{\small\ce{H2}} (-4.7,-1) -- (-4.7,0);
            \end{tikzpicture}
            \caption{Ionic.}
            \label{fig:noyorisCatalystMechb}
        \end{subfigure}
        \caption{Noyori's catalyst mechanism.}
        \label{fig:noyorisCatalystMech}
    \end{figure}
    \begin{itemize}
        \item Partially reacts via a bifunctional mechanism, but primarily engages in a different mechanistic paradigm called an outer sphere hydrogenation (see Figure \ref{fig:noyorisCatalystMecha}).
        \item In the extreme, we can achieve an ionic pathway (see Figure \ref{fig:noyorisCatalystMechb}).
        \begin{itemize}
            \item An example of a substance that does this is \ce{CpW(CO)2(PPh3)(OCEt2)+}.
        \end{itemize}
    \end{itemize}
    \item Asymmetric catalysis:
    \begin{itemize}
        \item At this point in time, hydrogenation is a highly optimized reaction.
        \begin{itemize}
            \item We can generate catalysts with millions of turnover numbers, very high ee's, etc.
            \item Indeed, hydrogenation is one of the most reliable late-stage steps in drug development or natural product synthesis to define chiral centers.
        \end{itemize}
        \item Chiral phosphines really shine here.
        \item Binap, biphenyl scaffolds with sufficiently large \ce{R} groups, tri-chicken foot phos, ferrocene derivatives (e.g., chiral dppf derivatives), alkyl backbones (e.g., DIOP), etc.
        \item One enantionmer is often bound over the other.
    \end{itemize}
    \item Transfer hydrogenations:
    \begin{figure}[h!]
        \centering
        \begin{tikzpicture}
            \node (M1) at (0,0) {\chemfig{M}};
            \node (M2) at (0,4) {\chemfig{M(-H)(-[2]H)}};
            \node (A1) at (-2.7,0) {\chemfig{R_1-[:30](-[2]OH)-[:-30]}};
            \node (A2) at (-2.7,4) {\chemfig{R_1-[:30](=[2]O)-[:-30]}};
            \node (K1) at (2.4,4) {\chemfig{R_2-[:30](=[2]O)-[:-30]}};
            \node (K2) at (2.4,0) {\chemfig{R_2-[:30](-[2]OH)-[:-30]}};
    
            \draw [blx,thick,-stealth] (M1) to[bend left=40] (M2);
            \draw [blx,thick,-stealth] (M2) to[bend left=40] (M1);
            \draw [blx,thick,-stealth] (A1) to[bend right=60] (A2);
            \draw [blx,thick,-stealth] (K1.south west) to[bend right=53] ([xshift=-4mm,yshift=2mm]K2);
        \end{tikzpicture}
        \caption{Transfer hydrogenation mechanism.}
        \label{fig:transferHydrogenations}
    \end{figure}
    \begin{itemize}
        \item Use these if you don't want to use hydrogen gas, e.g., because it's flammable.
        \item Basically, we use a transition metal catalyst to transfer two hydrogens (e.g., from an alcohol to a ketone) by means of gaining them to become a dihydride and losing them to return to being just a metal center.
        \item As in any hydrogenation reaction, these are all reversible reactions. As such, since we can't use excess hydrogen to push the reaction, we must rely on the electron richness of the alcohol (electron rich alcohols dehydrogenate more easily).
        \item The equilibrium can be pushed with Le Ch\^{a}telier's principle, but we mainly have to consider thermodynamics here (the thermodynamics must be favorable).
        \begin{itemize}
            \item For example, if $\ce{R2}=\ce{Ar}$, then we will transform the aryl ketone into an alcohol, but not in reverse (because the aryl is electron deficient and thus its alcohol does not dehydrogenate easily).
        \end{itemize}
    \end{itemize}
    \item Halpern's contributions to asymmetric catalysis:
    \begin{itemize}
        \item Reviews some papers.
        \item Increasing hydrogen pressure decreases stereoselectivity, which is often kinetically driven.
    \end{itemize}
\end{itemize}



\section{Lecture 17: Hydroformylation and Carbonylation}
\begin{itemize}
    \item \marginnote{5/7:}We will now talk about inserting carbonyls.
    \item Grandfather reaction: Monsanto Acetic Acid Synthesis.
    \begin{itemize}
        \item Used for many years, but replaced within the last few years.
        \item Huge scale: Produces approximately 17 billion pounds of acetic acid per year.
        \item About 80\% of all acetic acid we use is generated by this process.
        \item Several interations historically:
        \begin{itemize}
            \item BASF (1965) uses cobalt and iodide.
            \item Monsanto (1970) uses rhodium and iodide.
            \item BP "Cativa" (1996) uses iridium and iodide.
        \end{itemize}
    \end{itemize}
    \item A big theme in catalysis right now is using first row metals instead of second- and third-row metals because they're cheaper and more abundant.
    \begin{itemize}
        \item However, it makes sense to use iridium here: 5 million dollars of iridium can run in a reactor for decades.
    \end{itemize}
    \item General form:
    \begin{equation*}
        \ce{CH3OH + CO ->[Rh(CO)2I2-][{30-$\SI{40}{atm}$, $\SI{180}{\celsius}$}] Me-COOH}
    \end{equation*}
    \item Mechanism:
    % \begin{figure}[h!]
    %     \centering
    %     \begin{tikzpicture}
    %         \node (1) at (90:3) {\chemfig{Rh(-[1]I)(-[3]OC)(-[5]OC)(-[7]I)}};
    %         \node (2) at (-30:3) {\chemfig{Rh(-I)(>:[1]I)(-[2]Me)(-[4]OC)(<[5]OC)(-[6]I)}}
    %             ([xshift=2mm,yshift=2mm]2.north) edge [blx,semithick,stealth-,bend right=30] ([yshift=-3mm]1.east)
    %         ;
    %         \node (3) at ([xshift=-4mm]-150:3) {\chemfig{Rh(-I)(>:[1]I)(-[2]CO)(-[4](=[:120]O)-[:-120])(<[5]OC)(-[6]I)}}
    %             edge [blx,semithick,stealth-,bend right=32] node[black,below]{\small\ce{CO}} (2)
    %             ([xshift=2mm,yshift=2mm]3.north)edge [blx,semithick,-stealth,bend left=30] ([yshift=-3mm]1.west)
    %         ;
    
    %         \node (4) at (135:5) {\chemfig{-[:30](=[2]O)-[:-30]I}}
    %             edge [blx,semithick,stealth-,out=-90,in=80] (170:2.78)
    %         ;
    %         \node (5) at (90:5) {\ce{HI}}
    %             edge [blx,semithick,stealth-,bend right=20] ([xshift=4.5mm,yshift=4.5mm]4.center)
    %         ;
    %         \node (6) at (45:5) {\ce{MeI}}
    %             edge [blx,semithick,stealth-,bend right=20] (5)
    %             edge [blx,semithick,out=-90,in=100] (10:2.79)
    %         ;
    
    %         \node (7) at (-4.5,5) {\ce{H2O}};
    %         \node (8) at (-1,7) {\chemfig{-[:30](=[2]O)-[:-30]OH}}
    %             edge [blx,semithick,stealth-,bend left=50,looseness=1.33] (7)
    %         ;
    %         \node (9) at (1,6.3) {\ce{MeOH}};
    %         \node (10) at (4.5,5) {\ce{H2O}}
    %             edge [blx,semithick,stealth-,bend left=50,looseness=1.23] (9)
    %         ;
    %     \end{tikzpicture}
    %     \caption{Carbonylation mechanism.}
    %     \label{fig:mechanism-carbonylation}
    % \end{figure}
    \begin{figure}[h!]
        \centering
        \begin{tikzpicture}
            \node (1) at (90:3) {
                \chemleft{[}
                    \chemfig{Rh(-[1]I)(-[3]OC)(-[5]OC)(-[7]I)}
                \chemright{]^-}
            };
            \node (2) at ([xshift=4mm]-30:3) {
                \chemleft{[}
                    \chemfig{Rh(-I)(>:[1]I)(-[2]Me)(-[4]OC)(<[5]OC)(-[6]I)}
                \chemright{]^-}
            }
                ([xshift=1mm,yshift=2mm]2.north) edge [blx,semithick,stealth-,bend right=30] ([yshift=-3mm]1.east)
            ;
            \node (3) at ([xshift=-8mm]-150:3) {
                \chemleft{[}
                    \chemfig{Rh(-I)(>:[1]I)(-[2]CO)(-[4](=[:120]O)-[:-120])(<[5]OC)(-[6]I)}
                \chemright{]^-}
            }
                edge [blx,semithick,stealth-,bend right=32] node[black,below]{\small\ce{CO}} (2)
                ([xshift=1mm,yshift=2mm]3.north)edge [blx,semithick,-stealth,bend left=30] ([yshift=-3mm]1.west)
            ;
    
            \node (4) at (135:5) {\chemfig{-[:30](=[2]O)-[:-30]I}}
                edge [blx,semithick,stealth-,out=-90,in=80] (170:3.28)
            ;
            \node (5) at (90:5) {\ce{HI}}
                edge [blx,semithick,stealth-,bend right=20] ([xshift=4.5mm,yshift=4.5mm]4.center)
            ;
            \node (6) at (45:5) {\ce{MeI}}
                edge [blx,semithick,stealth-,bend right=20] (5)
                edge [blx,semithick,out=-90,in=100] (10:3.123)
            ;
    
            \node (7) at (-4.5,5) {\ce{H2O}};
            \node (8) at (-1,7) {\chemfig{-[:30](=[2]O)-[:-30]OH}}
                edge [blx,semithick,stealth-,bend left=50,looseness=1.33] (7)
            ;
            \node (9) at (1,6.3) {\ce{MeOH}};
            \node (10) at (4.5,5) {\ce{H2O}}
                edge [blx,semithick,stealth-,bend left=50,looseness=1.23] (9)
            ;
        \end{tikzpicture}
        \caption{Carbonylation mechanism.}
        \label{fig:mechanism-carbonylation}
    \end{figure}
    \begin{itemize}
        \item There are cocatalytic reactions that enable this reactivity.
        \item Trace amounts of \ce{HI} and \ce{H2O} facilitate the overall reaction.
        \begin{itemize}
            \item Inputs are \ce{MeOH} and \ce{CO}; output is acetic acid.
            \item Everything else is catalytic, or generated in situ.
        \end{itemize}
        \item Rate law: $\text{Rate}=k\ce{[Rh][MeI]}$.
        \item Side reactions:
        \begin{itemize}
            \item If the first intermediate is not immediately trapped by \ce{CO}, it can insert to form a 5-coordinate intermediate that dimerizes in an inactive off-cycle.
            \item This side reaction is not \emph{necessarily} deleterious, but it can be.
        \end{itemize}
    \end{itemize}
    \item This complex has been heavily studied, giving us great insight into each step.
    \item Oxidative addition.
    \begin{itemize}
        \item S\textsubscript{N}2-type reaction.
        \item An attack of the initial catalyst by acetyl iodide can lead to the second intermediate, directly.
        \item The 5-coordinate intermediate in the S\textsubscript{N}2 process can be attacked by methanol or acetic acid, giving you a methyl ester and acetic anhydride, respectively.
        \item Undesirable side reactions:
        \begin{itemize}
            \item The catalyst can react with \ce{HI}, forming a \ce{RhI3} precipitate and an \ce{RhI4(CO)2-} inactive ion; both steps serve to effectively remove rhodium from the catalytic cycle (this is a big problem). However, this can be fixed by adding \ce{H2O} (about 10\%). This increases the solubility of \ce{RhI3} and turns on the water-gas shift reaction \ce{CO + H2O -> CO2 + H2}; the \ce{H2} product of the latter reaction serves as a reductant that transforms \ce{Rh^{III} -> Rh^I}.
            \item Another undesirable side reaction is drawn out and its solution discussed. However\dots
            \item Problem: \ce{H2O} is hard to separate from acetic acid (they have similar boiling points and they're miscible). Thus, you would have distill, but that's expensive and time-consuming. Additionally, acetic acid is corrosive. Therefore, this is not a great solution.
            \item However, you can add the promotor \ce{LiI} or \ce{LiOAc}. It's not clear exactly what the promotor does, but one possible explanation is that you can get to a triiodide dianion that is nucleophilic to the point that it can speed up oxidative addition enough to promote the productive pathway.
            \item Another possibility is that iodide binding at other steps can forward the productive pathway.
        \end{itemize}
    \end{itemize}
    \item You can intercept acetyl intermediates to foster another catalytic process:
    \item Kovach-Eastman Acetic Anhydride Process.
    \begin{itemize}
        \item Invented in 1983.
        \item Produces 800 million pounds of acetic anhydride annually.
    \end{itemize}
    \item Mechanism:
    \begin{itemize}
        \item The only difference between this and the Monsanto acetic acid process (Figure \ref{fig:mechanism-carbonylation}) is that in the nonorganometallic cycle, we add acetic acid and get acetic anhydride, and then add methyl ester and get acetic acid.
        \item Inputs are \ce{CO} and \ce{Me-COOMe}.
    \end{itemize}
    \item Cativa process:
    \begin{itemize}
        \item Invented by BP.
        \item 5 times more active than rhodium-catalysed.
        \item Uses iridium with a ruthenium promotor.
        \item Iridium is in the same group as iridium but is faster for oxidative addition.
        \item The rate law is complicated, depending on \ce{[CO]}, \ce{[H2O]}, \ce{[MeOAc]}, \ce{[MeI]} (which you don't actually want in there), \ce{[Ru]}, and \ce{[Ir]}.
        \begin{itemize}
            \item This process is complex and nonlinear.
        \end{itemize}
        \item The main species in solution is \ce{Ir(CH3)(CO)2I3-}.
        \item The turnover limiting step here is insertion since oxidative addition is so much faster (approximately 150 times faster). However, insertion is $\num{e5}$ times slower for iridium.
        \begin{itemize}
            \item Simplistic explanation: There are even stronger bonds for iridium then for rhodium. This favors bond formation (i.e., in oxidative addition), but not bond breaking (i.e., in insertion).
        \end{itemize}
        \item However, we can fix the slower insertion rate with promoters, namely \ce{[Ru(CO)3I2]2}.
        \begin{figure}[h!]
            \centering
            \schemestart
                \chemleft{[}
                    \chemfig{Ir(>:[:25]I)(-[2]Me)(>:[:155]OC)(<[:-155]OC)(-[6]I)(<[:-25]I)}
                \chemright{]^-}
                \arrow(1--2){->[\small\ce{CO}][\footnotesize slow]}
                \chemfig{Ir(>:[:25]I)(-[2](=[:150]O)-[:30])(>:[:155]OC)(<[:-155]OC)(-[6]I)(<[:-25]I)}
                \arrow(@1--3){<=>[*{0}\small\ce{$\frac{1}{2}$ [Ru(CO)3I2]2, CO}][*{0}\small\ce{Ru(CO)3I3-}]}[-90,1.3]
                \chemfig{Ir(>:[:25]I)(-[2]Me)(>:[:155]OC)(<[:-155]OC)(-[6]CO)(<[:-25]I)}
                \arrow(@3--4){->[\small\ce{CO}][\footnotesize fast]}[,1.18]
                \chemfig{Ir(>:[:25]I)(-[2](=[:150]O)-[:30])(>:[:155]OC)(<[:-155]OC)(-[6]I)(<[:-25]CO)}
                \arrow(@4--@2){->[*{0.180}\small\ce{Ru(CO)3I3-}][*{0.0}\small\ce{-CO}]}
            \schemestop
            \caption{Promoting the Cativa process.}
            \label{fig:cativaPromoters}
        \end{figure}
        \item $\frac{k_\text{fast}}{k_\text{slow}}\approx 700$, so an almost 3 orders of magnitude gain.
    \end{itemize}
    \item The overall mechanism for the Cativa process is drawn out.
    \item Hydroformylation.
    \item General form.
    \begin{figure}[h!]
        \centering
        \schemestart
            \chemfig{R-[:30]=[:-30]}
            \arrow{->[\small\ce{\color{rex}H2\color{black}, CO}][\footnotesize 120-$\SI{180}{\celsius}$, 200-$\SI{300}{atm}$]}[,2.4]
            \chemfig{R-[:30](-[2]\textcolor{rex}{H})-[:-30]-[:30](=[2]O)-[:-30]\textcolor{rex}{H}}
            \arrow{0}[,0]\+{,,2.3em}
            \chemfig{R-[:30](-[2](=[:30]O)-[:150]\textcolor{rex}{H})-[:-30]-[:30]\textcolor{rex}{H}}
        \schemestop
        \caption{The general form of hydroformylation.}
        \label{fig:hydroformylation}
    \end{figure}
    \begin{itemize}
        \item The ratio of the first to the second product is 3-$4:1$.
    \end{itemize}
    \item Typical catalysts:
    \begin{itemize}
        \item \ce{HCo(CO)4} (synthesized from \ce{Co2(CO)8} as a precatalyst and \ce{H2}).
        \item \ce{HCo(CO)3(PR3)}.
        \item \ce{HRh(CO)2(PR3)2}.
    \end{itemize}
    \item On the gas used in this process (\ce{CO} and \ce{H2} in a $1:1$ ratio):
    \begin{itemize}
        \item Called synthesis (or syn) gas.
        \item Released when coal is heated.
        \item Composed of \ce{CO} and \ce{H2} in a $1:1$ ratio.
    \end{itemize}
    \item The rate at which the \ce{HCo(CO)4} catalyst acts is given by $\text{Rate}=k\ce{[H2][CO]^{-1}}$.
    \begin{itemize}
        \item Thus, we can tweak the rate by adjusting the relative concentrations of the gas.
        \item Adding \ce{H2} will increase the rate of reaction, and adding \ce{CO} will decrease the rate of reaction.
    \end{itemize}
    \item Mechanism:
    \begin{figure}[h!]
        \centering
        \schemestart
            \chemfig{Co(-CO)(-[2]H)(>:[:160]OC)(<[:-150]OC)(-[6]CO)}
            \arrow{<=>[][*{0}\small\ce{CO}]}[-90]
            \chemfig{Co(-CO)(-[2]H)(-[4]OC)(-[6]CO)}
            \arrow(--b){<=>[\small\chemfig{-[:30]=_[:-30]}]}[,1.3]
            \chemfig{Co(-\phantom{i}-[2,0.4,,,white]=^[6]-[:-30])(-[2]H)(>:[:160]OC)(<[:-150]OC)(-[6]CO)}
            \arrow{<=>[*{0}\footnotesize\ce{$1,2$}]}[-90,1.1]
            \chemfig{Co(-Pr)(-[2]CO)(-[4]OC)(-[6]CO)}
            \arrow{<=>[*{0}\small\ce{CO}]}[-90,1.1]
            \chemfig{Co(-CO)(-[2]Pr)(>:[:160]OC)(<[:-150]OC)(-[6]CO)}
            \arrow{<=>}[180,1.3]
            \chemfig{Co(-(=[:-60]O)(-[:60]Pr))(-[2]CO)(-[4]OC)(-[6]CO)}
            \arrow{->[*{0}\small\ce{H2}]}[90]
            \chemfig{Co(>:[:25]CO)(-[2](=[:150]O)-[:30]Pr)(>:[:155]H)(<[:-155]H)(-[6]CO)(<[:-25]CO)}
            \arrow{-U>[][*{0}\small\chemfig{-[:30]-[:-30]-[:30](=[2]O)-[:-30]H}][][][120]}[90]
            \arrow(@b--){<=>[\footnotesize\ce{$2,1$}]}
            \chemfig{Co(-(-[:60])-[:-60])(-[2]CO)(-[4]OC)(-[6]CO)}
            \arrow{->[$\longrightarrow$][$\longrightarrow$]}[-90]
            \chemfig{-[:30](-[2](=[:150]O)-[:30]H)-[:-30]}
        \schemestop
        \caption{Hydroformylation mechanism.}
        \label{fig:mechanism-hydroformylation}
    \end{figure}
    \begin{itemize}
        \item The active catalyst is a square planar, $d^8$, $16\,\e[-]$ complex.
        \item The first intermediate is the key intermediate to determine selectivity because it can insert in a $2,1$ or a $1,2$ fashion.
        \item The third intermediate along the $1,2$ branch can be trapped by \ce{CO}.
        \begin{itemize}
            \item The off-cycle intermediate can be isolated. Thus, it's a kind of resting state.
        \end{itemize}
        \item A similar process (as indicated by the triple arrow) can be used to get from the $2,1$ branch to the final branched product.
        \item Control of branching:
        \begin{itemize}
            \item It's complicated and not entirely clear.
            \item However, we are aware of \textbf{chain walking}.
        \end{itemize}
    \end{itemize}
    \item \textbf{Chain walking}: The transition from the $1,2$-inserted intermediate and the $2,1$-inserted intermediate and vice versa.
    \begin{itemize}
        \item An equilibrium process.
    \end{itemize}
    \item How to control selectivity.
    \begin{itemize}
        \item Switching \ce{HCo(CO)3(PR3)} increases the rate and selectivity.
        \item With a special phosphine, the selectivity of linear to branched is $8:1$.
        \item This also enables the hydrogenation of aldehydes to alcohols.
        \begin{itemize}
            \item An important industrial application of this is transforming (in one step) internal olefins into terminal aldehydes and terminal alcohols.
        \end{itemize}
    \end{itemize}
    \item First-row metals hydroformylate, but heavier ones do it better.
    \item Rh-catalyzed hydroformylation:
    \begin{itemize}
        \item Uses \ce{Rh(CO)2(PR3)2}.
        \item Linear-to-branched ratio of $11:1$.
        \item Milder conditions (5-$\SI{10}{atm}$ \ce{CO} / \ce{H2} and $\SI{90}{\celsius}$).
        \item The mechanism is similar to that of \ce{Co}.
        \item We make the catalyst water soluble with bulky, heavy phosphines (such as tris(phenyl sulfate)phos) to aid in separation.
        \item We are allowed to have two phosphines on rhodium because rhodium has a larger atomic radius. The larger atomic radius of rhodium also enables the use of chelating phosphines, such as\dots
        \begin{itemize}
            \item dppe, dppp, dppb, bis(diphenylphosphino)ferrocene (dppf), DPE Phos, Xant phos, and BISBI (the best).
        \end{itemize}
        \item Selectivity of chelating phosphines depends on bite angle.
        \begin{itemize}
            \item $\ce{PPh3}\to 9:1$ ratio.
            \item $\ce{dppe}\to 4:1$ ratio.
            \item $\ce{BISBI}\to 30:1$ ratio.
        \end{itemize}
        \item Hypothesis: Larger bite angles will favor \ce{PR3} in the equatorial plane.
        \begin{itemize}
            \item Favoring 5-coordinate intermediates over 4-coordinate intermediates.
        \end{itemize}
    \end{itemize}
\end{itemize}




\end{document}