\documentclass[../psets.tex]{subfiles}

\pagestyle{main}
\renewcommand{\leftmark}{Homework 2}

\begin{document}




\begin{enumerate}
    \setenumerate[2]{label={\alph*)}}
    \item \marginnote{4/19:}For the following pairs of complexes, predict which should exchange more rapidly and justify your answer.
    \begin{enumerate}
        \item \ce{[Cr(CN)6]^4-} or \ce{[Cr(CN)6]^3-}.
        \begin{proof}[Answer]
            \ce{[Cr(CN)6]^4-} should exchange more rapidly. This is due to the lesser magnitude oxidation state on the chromium.
        \end{proof}
        \item \ce{[Cr(H2O)6]^3+} or \ce{[Ni(H2O)6]^2+}.
        \begin{proof}[Answer]
            \ce{[Ni(H2O)6]^2+} should exchange more rapidly. This is because the nickel's $d^8$ configuration will be more labile than the chromium's $d^3$ configuration, despite the nickel's smaller size. The instability of the $d^8$ configuration comes from its partially filled antibonding orbitals, and the stability of the $d^3$ configuration comes from its half-filled $t_{2g}$ set.
        \end{proof}
        \item \ce{[Co(NH3)6]^3+} or \ce{[Co(NH3)6]^2+}.
        \begin{proof}[Answer]
            \ce{[Co(NH3)6]^2+} should exchange more rapidly. This is due to the lesser magnitude oxidation state on the cobalt.
        \end{proof}
        \item \ce{[Cu(NH3)4]^2+} or \ce{[Cu(en)2]^2+}.
        \begin{proof}[Answer]
            \ce{[Cu(NH3)4]^2+} should exchange more rapidly. This is due to the chelate effect, which \ce{[Cu(en)2]^2+} has but \ce{[Cu(NH3)4]^3+} doesn't. Indeed, even though the copper in \ce{[Cu(NH3)4]^3+} has a higher oxidation state, the extra stability from both entropic considerations and, to a lesser extent, effective concentration in the chelate effect slows the exchange rate of \ce{[Cu(en)2]^2+} even more.
        \end{proof}
        \item \ce{[PtCl3NH3]-} or \ce{[Pt(CN)3NH3]-}.
        \begin{proof}[Answer]
            \ce{[Pt(CN)3NH3]-} should exchange more rapidly. This is because \ce{CN-} is a stronger trans-directing ligand than \ce{Cl-}, so the \ce{NH3} opposite it should exchange much more rapidly.
        \end{proof}
    \end{enumerate}
    \item Consider two generic \ce{ML6} and \ce{ML4} complexes.
    \begin{enumerate}
        \item If you had to guess at a mechanism for the substitution of a ligand \ce{L$'$} in these complexes, which would be dissociative and which would be associative?
        \begin{proof}[Answer]
            \ce{ML6} ligand substitution would probably be dissociative, and \ce{ML4} ligand substitution would probably be associative. This is because there is greater steric clashing in \ce{ML6} which makes it more favorable for a ligand to dissociate, and sterics hinder a ligand \ce{L$'$} from attacking \ce{ML6} to a greater extent than \ce{ML4}.
        \end{proof}
        \item In the substitution reaction of the \ce{ML6} complex with \ce{L$'$}, you do detailed kinetic studies that suggest that the rate has a first order dependence on \ce{[ML6]} and \ce{[L$'$]}. Is this consistent with your answer from part (a)? What dependence should be observed for associative/dissociative mechanisms?
        \begin{proof}[Answer]
            A first order dependence on \ce{[ML6]} and \ce{[L$'$]} suggests an associative mechanism. This is not consistent with my answer from part (a). Indeed, if a dissociative mechanism for \ce{ML6} ligand substitution were present, we would expect to see pseudo-first order conditions in \ce{[ML6]}, alone, upon swamping the reaction with \ce{[L$'$]}.
        \end{proof}
    \end{enumerate}
    \item In the electron transfer reaction between \ce{[Co(NH3)5X]^2+} and \ce{[Cr(H2O)5]^2+}, the rate of the reaction depends strongly on the \ce{X-} ligand. When $\ce{X}=\ce{Cl-}$, the rate is $\num{60000}$ and when $\ce{X}=\ce{H2O}$ (note that the complex is $3+$ in this case) a rate of $0.5$ is observed. Rationalize this difference in rate.
    \begin{proof}[Answer]
        As ligand size and charge increase, so does reaction rate (for an inner sphere mechanism, which this likely is). Ligand size helps because the extra diffusion of the electron cloud literally generates more overlap area. Charge also helps because of its electrostatic influence (it's easier to transfer an electron if there are more electrons at play to assist in the process).
    \end{proof}
    \item In class we discussed the Creutz-Taube ion, which is a classic example of a mixed-valence complex. The degree of mixed-valency will be directly proportional to the rate of inter-valence charge transfer.
    \begin{enumerate}
        \item Mixed valency is more difficult to engender in first row complexes. For instance, the \ce{Fe}-analogue of the Creutz-Taube ion \ce{[(NH3)5Fe-{pyz}-Fe(NH3)5]^5+} is charge localized. Why might this be the case in this \ce{Fe} example?
        \begin{proof}[Answer]
            Although iron and ruthenium share a group (and thus an electron configuration and many properties), the iron ion is significantly smaller than the ruthenium ion. Thus, its less diffuse electron cloud hinders its ability to delocalize charge, meaning that it is both more reticent to give up and accept an electron. Additionally, since it can have a high spin configuration, there is more nuclear rearrangement to undergo during each transfer.
        \end{proof}
        \item Predict which of the following pairs will be most delocalized (fastest IVCT) and least delocalized (slowest IVCT) for the following Creutz-Taube ion analogues (all ligands are the same as in part (a)).
        \begin{enumerate}
            \item \ce{[Cr2]^5+} or \ce{[Cr2]^7+}.
            \begin{proof}[Answer]
                \ce{[Cr2]^7+} will be more delocalized. Despite the higher magnitude oxidation state, these chromiums have at most 3 $d$-electrons, whereas the other pair can go up to 4. For the others, going up to 4 $d$-electrons means having either an ${e_g}^*$ electron or forcing spin pairing; either outcome would increase the nuclear reorganization needed between each transition and decrease the IVCT.
            \end{proof}
            \item \ce{[Co2]^5+} or \ce{[Fe2]^5+} (all ions are low-spin).
            \begin{proof}[Answer]
                \ce{[Fe2]^5+} will be more delocalized. Even though it has fewer electrons than \ce{[Co2]^5+}, it has a larger radius, so it can more easily share them.
            \end{proof}
        \end{enumerate}
    \end{enumerate}
    \item Predict the reactivity of \ce{Ir(CO)(Cl)(PPh3)2} with the following reagents.
    \begin{enumerate}
        \item \ce{MeI}.
        \begin{proof}[Answer]
            This reaction will proceed as follows (S\textsubscript{N}2 mechanism).
            \begin{center}
                \schemestart
                    \chemfig{Ir^{I}(-Cl)(-[1]PPh_3)(-[4]OC)(-[5]Ph_3P)}
                    \arrow{->[MeI]}
                    \chemfig{Ir^{III}(-Cl)(-[1]PPh_3)(-[2]Me)(-[4]OC)(-[5]Ph_3P)(-[6]I)}
                \schemestop
            \end{center}
        \end{proof}
        \vspace{-0.7em}
        \item \ce{O2}.
        \begin{proof}[Answer]
            This reaction will proceed as follows (concerted mechanism).
            \begin{center}
                \schemestart
                    \chemfig{Ir^{I}(-Cl)(-[1]PPh_3)(-[4]OC)(-[5]Ph_3P)}
                    \arrow{->[{O\textsubscript{2}}]}
                    \chemfig{Ir^{III}(-PPh_3)(-[1]O?)(-[2]O?)(-[4]OC)(-[5]Ph_3P)(-[6]Cl)}
                \schemestop
            \end{center}
        \end{proof}
        \vspace{-0.7em}
        \item \ce{H2}.
        \begin{proof}[Answer]
            This reaction will proceed as follows (concerted mechanism).
            \begin{center}
                \schemestart
                    \chemfig{Ir^{I}(-Cl)(-[1]PPh_3)(-[4]OC)(-[5]Ph_3P)}
                    \arrow{->[{H\textsubscript{2}}]}
                    \chemfig{Ir^{III}(-H)(-[1]PPh_3)(-[2]H)(-[4]Cl)(-[5]Ph_3P)(-[6]\chembelow{C}{O})}
                \schemestop
            \end{center}
        \end{proof}
    \end{enumerate}
    \newpage
    \item Metal alkyl species have historically been difficult to isolate.
    \begin{enumerate}
        \item Why?
        \begin{proof}[Answer]
            Alkyl species are often very stable, so it's hard to get them to react in such a way that you can separate them from whatever else they're mixed up in. Additionally, some of the first alkyls to be identified were highly reactive, e.g., the extremely flammable \ce{ZnEt2} (isolated by reacting zinc metal with two equivalents of ethyl iodide) and the reactive Grignard reagents.
        \end{proof}
        \item Name three alkyl (or aryl, etc.) groups that should be stable.
        \begin{proof}[Answer]
            Alkyl groups without $\beta$-hydrogens are generally more stable. Two examples of these are fluoroalkyls and the methyl group. However, some alkyl groups have stable $\beta$-hydrogens for other reasons, such as ring strain. One example of such a group is norbornyl.
        \end{proof}
        \item In class, I said that metallacycles are more stable than other alkyls. Why? Would you expect a larger or smaller metallacycle to be more stable?
        \begin{proof}[Answer]
            Metallacycles are less reactive because their $\beta$-hydrogens are less reactive due to ring strain. As such, a smaller metallacycle would probably be more stable with respect to $\beta$-hydride elimination because it is less likely to bend into a favorable position. Additionally, the ideal size for metallacycle stability is a 5- or 6-membered ring, which is fairly small.
        \end{proof}
    \end{enumerate}
    \item Metal olefin complexes can react via insertion, electrophilic attack, and nucleophilic attack. For each of these transformations, indicate the change in electron count, coordination number, and oxidation state on the metal center.
    \begin{proof}[Answer]
        % Insertion: Electron count decreases by 2 (we lose an X-type ligand and an L-type ligand becomes an X-type ligand), coordination number decreases by 1 (an X-type ligand migrates away from the metal center), and the oxidation state does not change (we remove one X-type ligand and create a new one, so the net effect is null).\par
        % Electrophilic attack: Electron count decreases by 1 (L-type ligand to X-type ligand), coordination number does not change (we are only modifying a ligand), and the oxidation state increases by 2.\par
        % Nucleophilic attack: Electron count does not change (a 2-electron donor becomes a 1-electron donor, but the compound's overall charge also decreases by 1), coordination number does not change (we are only modifying a ligand), and the oxidation state does not change (an L-type ligand becomes an X-type ligand, but the compound's overall charge also decreases by 1).

        Insertion: $\e[-]$ count decreases by 2, C.N. decreases by 1, and O.S. does not change.\par
        Electrophilic attack: $\e[-]$ count decreases by 2, C.N. does not change, and O.S. does not change.\par
        Nucleophilic attack: $\e[-]$ count does not change, C.N. does not change, and O.S. does not change.
    \end{proof}
    \item Transition metal carbene complexes have been known for many years, but there are two distinct types: Fischer Carbenes and Schrock type Alkylidenes. Differentiate these ligand types using diagrams and resonance structures. How should their reactivity change?
    \begin{proof}[Answer]
        $\color{white}hi$
        \begin{figure}[H]
            \centering
            \begin{subfigure}[b]{0.35\linewidth}
                \centering
                \schemestart[-90]
                    \chemfig{L_{\emph{n}}M=C(-[1]XR_1)(-[7]R_2)}
                    \arrow(.south--.north){<->}
                    \chemfig{L_{\emph{n}}\charge{90[yshift=1mm]=$\ominus$}{M}-C(=[1]\charge{90[yshift=1mm]=$\oplus$}{X}R_1)(-[7]R_2)}
                \schemestop
                \caption{Fischer carbenes.}
                \label{fig:carbeneResonancea}
            \end{subfigure}
            \begin{subfigure}[b]{0.35\linewidth}
                \centering
                \schemestart[-90]
                    \chemfig{L_{\emph{n}}M=C(-[1]R_1)(-[7]R_2)}
                    \arrow(.south--.north){<->}
                    \chemfig{L_{\emph{n}}\charge{90[yshift=1mm]=$\oplus$}{M}-\charge{90[yshift=1mm]=$\ominus$}{C}(-[1]R_1)(-[7]R_2)}
                \schemestop
                \caption{Schrock alkylidenes.}
                \label{fig:carbeneResonanceb}
            \end{subfigure}
            \caption{Carbene regime resonance structures.}
            \label{fig:carbeneResonance}
        \end{figure}
        Fischer carbenes are electrophilic at C, their heteroatom is stabilized, and they typically react with late metals. Schrock alkylidenes are nucleophilic at C, have no heteroatom, and typically react with early metals.
    \end{proof}
    \newpage
    \item Rank the following complexes, from highest to lowest, in terms of their \ce{CO} stretching frequency and rationalize: \ce{Pt(CO)4^2+}, \ce{Ni(CO)4^2+}, \ce{PtCl2(CO)2}, \ce{Fe(CO)5^2-}, \ce{Re(CO)4^3-}, and \ce{Mo(CO)6}.
    \begin{proof}[Answer]
        % \ce{Ni(CO)4^2+}: Ni should backbond less strongly as a first row T
        % \ce{Pt(CO)4^2+}: Pt should backbond more strongly as a third row TM.
        % \ce{PtCl2(CO)2}: Pt should backbond more strongly as a third row TM. Same oxidation state but more $\sigma$ donation from the chlorides means higher backbonding than \ce{Pt(CO)4^2+}.
        % \ce{Mo(CO)6}: Mo should backbond intermediately as a second row TM.
        % \ce{Fe(CO)5^2-}: Fe should backbond less strongly as a first row TM.
        % \ce{Re(CO)4^3-}: Re should backbond more strongly as a third row TM. Earlier TM, same $d$ count, lower TM than \ce{Fe(CO)5^2-}.

        \begin{equation*}
            \ce{Pt(CO)4^2+}
            > \ce{PtCl2(CO)2}
            > \ce{Ni(CO)4^2+}
            > \ce{Mo(CO)6}
            > \ce{Fe(CO)5^2-}
            > \ce{Re(CO)4^3-}
        \end{equation*}
        $\nu_{\ce{CO}}$ increases as backbonding decreases. As such, to rank the complexes by their \ce{CO} stretching frequency, it will suffice to rank them by the strength of their backbonding. Note that backbonding decreases as $d$ count increases, and as we move up and to the right in the transition metals on the periodic table. \ce{L -> M} $\sigma$ donation can also increase metal backbonding.\par\medskip
        We will show that every compound in the list exhibits less backbonding than the one after it.\par\smallskip
        To begin, the only difference between \ce{Pt(CO)4^2+} and \ce{PtCl2(CO)2} is the exchange of two carbonyl ligands for chlorides. Since chlorides are stronger $\sigma$ donors, they will make their \ce{Pt} center slightly more electron rich, causing it to engage in slightly more backbonding.\par
        To compare \ce{PtCl2(CO)2} and \ce{Ni(CO)4^2+}, we will focus on the differences in the metal center and neglect the ones in the ligands. Indeed, platinum has a higher electronegativity than nickel, so it will hold onto its electrons more tightly. Thus, it engages in less backbonding.\par
        To compare \ce{Ni(CO)4^2+} and \ce{Mo(CO)6}, we will focus on the large differences in the metal center and neglect the smaller ones in the ligands. Indeed, although \ce{Pt} is a third row TM and \ce{Mo} is second row, platinum both has a higher oxidation state and comes later in the periodic table (i.e., has a higher electronegativity). Both of these factors decrease its relative backbonding.\par
        There are a number of differences between \ce{Mo(CO)6} and \ce{Fe(CO)5^2-}. To begin with, \ce{Mo} is lower and earlier in the periodic table than \ce{Fe}. However, molybdenum also has a relatively higher oxidation state and more ligands to which it can backbond. As such, it backbonds less to each ligand overall.\par
        Lastly, we must discuss \ce{Fe(CO)5^2-} and \ce{Re(CO)4^3-}. Both have identical $d$ counts, but \ce{Fe} comes later in the periodic table, is higher in the periodic table, has a higher oxidation state, and has more ligands than \ce{Re}. As such, the iron center most certainly backbonds less strongly than the rhenium center.
    \end{proof}
    \item Provide synthetic routes to the following compounds from the specified starting material. Show all coproducts and provide balanced reactions for each step in your synthesis.
    \begin{enumerate}
        \item \ce{(CpMe)Mn(CO)2(py)} from \ce{Mn(CO)5Br}.
        \begin{proof}[Answer]
            \begin{equation*}
                \ce{Mn(CO)5Br ->[LiCpMe][-LiBr, 2CO] (CpMe)Mn(CO)3 ->[py][-CO] (CpMe)Mn(CO)2(py)}
            \end{equation*}
        \end{proof}
        \item \ce{Cp2Zr(CO)2} from \ce{ZrCl4}.
        \begin{proof}[Answer]
            \begin{equation*}
                \ce{ZrCl4 ->[2LiCp][-2LiCl] Cp2ZrCl2 ->[2KC8][-2KCl] Cp2Zr ->[2CO] Cp2Zr(CO)2}
            \end{equation*}
        \end{proof}
        \item \ce{[(C5H4Pr)CpFe]+[PF6]-} from \ce{FeCl2}.
        \begin{proof}[Answer]
            \begin{equation*}
                \ce{FeCl2 ->[LiCp][-LiCl] CpFeCl ->[LiC5H4Pr][-LiCl] (C5H4Pr)CpFe ->[AgPF6][-Ag^0] [(C5H4Pr)CpFe]+[PF6]-}
            \end{equation*}
        \end{proof}
        \item \ce{PhMn(CO)5} from \ce{Mn2(CO)10}.
        \begin{proof}[Answer]
            \begin{equation*}
                \ce{$\frac{1}{2}$ Mn2(CO)10 ->[$\frac{1}{2}$Br2] Mn(CO)5Br ->[LiPh][-LiBr] PhMn(CO)5}
            \end{equation*}
        \end{proof}
        \vspace{-0.5em}
        \item \ce{CpMo(CO)2(PPh3)\{C(=O)Bu\}} from \ce{Mo(CO)6}.
        \begin{proof}[Answer]
            \begin{align*}
                \ce{Mo(CO)6} &\ce{->[LiCp][-3CO]} \ce{Li[CpMo(CO)3] ->[LiBu, PPh3] Li2[CpMo(CO)2(PPh3)\{C(=O)Bu\}]}\\
                &\ce{->[2AgOTf][-2Ag^0, 2LiOTf]} \ce{CpMo(CO)2(PPh3)\{C(=O)Bu\}}
            \end{align*}
        \end{proof}
        \vspace{-0.5em}
        \item \ce{[Cp^*Ta($\mu_2${-}Br)2]2} from \ce{TaBr5}.
        \begin{proof}[Answer]
            \begin{equation*}
                \ce{2TaBr5 ->[2LiCp^*][-2LiBr] 2CpTaBr4 ->[4KC8][-4KBr] 2CpTaBr2 -> [Cp^*Ta($\mu_2${-}Br)2]2}
            \end{equation*}
        \end{proof}
        \vspace{-0.5em}
        \item \ce{(PPh3)2NiClNO} from \ce{NiCl2}.
        \begin{proof}[Answer]
            \begin{equation*}
                \ce{NiCl2 ->[2PPh3] (PPh3)2NiCl2 ->[CoCp2] (PPh3)2NiCl ->[NO] (PPh3)2NiClNO}
            \end{equation*}
        \end{proof}
    \end{enumerate}
    \item Vinyl ethers can bind to metals in an "end-on" or "side-on" manner, as shown below.
    \begin{figure}[h!]
        \centering
        \begin{subfigure}[b]{0.25\linewidth}
            \centering
            \chemfig{{L_\emph{n}}M-\phantom{i}-[6,0.3,,,white]=[2]-[1]OR}
            \caption*{Side-on.}
        \end{subfigure}
        \begin{subfigure}[b]{0.25\linewidth}
            \centering
            \chemfig{{L_\emph{n}}M-OR(-[6,,1]=[7])}
            \caption*{End-on.}
        \end{subfigure}
    \end{figure}
    \begin{enumerate}
        \item Describe the key bonding interactions in each case. Please be specific about which orbitals are involved and illustrate your description with clear diagrams.
        \begin{proof}[Answer]
            ${\color{white}hi}$
            \begin{figure}[H]
                \centering
                \begin{subfigure}[b]{0.4\linewidth}
                    \centering
                    \begin{tikzpicture}[scale=0.7]
                        \footnotesize
                        \node {M};
                        \filldraw [semithick,fill=blz,scale=2,xshift=-1.5mm,yshift=-1.5mm] plot [domain=-pi:-pi/2,smooth,variable=\thta] ({\thta r}:{15^0.5/2*cos(\thta r)*sin(\thta r)});
                        \filldraw [semithick,fill=white,scale=2,xshift=-1.5mm,yshift=1.5mm] plot [domain=-pi/2:0,smooth,variable=\thta] ({\thta r}:{15^0.5/2*cos(\thta r)*sin(\thta r)});
                        \filldraw [semithick,fill=blz,scale=2,xshift=1.5mm,yshift=1.5mm] plot [domain=0:pi/2,smooth,variable=\thta] ({\thta r}:{15^0.5/2*cos(\thta r)*sin(\thta r)});
                        \filldraw [semithick,fill=white,scale=2,xshift=1.5mm,yshift=-1.5mm] plot [domain=pi/2:pi,smooth,variable=\thta] ({\thta r}:{15^0.5/2*cos(\thta r)*sin(\thta r)});
            
                        \begin{scope}[xshift=4.1cm]
                            \filldraw [semithick,fill=blz] (-0.5,0) ellipse (3mm and 5mm);
                            \draw [semithick] (0.5,0) ellipse (3mm and 5mm);
                            \filldraw [semithick,fill=white,xshift=-2mm,yshift=-6mm] plot [domain=-pi:-pi/2,smooth,variable=\thta] ({\thta r}:{15^0.5/2*cos(\thta r)*sin(\thta r)});
                            \filldraw [semithick,fill=blz,xshift=-2mm,yshift=6mm] plot [domain=-pi/2:0,smooth,variable=\thta] ({\thta r}:{15^0.5/2*cos(\thta r)*sin(\thta r)});
                            \filldraw [semithick,fill=white,xshift=2mm,yshift=6mm] plot [domain=0:pi/2,smooth,variable=\thta] ({\thta r}:{15^0.5/2*cos(\thta r)*sin(\thta r)});
                            \filldraw [semithick,fill=blz,xshift=2mm,yshift=-6mm] plot [domain=pi/2:pi,smooth,variable=\thta] ({\thta r}:{15^0.5/2*cos(\thta r)*sin(\thta r)});
                            \node at (0.6,0.4) {\chemfig{=[2]-[1]OR}};
                        \end{scope}
            
                        \node [rotate=-10] at (2.4,1.3) {\normalsize$\cdot\longrightarrow$};
                        \node [rotate=10] at (2.4,-1.3) {\normalsize$\cdot\longrightarrow$};
                        \node [rotate=180] at (2.5,0) {\Large$:\longrightarrow$};
                    \end{tikzpicture}
                    \caption*{Side-on.}
                \end{subfigure}
                \begin{subfigure}[b]{0.4\linewidth}
                    \centering
                    \begin{tikzpicture}[scale=0.7]
                        \footnotesize
                        \node {M};
                        \draw [semithick,scale=1.5,xshift=-2mm] plot [domain=pi/2:3*pi/2,smooth,variable=\thta] ({\thta r}:{15^0.5/4*cos(\thta r)^2});
                        \draw [semithick,scale=1.5,xshift=2mm] plot [domain=-pi/2:pi/2,smooth,variable=\thta] ({\thta r}:{15^0.5/4*cos(\thta r)^2});
                        \filldraw [semithick,fill=blz,scale=1.5,rotate=90,xshift=-2mm] plot [domain=pi/2:3*pi/2,smooth,variable=\thta] ({\thta r}:{15^0.5/4*cos(\thta r)^2});
                        \filldraw [semithick,fill=blz,scale=1.5,rotate=90,xshift=2mm] plot [domain=-pi/2:pi/2,smooth,variable=\thta] ({\thta r}:{15^0.5/4*cos(\thta r)^2});
            
                        \begin{scope}[xshift=4.6cm]
                            \fill [fill=blz] ({6^0.5/4-1.5},0) circle (6^0.5/4);
                            \draw [semithick] ({6^0.5/4-1.5},0) circle (6^0.5/4);
                            \node at (0.2,-0.8) {\chemfig{OR-[6,,1]=[7]}};
                            \draw [semithick] ({6^0.5/4+0.3},0) circle (6^0.5/4);
                        \end{scope}
            
                        \node [rotate=180] at (2.4,0) {\normalsize$:\longrightarrow$};
                    \end{tikzpicture}
                    \caption*{End-on.}
                \end{subfigure}
            \end{figure}
            In the side-on manner, there is $\sigma$ donation and $\pi$ acceptance (the metal $d_{xy}$ orbital is involved). In the end-on case, there is only $\sigma$ donation (the metal $d_{x^2-y^2}$ orbital is involved).
        \end{proof}
        \item What properties of the \ce{L_nM} unit will favor the end-on binding mode? Which properties will favor the side-on mode?
        \begin{proof}[Answer]
            If the metal center is electron rich and soft, the side-on mode will be favored as it gives the metal center the chance to delocalize its electron density through \ce{M -> L} $\pi$ backbonding. If it is electron poor and hard, the end-on mode will be favored due to the dative nature of the bond from the ether and the hardness of the oxygen atom.
        \end{proof}
    \end{enumerate}
    \newpage
    \item Equilibrium constants for the reaction \ce{Co(CO)Br2L2 -> CoBr2L2 + CO} are given below. All values of $K$ were measured at the same temperature.
    \begin{table}[H]
        \centering
        \setlength{\tabcolsep}{1em}
        \begin{tabular}{ll}
            \textbf{L} & $\bm{K}$\\
            \ce{PEt3} & 1\\
            \ce{PEt2Ph} & 2.5\\
            \ce{PEtPh2} & 24.2
        \end{tabular}
    \end{table}
    \begin{enumerate}
        \item Depict the orbital interactions between a phosphine ligand and a metal center.
        \begin{proof}[Answer]
            ${\color{white}hi}$
            \begin{figure}[H]
                \centering
                \begin{subfigure}[b]{0.4\linewidth}
                    \centering
                    \begin{tikzpicture}[scale=0.7]
                        \footnotesize
                        \node {M};
                        \draw [semithick,scale=1.5,xshift=-2mm] plot [domain=pi/2:3*pi/2,smooth,variable=\thta] ({\thta r}:{15^0.5/4*cos(\thta r)^2});
                        \draw [semithick,scale=1.5,xshift=2mm] plot [domain=-pi/2:pi/2,smooth,variable=\thta] ({\thta r}:{15^0.5/4*cos(\thta r)^2});
                        \filldraw [semithick,fill=blz,scale=1.5,rotate=90,xshift=-2mm] plot [domain=pi/2:3*pi/2,smooth,variable=\thta] ({\thta r}:{15^0.5/4*cos(\thta r)^2});
                        \filldraw [semithick,fill=blz,scale=1.5,rotate=90,xshift=2mm] plot [domain=-pi/2:pi/2,smooth,variable=\thta] ({\thta r}:{15^0.5/4*cos(\thta r)^2});
                
                        \begin{scope}[xshift=4.6cm]
                            \fill [fill=blz] ({6^0.5/4-1.5},0) circle (6^0.5/4);
                            \draw [semithick] ({6^0.5/4-1.5},0) circle (6^0.5/4);
                            \node at (0.6,-0.1) {\chemfig{P(-[:60]R_1)(>:[:-30]R_2)(<[:-70]R_3)}};
                            \draw [semithick] ({6^0.5/4+0.3},0) circle (6^0.5/4);
                        \end{scope}
                
                        \node [rotate=180] at (2.4,0) {\normalsize$:\longrightarrow$};
                    \end{tikzpicture}
                    \caption{$\sigma$ donation.}
                \end{subfigure}
                \begin{subfigure}[b]{0.4\linewidth}
                    \centering
                    \begin{tikzpicture}[scale=0.7]
                        \footnotesize
                        \node {M};
                        \filldraw [semithick,fill=blz,scale=2,xshift=-1.5mm,yshift=-1.5mm] plot [domain=-pi:-pi/2,smooth,variable=\thta] ({\thta r}:{15^0.5/2*cos(\thta r)*sin(\thta r)});
                        \filldraw [semithick,fill=white,scale=2,xshift=-1.5mm,yshift=1.5mm] plot [domain=-pi/2:0,smooth,variable=\thta] ({\thta r}:{15^0.5/2*cos(\thta r)*sin(\thta r)});
                        \filldraw [semithick,fill=blz,scale=2,xshift=1.5mm,yshift=1.5mm] plot [domain=0:pi/2,smooth,variable=\thta] ({\thta r}:{15^0.5/2*cos(\thta r)*sin(\thta r)});
                        \filldraw [semithick,fill=white,scale=2,xshift=1.5mm,yshift=-1.5mm] plot [domain=pi/2:pi,smooth,variable=\thta] ({\thta r}:{15^0.5/2*cos(\thta r)*sin(\thta r)});
                
                        \begin{scope}[xshift=4.6cm]
                            \filldraw [semithick,fill=blz,scale=1.5,rotate=-40,xshift=-2mm] plot [domain=pi/2:3*pi/2,smooth,variable=\thta] ({\thta r}:{15^0.5/4*cos(\thta r)^2});
                            \draw [semithick,scale=1.5,rotate=40,xshift=-2mm] plot [domain=pi/2:3*pi/2,smooth,variable=\thta] ({\thta r}:{15^0.5/4*cos(\thta r)^2});
                            \node at (0.6,-0.1) {\chemfig{P(-[:60]R_1)(>:[:-30]R_2)(<[:-70]R_3)}};
                            \filldraw [semithick,fill=blz] (40:{6^0.5/4}) circle (6^0.5/10);
                            \draw [semithick] (-40:{6^0.5/4}) circle (6^0.5/10);
                        \end{scope}
                
                        \node [rotate=-10] at (2.4,1.3) {\normalsize$\cdot\longrightarrow$};
                        \node [rotate=10] at (2.4,-1.3) {\normalsize$\cdot\longrightarrow$};
                    \end{tikzpicture}
                    \caption{$\pi$ acceptance.}
                \end{subfigure}
            \end{figure}
        \end{proof}
        \item Account for this trend in equilibrium constants.
        \begin{proof}[Answer]
            Phenyl groups have stronger electron withdrawing effects than ethyl groups. As such, with them, backbonding is weakened, so it is easier for cobalt to dissociate a carbonyl. Additionally, steric effects from the larger phenyl groups help push the carbonyl off.
        \end{proof}
        \item Of these three phosphines, which will give the cobalt complex with the lowest energy carbonyl stretch in the infrared spectrum? Explain briefly.
        \begin{proof}[Answer]
            \ce{PEt3} because its inductive donation effects are the greatest. The extra electron density that it donates to the cobalt center will increase cobalt's backbonding to the carbonyl ligand, lowering $\nu_{\ce{CO}}$.
        \end{proof}
    \end{enumerate}
    \item This question is meant to guide you through through Prof. Jack Halpern's classic kinetic study of Vaska's complex \ce{IrCl(CO)(PPh3)2} and its \ce{Br} and \ce{I} analogues for the oxidative addition of \ce{H2}, \ce{O2}, and \ce{CH3I} \parencite{bib:HalpernVaska}.
    \begin{enumerate}
        \item Provide three-dimensional drawings for the three \ce{IrX(CO)(PPh3)2} complexes studied for \ce{CH3I} and \ce{H2} reactivity. Classify each ligand as $\sigma$/$\pi$ and as donor/acceptor.
        \begin{proof}[Answer]
            ${\color{white}hi}$
            \begin{figure}[H]
                \centering
                \begin{subfigure}[b]{0.3\linewidth}
                    \centering
                    \chemfig{Ir(-X)(-[1]PPh_3)(-[4]OC)(-[5]Ph_3P)}
                \end{subfigure}
                \begin{subfigure}[b]{0.3\linewidth}
                    \centering
                    \chemfig{Ir(-X)(-[1]PPh_3)(-[2]CH_3)(-[4]OC)(-[5]Ph_3P)(-[6]I)}
                \end{subfigure}
                \begin{subfigure}[b]{0.3\linewidth}
                    \centering
                    \chemfig{Ir(-H)(-[1]PPh_3)(-[2]H)(-[4]X)(-[5]Ph_3P)(-[6]\chembelow{C}{O})}
                \end{subfigure}
            \end{figure}
            Every \ce{X} is a $\sigma,\pi$-donor. \ce{H} and \ce{CH3} are pure $\sigma$-donors. \ce{CO} and \ce{PPh3} are $\sigma$-donors/$\pi$-acceptors.
        \end{proof}
        \newpage
        \item Draw a general mechanism for the oxidative addition of \ce{CH3I} to \ce{IrX(CO)(PPh3)2}. Explain each step in your own words and label species as a nucleophile or an electrophile as appropriate.
        \begin{proof}[Answer]
            ${\color{white}hi}$
            \begin{center}
                \begin{tikzpicture}
                    \begin{scope}[xshift=1cm]
                        \node{\chemfig{\charge{[extra sep=3pt]90=\:}{Ir}(-X)(-[1]PPh_3)(-[4]OC)(-[5]Ph_3P)}};
                    \end{scope}
                    \begin{scope}[xshift=2.5cm]
                        \node at (0,2) {\chemfig{H_3C-I}};
                    \end{scope}
                    \draw [very thick,-latex] [xshift=3.5cm] (0,0) -- (1.3,0);
                    \begin{scope}[xshift=7cm,yshift=0.1cm]
                        \node{
                            \chemleft{[}
                                \chemfig{Ir(-X)(-[1]PPh_3)(-[2]CH_3)(-[4]OC)(-[5]Ph_3P)}
                            \chemright{]^+}
                        };
                    \end{scope}
                    \begin{scope}[xshift=8.5cm]
                        \node at (-0.4,-1) {\charge{[extra sep=3pt]180=\:}{I^-}};
                    \end{scope}
                    \draw [very thick,-latex] [xshift=9.5cm] (0,0) -- (1.3,0);
                    \begin{scope}[xshift=13cm]
                        \node{\chemfig{Ir(-X)(-[1]PPh_3)(-[2]CH_3)(-[4]OC)(-[5]Ph_3P)(-[6]I)}};
                    \end{scope}
            
                    \begin{scope}[semithick,blx,-stealth]
                        \draw (1,0.4) to[out=90,in=180] (1.7,2);
                        \draw (2.6,2.1) to[out=60,in=150] (3.1,2.2);
                        \draw (7.7,-1) to[out=180,in=-90] (7,-0.2);
                    \end{scope}
                \end{tikzpicture}
            \end{center}
            The first step is S\textsubscript{N}2, with Vaska's complex acting as the incoming nucleophile for the backside attack of methyl iodide. As a result of this step, the methyl group of methyl iodide bonds to Vaska's complex and the electrons that had been bonding the iodine to the methyl group get pushed back on to the electrophilic iodine, which then leaves. Newly saturated with electrons, the iodide ion engages in another backside attack, this time on Vaska's complex \emph{trans} to where the methyl added at the open octahedral coordination site.
        \end{proof}
        \item Rank these complexes in increasing order of electron density on the metal. Explain your ranking and support it with experimental data.
        \begin{proof}[Answer]
            \begin{equation*}
                \ce{IrI(CO)(PPh3)2}
                < \ce{IrBr(CO)(PPh3)2}
                < \ce{IrCl(CO)(PPh3)2}
            \end{equation*}
            For the three forms of Vaska's complex (one with each halogen save fluorine), we have $\nu_{\ce{CO}}=1950,1955,\SI{1975}{\per\centi\meter}$ for $\ce{X}=\ce{Cl},\ce{Br},\ce{I}$, respectively. This implies that $\pi$ backbonding follows the order $\ce{Cl}>\ce{Br}>\ce{I}$. It follows that the order of electron density is the same, as shown above.
        \end{proof}
        \item Which \ce{IrX(CO)(PPh3)2} complex reacts with \ce{CH3I} the fastest?
        \begin{proof}[Answer]
            \ce{IrCl(CO)(PPh3)2} reacts with \ce{CH3I} the fastest.
        \end{proof}
        \item Give a general mechanism for the oxidative addition of \ce{H2} to \ce{IrX(CO)(PPh3)2} and explain each step in your own words. Consider our discussion in lecture when drawing the transition state.
        \begin{proof}[Answer]
            ${\color{white}hi}$
            \begin{center}
                \schemestart
                    \chemfig{Ir(-X)(-[1]PPh_3)(-[4]OC)(-[5]Ph_3P)}
                    \arrow{->[\ce{H2}]}
                    \chemleft{[}
                        \chemfig{Ir(-[:60,1.5,,,dashed]H?)(-[1]PPh_3)(-[:120,1.5,,,dashed]H?[,,dashed])(-[:-160]OC)(-[5]Ph_3P)(-[:-20]X)}
                    \chemright{]^\ddagger}
                    \arrow
                    \chemfig{Ir(-H)(-[1]PPh_3)(-[2]H)(-[4]OC)(-[5]Ph_3P)(-[6]X)}
                \schemestop
            \end{center}
            This is a concerted oxidative addition. We know that it is concerted because the rate law is first order in both reactants. In the one step, the \ce{H2} molecule approaches the iridium center, pushing one of the ligands out of the way and forming a transition state where each hydrogen begins to bond with the iridium and the \ce{H-H} bond itself begins to break.
        \end{proof}
        \item Which complex reacts with \ce{H2} the fastest? Considering the transition state you drew above, speculate as to why this is the case.
        \begin{proof}[Answer]
            \ce{IrI(CO)(PPh3)2} reacts with \ce{H2} the fastest. This could be the iridium in this complex has the least electron density, meaning it is most keen for the $\sigma$ donation of the incoming \ce{H2} molecule.
        \end{proof}
    \end{enumerate}
\end{enumerate}




\end{document}